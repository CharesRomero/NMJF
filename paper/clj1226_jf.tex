\documentclass[iop,numberedappendix,apj]{emulateapj}
%\documentclass[iop,numberedappendix,apj]{aastex}
%\documentclass{article}
%\usepackage{emulateapj}
%\pdfoutput=1
\usepackage{color}
\usepackage{amssymb}
%\PassOptionsToPackage{hyphens}{url}
\usepackage{hyperref}
%\usepackage{breakurl}
\def\UrlBreaks{\do\/\do-}
\usepackage{natbib}
\usepackage{graphicx}
\usepackage{epsfig}
%\usepackage[hyphens]{url}
%\usepackage{csquotes}
%\usepackage{lscape}
\usepackage{afterpage}

\usepackage[tbtags]{amsmath}
\usepackage{hyperref,xcolor}
\hypersetup{colorlinks,linkcolor={blue!50!black},citecolor={blue!50!black},urlcolor={blue!80!black}}
\setlength{\tabcolsep}{0.04in} 
%\usepackage{epsfig}
%\usepackage{fullpage}
%\usepackage{hyperref}
%%%%%%%%%%%%%%%%%%%%%%%%%%%%%%%%%%%%%%%%%%%%%%%%%%%%%%%%%%%%%%%%%%%%%%%%%%%%%%%
%%% NOTES ON COMPILING / PRINTING THIS DOCUMENT
%%% do the latex <filename>
%%% dvips -O 0cm,2.0cm <filename.dvi>
%%% dvips -O 0cm,2.0cm clj1226_jf.dvi

\newcommand {\apgt} {\ {\raise-.5ex\hbox{$\buildrel>\over\sim$}}\ }
\newcommand {\aplt} {\ {\raise-.5ex\hbox{$\buildrel<\over\sim$}}\ }
\newcommand{\dmod}{\overrightarrow{d}_{mod}}
\newcommand{\dvec}{\overrightarrow{d}}
\newcommand{\avec}{\overrightarrow{a}}
\newcommand{\asec}{$^{\prime \prime}$}
\newcommand{\asecs}{$^{\prime \prime}\ $}
\newcommand{\amin}{$^{\prime}$}
\newcommand{\amins}{$^{\prime}\ $}
\newcommand{\sigT}{\mbox{$\sigma_{\mbox{\tiny T}}$}}
\newcommand{\Tcmb}{\mbox{$T_{\mbox{\tiny CMB}}$}}
\newcommand{\kB}{\mbox{$k_{\mbox{\tiny B}}$}}
\newcommand{\kBT}{\mbox{$k_{\mbox{\tiny B}}T_{\mbox{\tiny e}}$}}
\newcommand{\nH}{\mbox{$n_{\mbox{\tiny H}}$}}
\newcommand{\NH}{\mbox{$N_{\mbox{\tiny H}}$}}
\newcommand{\LameH}{\mbox{$\Lambda_{e \mbox{\tiny H}}$}}
\newcommand{\Lamee}{\mbox{$\Lambda_{ee}$}}
\newcommand{\rhogas}{\mbox{$\rho_{\mbox{\scriptsize gas}}$}}
\newcommand{\rhotot}{\mbox{$\rho_{\mbox{\scriptsize tot}}$}}
\newcommand{\Mgas}{\mbox{$M_{\mbox{\scriptsize gas}}$}}
\newcommand{\Mtot}{\mbox{$M_{\mbox{\scriptsize tot}}$}}
\newcommand{\Mvir}{\mbox{$M_{\mbox{\scriptsize vir}}$}}
\newcommand{\Yint}{\mbox{$Y_{\mbox{\scriptsize int}}$}}
\newcommand{\Ycyl}{\mbox{$Y_{\mbox{\scriptsize cyl}}$}}
\newcommand{\Ysph}{\mbox{$Y_{\mbox{\scriptsize sph}}$}}
\newcommand{\fgas}{\mbox{$f_{\mbox{\scriptsize gas}}$}}
\newcommand{\LCDM}{\mbox{$\Lambda$CDM}}
\newcommand{\Pe}{\mbox{$P_{\mbox{\scriptsize e}}$}}
\newcommand{\msun}{$M_{\odot}$}
\newcommand{\etal}{{\it et al.}}
\newcommand{\mJy}{\,{\rm mJy} }
\newcommand{\um}{\,\mu {\rm m} }
\newcommand{\mJySr}{\,{\rm MJy/Sr} }
\newcommand{\mJyBm}{\,{\rm mJy/Bm} }
\newcommand{\mK}{\,{\rm mK} }
\newcommand{\K}{\,{\rm K} }
\newcommand{\uJy}{\,{\rm \mu Jy} }
\newcommand{\uK}{\,{\rm \mu K} }
\newcommand{\kHz}{\, {\rm kHz} }
\newcommand{\eg}{{\it e.g.}}
\newcommand{\ie}{{\it i.e.}}
\newcommand{\etc}{{\it etc.}}
\newcommand{\aips}{{\tt AIPS++}}
\newcommand{\nusp}{\nu_{sp}}
\newcommand{\ghz}{{\, \rm GHz}}
\newcommand{\db}{{\, \rm dB}}
\newcommand{\degsqr}{\, {\rm deg^2}}
\newcommand{\Tew}{\mbox{$T_{\mathrm{ew}}$}}
\newcommand{\Tspec}{\mbox{$T_{\mathrm{spec}}$}}
\newcommand{\chandra}{{\it Chandra}}
\newcommand{\asca}{{ASCA}}
\newcommand{\wmap}{{WMAP}}
\newcommand{\rosat}{{ROSAT}}
\newcommand{\xmm}{{XMM-{\it Newton}}}
\newcommand{\planck}{{\it Planck}}
\newcommand{\hubble}{{\it Hubble}}
\newcommand{\rxj}{RX J1347.5-1145}
\newcommand{\clj}{CL J1226.9+3332}
\newcommand{\macsa}{MACS J0647.7+7015}
\newcommand{\macsb}{MACS J1206.2-0847}
\newcommand{\macsc}{MACS J0717.5+3745}
\newcommand{\macsd}{MACS J1423.8+2404}
\newcommand{\macse}{MACS J0329.6-0211}
\newcommand{\macsf}{MACS J0429-0253}
\newcommand{\macsg}{MACS J0744.9+3927}
\newcommand{\macsh}{MACS J1149+2223}
\newcommand{\macsi}{MACS J1115+0130}
\newcommand{\Tx}{\mbox{$T_{\mbox{\tiny X}}$}}
\newcommand{\te}{\mbox{$T_{\mbox{\tiny e}}$}}
\newcommand{\mec}{\mbox{$m_{\mbox{\tiny e}} c^2$}}
\newcommand{\dene}{\mbox{$n_{\mbox{\tiny e}}$}}
\newcommand{\denesq}{\mbox{$n^2_{\mbox{\tiny e}}$}}
\newcommand{\yx}{\mbox{$Y_{\mbox{\tiny X}}$}}
\newcommand{\ysze}{\mbox{$Y_{\mbox{\tiny tSZE}}$}}
\newcommand{\sx}{\mbox{$S_{\mbox{\tiny X}}$}}
\newcommand{\Itsz}{\mbox{$I_{\mbox{\tiny tSZE}}$}}
\newcommand{\Iksz}{\mbox{$I_{\mbox{\tiny kSZE}}$}}
\newcommand{\chisq}{\mbox{$\chi^{2}$}}
\newcommand{\chired}{\mbox{$\chi^{2}_{red}$}}

\defcitealias{arnaud2010}{A10}
\defcitealias{cavagnolo2009}{C09}
\defcitealias{bulbul2010}{B10}
\defcitealias{vikhlinin2006}{V06}

\newcommand{\quotes}[1]{``#1''}

\slugcomment{}
\shortauthors{Romero \etal}
\shorttitle{Joint SZE Map Fitting with MUSTANG and Bolocam}
%altaffilmark{#}

\begin{document}

\title{Joint Fitting of CLJ1226.9}
\author{
%  Order TBD ,
  Charles E. Romero\altaffilmark{1,2,3},
  Matthew McWilliam \altaffilmark{4},
  Juan Macias-Perez \altaffilmark{4},
  NIKA Collaboration \altaffilmark{4}
} 
\date{\today}

%%%%%%%%%%%%%%%%%%%%%%%%%%%%%%%%%%%%%%%%%%%%%%%%%%%%%%%%%%%%%%%%%%%%%%%%%%%%%%%
\altaffiltext{1}{Institut de Radioastronomie Millim\'{e}trique
300 rue de la Piscine, Domaine Universitaire
38406 Saint Martin d'H\`{e}res, France} 
\altaffiltext{2}{Department of Astronomy, University of Virginia,
  P.O. Box 400325, Charlottesville, VA 22904, USA}
\altaffiltext{3}{National Radio Astronomy Observatory, 520 Edgemont Rd.,
Charlottesville, VA 22904, USA}
\altaffiltext{4}{Department of Physics, Math, and Astronomy,
  California Institute of Technology, Pasadena, CA 91125, USA}
\altaffiltext{5}{Author contact: \email{romero@iram.fr}}
%\begin{document}

%%%%%%%%%%%%%%%%%%%%%%%%%%%%%%%%%%%%%%%%%%%%%%%%%%%%%%%%%%%%%%%%%%%%%%%%%%%%%%%

\begin{abstract}
We present pressure profiles of galaxy clusters determined from high resolution 
\end{abstract}

\keywords{galaxy clusters: individual: CLJ1226.9+3352}

\maketitle

%%%%%%%%%%%%%%%%%%%%%%%%%%%%%%%%%%%%%%%%%%%%%%%%%%%%%%%%%%%%%%%%%%%%%%%%%%%%%%%
\section{Introduction}
\label{sec:intro}
%%%%%%%%%%%%%%%%%%%%%%%%%%%%%%%%%%%%%%%%%%%%%%%%%%%%%%%%%%%%%%%%%%%%%%%%%%%%%%%

%\textcolor{red}{Trimming my own notes now.}

Galaxy clusters are the largest gravitationally bound objects in the universe and thus serve as ideal cosmological probes 
and astrophysical laboratories. Within a galaxy cluster, the gas in the intracluster medium (ICM) constitutes 90\% of the
baryonic mass \citep{vikhlinin2006b} and is directly observable in the X-ray due to bremsstrahlung emission. 
At millimeter and sub-millimeter wavelengths, the ICM is observable via the Sunyaev-Zel'dovich (SZ) effect 
\citep{sunyaev1972}: the inverse Compton scattering of cosmic microwave background (CMB) photons off of
the hot ICM electrons. The thermal SZ is observed as an intensity decrement relative to the CMB at wavelengths longer 
than $\sim$1.4 mm (frequencies less than $\sim$220 GHz). The amplitude of the thermal SZ is proportional to the integrated
line-of-sight electron pressure, and is often parameterized as Compton $y$: $y = (\sigma_T / m_e c^2) \int P_e dl$, where
$\sigma_T$ is the Thomson cross section, $m_e$ is the electron mass, $c$ is the speed of light, and $P_e$ is the electron
pressure.
%At longer radio wavelengths, if relativistic electrons are present, parts of the ICM may emit synchrotron emission.

%\textcolor{red}{[I need to revise this paragraph.]}
Cosmological constraints derived from galaxy cluster samples are generally limited by the accuracy of mass calibration of 
galaxy clusters \citep[e.g.][]{hasselfield2013, reichardt2013}, which is often calculated via a scaling relation with 
respect to some integrated observable quantity. Scatter in the scaling relations will then depend on the regularity of 
clusters and the adopted integration radius of the clusters. Determining pressure profiles of galaxy clusters provides an 
assessment of the relative impact and frequency of various astrophysical processes in the ICM and can refine the choice of 
extent of galaxy clusters to reduce the scatter in scaling relations.

In the core of a galaxy cluster, some observed astrophysical processes include shocks and cold fronts 
\citep[e.g.][]{markevitch2007}, sloshing \citep[e.g.][]{fabian2006}, and X-ray cavities \citep{mcnamara2007}. 
It is also theorized that helium sedimentation should occur, most noticeably in low redshift, dynamically-relaxed 
clusters \citep{abramopoulos1981, gilfanov1984} 
and recently the expected helium enhancement via sedimentation has been numerically simulated \citep{peng2009}. 
This would result in an offset between X-ray and SZ derived pressure profiles if not accounted for correctly.

At large radii ($R \gtrsim R_{500}$),\footnote{$R_{500}$ is the radius at which the enclosed average mass density is 
500 times the critical density, $\rho_c(z)$, of the universe} equilibration timescales are longer, accretion is ongoing, 
and hydrostatic equilibrium (HSE) is a poor approximation. Several numerical simulations show that the fractional contribution
from non-thermal pressure increases with radius \citep{shaw2010,battaglia2012,nelson2014}. 
For all three studies, non thermal pressure fractions between 15\% and 30\% are found at ($R \sim R_{500}$)
for redshifts $0 < z < 1$. Additionally, clumping is expected to increase with radius \citep{kravtsov2012}, and is expected to
increase the scatter of pressure profiles at large radii \citep{nagai2011} as well as biasing X-ray derived gas density high,
and thus X-ray derived thermal pressure low \citep{battaglia2015}.

However, the intermediate region, between the core and outer regions of the galaxy cluster, 
is often the best region to apply self-similar scaling relations derived from HSE to describe simulations 
and observations \citep[e.g.][]{kravtsov2012}. Moreover, both simulations and observations find low
cluster-to-cluster scatter in pressure profiles within this intermediate radial range \citep[e.g.][]{borgani2004,
nagai2007,arnaud2010,bonamente2012,planck2013a,sayers2013}.

There are many existing facilities capable of making SZ observations, but most have
angular resolutions of one arcminute or larger. In recent years, the SZ community has adopted the pressure profile
presented in \citet{arnaud2010} (hereafter, A10), who derive their pressure profiles from X-ray data from the 
REXCESS sample of 31 nearby ($z < 0.2$) clusters out to $R_{500}$ and numerical simulations for larger radii. The
adoption of the A10 pressure profile has allowed for the extraction of an integrated observable quantity which,
via scaling relations, can then be used to determine the mass of the clusters. In this paper, we use high resolution
SZ data to test the validity of this pressure profile in our sample of 14 clusters at intermediate redshifts.

The MUSTANG camera \citep{dicker2008}
on the 100 meter Robert C. Byrd Green Bank Telescope \citep[GBT, ][]{jewell2004} with its angular resolution of 9\asec 
(full-width, half-maximum FWHM) is one of only a few SZ effect instruments with sub-arcminute resolution.
However, MUSTANG's instantaneous field of view (FOV) limits its sensitivity to scales larger than $1$\amin. 
To probe a wider range of scales we complement our MUSTANG data with SZ data from Bolocam \citep{glenn1998}. 
Bolocam is a 144-element bolometer
array on the Caltech Submillimeter Observatory (CSO) with a beam FWHM of 58\asecs at 140 GHz and circular FOV with 8\amins 
diameter, which is well matched to the angular size of $R_{500}$ ($\sim 4$\amin) of the clusters in our sample. 

This paper is organized as follows. In Section~\ref{sec:obs} we describe the MUSTANG and Bolocam observations and reduction. 
In Section~\ref{sec:jointfitting} we review the method used to jointly fit pressure profiles to MUSTANG and Bolocam data. We
present results from the joint fits in Section~\ref{sec:pp_constraints} and compare our results to X-ray derived pressures 
in Section~\ref{sec:xray_comp}. 
Throughout this paper we assume a $\Lambda$CDM cosmology with $\Omega_m = 0.3$, $\Omega_{\lambda} = 0.7$, and $H_0 = 70$ 
km s$^{-1}$ Mpc$^{-1}$.
%consistent with the 9-year \emph{Wilkinson Microwave Anisotropy Probe} (WMAP) results reported in \cite{hinshaw2013}.

%%%%%%%%%%%%%%%%%%%%%%%%%%%%%%%%%%%%%%%%%%%%%%%%%%%%%%%%%%%%%%%%%%%%%%%%%%%%%%%
\section{Observations and Data Reduction}
\label{sec:obs}
%%%%%%%%%%%%%%%%%%%%%%%%%%%%%%%%%%%%%%%%%%%%%%%%%%%%%%%%%%%%%%%%%%%%%%%%%%%%%%%

\subsection{Sample}

\begin{deluxetable*}{lllllllllll}
\tabletypesize{\scriptsize}
\tablecolumns{10}
\tablewidth{0pt} 
\tablecaption{CLASH cluster properties \label{tbl:cluster_properties}}
\tablehead{ 
    \colhead{Cluster} & \colhead{$z$} & \colhead{$M_{500}$} & \colhead{$P_{500}$} & \colhead{$R_{500}$} & \colhead{$T_x^a$} 
              & \colhead{$T_x^b$} & \colhead{$T_{mg}$} & \colhead{Dynamical} & \colhead{$\Delta r_0$} \\
              \colhead{}  & \colhead{} & \colhead{($10^{14} M_{\odot}$)} & \colhead{(keV/cm$^{3}$)} & \colhead{(kpc)} & 
              \colhead{(keV)} & \colhead{(keV)} & \colhead{(keV)} & \colhead{state} & \colhead{(\asec)}
}
\startdata

    \textbf{CLJ1226}     & 0.888 & 7.8  & 0.01184   & 1000   & 12.0 & 11.7 & 8.39 & --      & 15.3  \\
\enddata
\tablecomments{$z$, $M_{500}$, and $T_X^a$ are taken from \citet{mantz2010}:  $T_X^a$ is calculated from a 
  single spectrum over $0.15 R_{500} < r < R_{500}$ for each cluster. $T_X^b$ is from \citet{morandi2015},
  and is calculated over $0.15 R_{500} < r < 0.75 R_{500}$.  $T_{mg}$ is a fitted gas mass weighted temperature,
  determined by fitting the ACCEPT \citep{cavagnolo2009} temperature profiles to the profile found in
  \citet{vikhlinin2006}. The dynamical states: cool core (CC) and disturbed (D) are taken from (and defined in)
  \citet{sayers2013}. The bolded clusters are the 14 clusters in our sample.
  $\Delta r_0$ denotes the offset between the ACCEPT and Bolocam centroids. }
\end{deluxetable*}

\begin{deluxetable*}{lllllllllll}
\tabletypesize{\footnotesize}
\tablecolumns{10}
\tablewidth{0pt} 
\tablecaption{Bolocam and MUSTANG observational properties. \label{tbl:cluster_obs}}
\tablehead{ 
    \colhead{Cluster} & \colhead{$z$} & \colhead{R.A.} & \colhead{Decl.} & 
              \colhead{$T_{obs,B}$} & \colhead{Noise$_{B}$} & \colhead{A10$_{B}$} & 
              \colhead{$T_{obs,M}$} & \colhead{Noise$_{M}$} & \colhead{A10$_{M}$}    \\
            & \colhead{} & \colhead{(J2000)} & \colhead{(J2000)} &  
              \colhead{(hours)} & \colhead{$\mu K_{CMB}$-amin} & \colhead{($\sigma$)} &
              \colhead{(hours)} & \colhead{$\mu$Jy/bm}        & \colhead{($\sigma$)}
}
\startdata
   \textbf{CLJ1226}     & 0.888 & 12:26:57.9 & +33:32:49 & 11.8 & 22.9 & 13.7 & 4.9  & 85.6 & 9.43 
\enddata
\tablecomments{Subscripts $_{B}$ and $_{M}$ denote Bolocam and MUSTANG properties respectively. Noise$_{B}$
  and $T_{obs,B}$ are those reported in \citet{sayers2013}. Noise$_{M}$ is calculated on MUSTANG maps with 
  $10$\asecs smoothing, in the central arcminute. $T_{obs}$ are the integration times (on source) for the 
  given instruments. A10$_B$ and A10$_M$ values indicate the significance (in $\sigma$) of $P_0$ when we
  fit a spherical A10 \citep{arnaud2010} profile (see Section~\ref{sec:bulk_ICM}).}
\end{deluxetable*}

%%%%%%%%%%%%%%%%%%%%%%%%%%%%%%%%%%%%%%%%%%%%%%%%%%%%%%%%%%%%%%%%%%%%%%%%%%%%%%%
\subsection{NIKA Observations and Reduction}
\label{sec:nikaobs}

%%% How was the transfer function calculated 

%%%%%%%%%%%%%%%%%%%%%%%%%%%%%%%%%%%%%%%%%%%%%%%%%%%%%%%%%%%%%%%%%%%%%%%%%%%%%%%
\subsection{MUSTANG Observations and Reduction}
\label{sec:musobs}

MUSTANG is a 64 pixel array of Transition Edge Sensor (TES) bolometers arranged in an $8 \times 8$ array
located at the Gregorian focus on the 100 m GBT. Operating at 90 GHz (81--99~GHz),
MUSTANG has an angular resolution of 9\asec and pixel spacing of 0.63$f \lambda$ resulting in a FOV
of 42\asec. More detailed information about the instrument can be found in \citet{dicker2008}.

Our observations and data reduction are described in detail in \citet{romero2015a}, and we briefly review them
here. Absolute flux calibrations are based on the planets Mars, Uranus, or Saturn, nebulae, or the star Betelgeuse 
($\alpha_{Ori}$).

%%% How was the transfer function calculated 
\footnote{MUSTANG data is publicaly available at 
\href{http://irsa.ipac.caltech.edu/data/Planck/release\_2/ancillary-data/bolocam/}{Hi}}

%%%%%%%%%%%%%%%%%%%%%%%%%%%%%%%%%%%%%%%%%%%%%%%%%%%%%%%%%%%%%%%%%%%%%%%%%%%%%%%
\subsection{Bolocam Observations and Reduction}
\label{sec:bolocamobs}

Bolocam is a 144-element camera that was a facility instrument on the Caltech Submillimeter Observatory (CSO) from
2003 until 2012. Its field of view is 8\amins in diameter, and at 140 GHz it has a resolution of 58\asecs FWHM
(\citet{glenn1998,haig2004}). The clusters were observed with a Lissajous pattern that results in a tapered
coverage dropping to 50\% of the peak value at a radius of roughly 5\amin, and to 0 at a radius of 10\amin.
The Bolocam maps used in this analysis are $14\arcmin \times 14\arcmin$. The Bolocam data 
\footnote{Bolocam data is publicaly available at 
\href{http://irsa.ipac.caltech.edu/data/Planck/release\_2/ancillary-data/bolocam/}
{http://irsa.ipac.caltech.edu/data/Planck/release\_2/ancillary-data/} 
\href{http://irsa.ipac.caltech.edu/data/Planck/release\_2/ancillary-data/bolocam/}{bolocam/}}
are the same as those used in \citet{czakon2015} and \citet{sayers2013}; the details of the reduction are 
given therein, along with \citet{sayers2011}. 
The reduction and calibration is similar to that used for MUSTANG, and Bolocam achieves a 
5\% calibration accuracy and 5\asecs pointing accuracy.

%%% How was the transfer function calculated 

%%%%%%%%%%%%%%%%%%%%%%%%%%%%%%%%%%%%%%%%%%%%%%%%%%%%%%%%%%%%%%%%%%%%%%%%%%%%%%%
\section{Joint Map Fitting Technique}
\label{sec:jointfitting}
%%%%%%%%%%%%%%%%%%%%%%%%%%%%%%%%%%%%%%%%%%%%%%%%%%%%%%%%%%%%%%%%%%%%%%%%%%%%%%%

\subsection{Overview}
\label{sec:jf_overview}

The joint map fitting technique used in this paper is described in detail in \citet{romero2015a}. We review
it briefly here. The general approach follows that of a least squares fitting procedure, which assumes that
we can make a model map as a linear combination of model components. 

This linear combination can be written as:
\begin{equation}
  \vec{d}_m = \mathbf{A} \vec{a}_m,
\end{equation}
where $d_m$ is the total model, each column in $\mathbf{A}$ is a filtered model component (Section~\ref{sec:components}), 
and $\vec{a}_m$ is 
an array of amplitudes of the components. There are up to four types of components for which fit: 
a bulk component, point source(s), residual component(s), and a mean level. Of these, we produce a
sky model for the bulk component and point source to be filtered. The residual component is calculated 
directly as a filtered component.

We wish to fit $\vec{d}_m$ to our data, $\vec{d}$, and allow for a calibration offset between Bolocam and
MUSTANG data. We therefore define our data vector as:
\begin{equation}
  \vec{d} = [ \vec{d}_{B}, k \vec{d}_{M}, k ] ,
\end{equation}
where $\vec{d}_{B}$ is the Bolocam data, $\vec{d}_{M}$ is the MUSTANG data, and $k$ is the calibration offset of
MUSTANG relative to Bolocam, with an RMS uncertainty of 11.2\%.

We use the $\chi^2$ statistic as our goodness of fit:
\begin{equation}
  \chi^2 = (\overrightarrow{d} - \overrightarrow{d}_m)^T \mathbf{N}^{-1} (\overrightarrow{d} - \overrightarrow{d}_m),
\end{equation}
where $\mathbf{N}$ is the covariance matrix; however, because we wish to fit for $k$ in addition to the 
amplitude of model components, we no longer have
completely linearly independent variables, and thus we employ MPFIT \citep{markwardt2009} to solve for these
variables.

\subsection{Preprocessing}

\subsection{Mean Level}
\label{sec:mean_level}

%%% JSayers: Bolocam too? Yes...if done, both are done. But I removed the mean level fit in the simple sense.
%%% I don´t think I fit out a Bolocam mean level, because it was so low.
Similar to \citet{czakon2015}, we wish to account for a mean level (signal offset) in the MUSTANG maps.
We do not wish to fit for a mean level simultaneously as a bulk component given the degeneracies. Therefore,
to determine the mean level independent of the other components, we create a MUSTANG noise map
% from time-flipped TOD 
and calculate the mean within the inner arcminute for each cluster. This mean is then subtracted before 
the other components are fit. 

\subsubsection{Point Sources}
\label{sec:ptsrcs}

For MUSTANG, point sources are treated in the same manner as in \citet{romero2015a}. 
A point source is identified by NIKA \citep{adam2015} in CLJ1226, which is posited to be a submillimeter galaxy (SMG) 
behind the cluster. 

%%%%%%%%%%%%%%%%%%%%%%%%%%%%%%%%%%%%%%
For the Bolocam image

\subsubsection{Centroid}

The default centroids used when gridding our bulk ICM component are the ACCEPT centroids. Given the offsets
between ACCEPT and Bolocam centroids (Table~\ref{tbl:cluster_properties}), we perform a second set of
fits where we grid the bulk ICM component using the Bolocam centroids. The ACCEPT centroid are taken to be the
X-ray peaks unless their centroiding algorithm produced a centroid more than 70 kpc from the X-ray peak, in which
case they adopt that centroid \citep{cavagnolo2008a}. 
%We do not find significant changes in
%the fitted gNFW parameters (Section~\ref{sec:pp_constraints}), 

\subsection{Parametric Fits: gNFW}
\label{sec:parfits}

The cluster is taken to be a 
spherically symmetric 3D electron pressure profile as parameterized by a generalized Navarro, Frenk,
and White profile \citep[hereafter, gNFW][]{navarro1997,nagai2007}:
\begin{equation}
  \Tilde{P} = \frac{P_0}{(C_{500} X)^{\gamma} [1 + (C_{500} X)^{\alpha}]^{(\beta - \gamma)/\alpha}}
\end{equation}
where $X = R / R_{500}$, and $C_{500}$ is the concentration parameter; one can also write ($C_{500} X$) as
($R / R_s$), where $R_s = R_{500}/C_{500}$. $\Tilde{P}$ is the electron pressure in units of the characteristic
pressure $P_{500}$. This pressure profile is integrated along the line of sight to produce 
a Compton $y$ profile, given as 
\begin{equation}
  y(r) = \frac{P_{500} \sigma_{T}}{m_e c^2} \int_{-\infty}^{\infty} \Tilde{P}(r,l) dl
\end{equation}
where $R^2 = r^2 + l^2$, $r$ is the projected physical radius, and $l$ is the distance from the center of the cluster
along the line of sight. Once integrated, $y(r)$ is gridded as $y(\theta)$ ($\theta$ being the angular radius) 
and is realized as two maps with
the same astrometry as the MUSTANG and Bolocam data maps (pixels of 1\asecs and 20\asecs on a side, respectively). 
%From here, we produce two model maps: one for Bolocam and one for MUSTANG. 
In each case, we convolve the Compton $y$ map by the appropriate beam shape. For Bolocam we use a Gaussian with FWHM
$= 58$\asec, and for MUSTANG we use the double Gaussian as determined in \citet{romero2015a}.

\subsubsection{Parameter Space}
\label{sec:param_space}

As in \citet{romero2015a}, we fix MUSTANG's centroid, but allow Bolocam's pointing to vary by $\pm 10$\asecs 
in RA and Dec with a prior on Bolocam's radial pointing accuracy with an RMS uncertainty of $5$\asec. Our 
approach to find the absolute calibration offset between Bolocam and MUSTANG is the same as in
\citet{romero2015a} (see also Section~\ref{sec:jf_overview}). 

In \citet{romero2015a}, we performed a grid search over $\gamma$ and $C_{500}$, marginalizing over $P_0$,
where $\alpha$ and $\beta$ are fixed to values determined from previous studies. We find that fixing
$\alpha$ and $\beta$ at different values, the pressure profiles are in very 
good agreement with one another and that the 
differences in $\chi^2$ values between these fits is not significant. Thus, for our fits 
we assume the values of $\alpha$ and $\beta$ given in \citetalias{arnaud2010}.

We search over $0 < \gamma < 1.3$ in steps of $\delta \gamma = 0.1$, and over
$0.1 < C_{500} < 3.3$ in steps of $\delta C_{500} = 0.1$. To create models in finer steps than $\delta \gamma$ 
and $\delta C_{500}$, we interpolate filtered model maps from nearest neighbors from the grid of original 
filtered models. 

\subsection{Non Parametric Fits: radial pressure bins}

\subsection{Geometric Deprojection}

%%%%%%%%%%%%%%%%%%%%%%%%%%%%%%%%%%%%%%%%%%%%%%%%%%%%%%%%%%%%%%%%%%%%%%%%%%%%%%%
\section{SZ Pressure Profile Constraints}
\label{sec:pp_constraints}
%%%%%%%%%%%%%%%%%%%%%%%%%%%%%%%%%%%%%%%%%%%%%%%%%%%%%%%%%%%%%%%%%%%%%%%%%%%%%%%

We have constrained the gNFW parameters $P_0$, $C_{500}$, and $\gamma$.

%%%%%%%%%%%%%%%%%%%%%%%%%%%%%%%%%%%%%%%%%%%%%%%%%%%%%%%%%%%%%%%%%%%%%%%%%%%%%%%%%%%%%%%%%%%%%%%%%%%%%%%%%%%
%%%                                                SOME FIGURES                                         %%%
%%%%%%%%%%%%%%%%%%%%%%%%%%%%%%%%%%%%%%%%%%%%%%%%%%%%%%%%%%%%%%%%%%%%%%%%%%%%%%%%%%%%%%%%%%%%%%%%%%%%%%%%%%%

%%%%%%%%%%%%%%%%%%%%%%%%%%%%%%%%%%%%%%%%%%%%%%%%%%%%%%%%%%%%%%%%%%%%%%%%%%%%%%%
\subsection{Systematics}
\label{sec:systematics}
%%%%%%%%%%%%%%%%%%%%%%%%%%%%%%%%%%%%%%%%%%%%%%%%%%%%%%%%%%%%%%%%%%%%%%%%%%%%%%%

%%%%%%%%%%%%%%%%%%%%%%%%%%%%%%%%%%%%%%%%%%%%%%%%%%%%%%%%%%%%%%%%%%%%%%%%%%%%%%%%%%%%%%%%%%%%%%%%%%%%%%%%%%%
%%%                  Table of just the ensembles (and A10 to compare against)                           %%%
%%%%%%%%%%%%%%%%%%%%%%%%%%%%%%%%%%%%%%%%%%%%%%%%%%%%%%%%%%%%%%%%%%%%%%%%%%%%%%%%%%%%%%%%%%%%%%%%%%%%%%%%%%%

%%%%%%%%%%%%%%%%%%%%%%%%%%%%%%%%%%%%%%%%%%%%%%%%%%%%%%%%%%%%%%%%%%%%%%%%%%%%%%%%%%%%%%%%%%%%%%%%%%%%%%%%%%%
%%%    END OF THAT TABLE!    NOW LET'S PUT IN THE TABLE WE WANT INSTEAD                                 %%%
%%%%%%%%%%%%%%%%%%%%%%%%%%%%%%%%%%%%%%%%%%%%%%%%%%%%%%%%%%%%%%%%%%%%%%%%%%%%%%%%%%%%%%%%%%%%%%%%%%%%%%%%%%%

%%%%%%%%%%%%%%%%%%%%%%%%%%%%%%%%%%%%%%%%%%%%%%%%%%%%%%%%%%%%%%%%%%%%%%%%%%%%%%%%%%%%%%%%%%%%%%%%%%%%%%%%%%%
%%%                                             END OF THAT TABLE!                                      %%%
%%%%%%%%%%%%%%%%%%%%%%%%%%%%%%%%%%%%%%%%%%%%%%%%%%%%%%%%%%%%%%%%%%%%%%%%%%%%%%%%%%%%%%%%%%%%%%%%%%%%%%%%%%%

%%%%%%%%%%%%%%%%%%%%%%%%%%%%%%%%%%%%%%%%%%%%%%%%%%%%%%%%%%%%%%%%%%%%%%%%%%%%%%%%%%%%%%%%%%%%%%%%%%%%%%%%%%%
%%%                                                NEXT SECTION                                         %%%
%%%%%%%%%%%%%%%%%%%%%%%%%%%%%%%%%%%%%%%%%%%%%%%%%%%%%%%%%%%%%%%%%%%%%%%%%%%%%%%%%%%%%%%%%%%%%%%%%%%%%%%%%%%

\section{Integrated Compton $Y$ Scaling Relations}


%%%%%%%%%%%%%%%%%%%%%%%%%%%%%%%%%%%%%%%%%%%%%%%%%%%%%%%%%%%%%%%%%%%%%%%%%%%%%%%%%%%%%%%%%%%%%%%%%%%%%%%%%%%
%%%                                                SOME FIGURES                                         %%%
%%%%%%%%%%%%%%%%%%%%%%%%%%%%%%%%%%%%%%%%%%%%%%%%%%%%%%%%%%%%%%%%%%%%%%%%%%%%%%%%%%%%%%%%%%%%%%%%%%%%%%%%%%%

%%%%%%%%%%%%%%%%%%%%%%%%%%%%%%%%%%%%%%%%%%%%%%%%%%%%%%%%%%%%%%%%%%%%%%%%%%%%%%%%%%%%%%%%%%%%%%%%%%%%%%%%%%%%%%%%
%%%%%%%%%%%%%%%%%%%%%%%%                    CONCLUSIONS!!!                           %%%%%%%%%%%%%%%%%%%%%%%%%%%
%%%%%%%%%%%%%%%%%%%%%%%%%%%%%%%%%%%%%%%%%%%%%%%%%%%%%%%%%%%%%%%%%%%%%%%%%%%%%%%%%%%%%%%%%%%%%%%%%%%%%%%%%%%%%%%%

\section{Conclusions}
\label{sec:conclusions}

We developed an algorithm to jointly fit gNFW pressure profiles to clusters observed via the SZ
effect with MUSTANG and Bolocam. We apply this algorithm to 14 clusters and find the profiles are 
consistent with a universal pressure profile found in \citet{arnaud2010}. Specifically, the 
pressure profile is of the form:
\begin{equation*}
  \Tilde{P_e} = \frac{P_0}{(C_{500} X)^{\gamma} [1 + (C_{500} X)^{\alpha}]^{(\beta - \gamma)/\alpha}},
%  \label{eqn:norm_gnfw}
\end{equation*}
where we fixed $\alpha$ and $\beta$ to values found in \citet{arnaud2010}. 

\section*{Acknowledgements}

Support for CR was provided through the Grote Reber Fellowship at NRAO. Support for CR, PK, and AY was 
provided by the Sudent Observing Support (SOS) program. Support for TM is provided by the National Research 
Council Research Associateship Award at the U.S.\ Naval Research Laboratory. Basic research in radio 
astronomy at NRL is supported by 6.1 Base funding. JS was partially supported by a
Norris Foundation CCAT Postdoctoral Fellowship and by NSF/AST-1313447.

The National Radio Astronomy Observatory is a facility of the National Science Foundation which is operated
under cooperative agreement with Associated Universities, Inc. The GBT observations used in this paper were
taken under NRAO proposal IDs GBT/08A-056, GBT/09A-052, GBT/09C-020, GBT/09C-035, GBT/09C-059, GBT/10A-056, 
GBT/10C-017, GBT/10C-026, GBT/10C-031, GBT/10C-042, GBT/11A-001, and GBT/11B-009 and VLA/12A-340.
We  thank the GBT operators Dave Curry, Greg Monk, Dave Rose, Barry Sharp, and Donna Stricklin for their
assistance. 

The Bolocam observations presented here were obtained form the Caltech Submillimeter Observatory, which,
when the data used in this analysis were taken, was operated by the California Institute of Technology under
cooperative agreement with the National Science Foundation. Bolocam was constructed and commissioned using funds
from NSF/AST-9618798, NSF/AST-0098737, NSF/AST-9980846, NSF/AST-0229008, and NSF/AST-0206158. Bolocam observations
were partially supported by the Gordon and Betty Moore Foundation, the Jet Propulsion Laboratory Research and
Technology Development Program, as well as the National Science Council of Taiwan grant NSC100-2112-M-001-008-MY3.


%%%%%%%%%%%%%%%%%%%%%%%%%%%%%%%%%%%%%%%%%%%%%%%%%%%%%%%%%%%%%%%%%%%%%%%%%%%%%%%%%%%%%%%%%%%%%%%%%%%%%%%%%%%%%%%%
%%%%%%%%%%%%%%%%%%%%%%%%                       APPENDIX                              %%%%%%%%%%%%%%%%%%%%%%%%%%%
%%%%%%%%%%%%%%%%%%%%%%%%%%%%%%%%%%%%%%%%%%%%%%%%%%%%%%%%%%%%%%%%%%%%%%%%%%%%%%%%%%%%%%%%%%%%%%%%%%%%%%%%%%%%%%%%

\appendix

%%%%%%%%%%%%%%%%%%%%%%%%%%%%%%%%%%%%%%%%%%%%%%%%%%%%%%%%%%%%%%

\subsection{CLJ 1226 (z=0.89)}
\label{sec:results_clj1226}

%%%%%%%%%%%%%%%%%%%%%%%%%%%%%%%%%%%%%%%%%%%%%%%%%%%%%%%%%%%%%%

CLJ 1226 is a well studied high redshift cluster \citep[e.g.][]{mroczkowski2009,bulbul2010,adam2015}. 
\citet{adam2015} find a point source at RA 12:12:00.01 and Dec +33:32:42 with a flux density of 
$6.8 \pm 0.7 \text{ (stat.)} \pm 1.0 \text{ (cal.)}$ mJy at 260 GHz and $1.9 \pm 0.2 \text{ (stat.)}$ at 150 GHz. 
This is not the same point source seen in \citet{korngut2011}, which is reported as a point source
with $4.6\sigma$ significance in surface brightness, and can be fit in our current analysis as a point source 
with a flux density of $0.33 \pm 0.13$ mJy. A short VLA filler observation (VLA-12A-340, D-array, at 7 GHz) 
was performed to follow up this potential source. To a limit of $\sim 50 {\rm \mu Jy}$ nothing is seen, 
other than the clearly spatially distinct radio source associated with the BCG at the cluster center 
(1 mJy at 7 GHz and 3.2 mJy in NVSS). In contrast, the point source found in \citet{adam2015} is fit to our 
data with a flux density of $0.36 \pm 0.11$ mJy. Given the slight increase in significance of the point source
from \citet{adam2015}, we adopt that point source location for our pressure profile analysis of CLJ 1226.
 
%%% K09 flux is 0.34 +/- 0.13 mJy in our maps. Now I've written it in. Jan 2016.

In the previous analysis of the MUSTANG data, \citet{korngut2011} find a ridge of significant substructure after 
subtracting a bulk SZ profile (N07, fitted to SZA data). They find that this ridge, southwest of the cluster
center, alongside X-ray profiles, are consistent with a merger scenario. However, in this work, we do not find
any significant substructure after fitting a bulk component.
%\textcolor{red}{[I will do an analysis with the point source found in Korngut+2011. The question is if I already 
%have the point source modelled (and to find it).}

\section{Data Products}

We have made MUSTANG data products for the sample of clusters analysed in this paper available at: 
\protect{\url{https://safe.nrao.edu/wiki/bin/view/GB/Pennarray/MUSTANG_CLASH}}. Links to accompanying
Bolocam and ACCEPT data are available from this website as well. In particular, we have publicized the final
data maps, noise maps, and signal-to-noise (SNR) maps used in this analysis, as well as transfer functions
for individual clusters. Further documentation is available on the website.

%%%%%%%%%%%%%%%%%%%%%%%%%%%%%%%%%%%%%%%%%%%%%%%%%%%%%%%%%%%%%%%%%%%%%%%%%%%%%%%%%%%%%%%%%%%%%%%%%%%%%%%%%%%%%%%%
%%%%%%%%%%%%%%%%%%%%%%%%                    BIBLIOGRAPHY!!!                          %%%%%%%%%%%%%%%%%%%%%%%%%%%
%%%%%%%%%%%%%%%%%%%%%%%%%%%%%%%%%%%%%%%%%%%%%%%%%%%%%%%%%%%%%%%%%%%%%%%%%%%%%%%%%%%%%%%%%%%%%%%%%%%%%%%%%%%%%%%%


\bibliographystyle{apj}
\bibliography{mycluster}
\label{references}

\end{document}


