%%% COPY AND PASTE what I want // Excise the bad stuff!


Galaxy clusters are the largest gravitationally bound objects in the universe and thus serve as ideal cosmological probes 
and astrophysical laboratories. Within a galaxy cluster, the gas in the intracluster medium (ICM) constitutes 90\% of the
baryonic mass \citep{vikhlinin2006b} and is directly observable in the X-ray due to bremsstrahlung emission. 
At millimeter and sub-millimeter wavelengths, the ICM is observable via the Sunyaev-Zel'dovich effect (SZE) 
\citep{sunyaev1972}: the inverse Compton scattering of cosmic microwave background (CMB) photons off of
the hot ICM electrons. The thermal SZE is observed as an intensity decrement relative to the CMB at wavelengths longer 
than $\sim$1.4 mm (frequencies less than $\sim$220 GHz). The amplitude of the thermal SZE is proportional to the integrated
line-of-sight electron pressure, and is often parameterized as Compton y: $y = (\sigma_T / m_e c^2) \int P_e dl$, where
$\sigma_T$ is the Thomson cross section, $m_e$ is the electron mass, $c$ is the speed of light, and $P_e$ is the electron
pressure.
%At longer radio wavelengths, if relativistic electrons are present, parts of the ICM may emit synchrotron emission.

Cosmological constraints are generally limited by the accuracy of mass estimation of galaxy clusters. Mass estimation of
clusters found in cosmological surveys is generally calculated via a scaling relation, where some integrated (over the extent
of the cluster) observable is related to the mass. Scatter in the scaling relations will then depend on the regularity of
clusters and the assumed extent of the clusters. Determining pressure profiles of galaxy clusters provides an assessment of 
the relative impact and frequency of various astrphysical processes in the ICM and can refine the choice of extent of 
galaxy clusters to reduce the scatter in scaling relations.

In the core of a galaxy cluster, some observed astrophysical processes include shocks and cold fronts 
\citep[e.g.][]{markevitch2007}, sloshing \citep[e.g.][]{fabian2006}, and X-ray cavities \citep{mcnamara2007}. 
It is also theorized that helium sedimentation should occur, most noticeably in low redshift, dynamically-relaxed 
clusters \citep{abramopoulos1981, gilfanov1984} 
and recently the expected helium enhancement via sedimentation has been numerically simulated \citep{peng2009}. 
This would result in an offset between X-ray and SZE derived pressure profiles.

At large radii ($R \gtrsim R_{500}$),\footnote{$R_{500}$
is the radius at which the enclosed average mass density is 500 times the critical density, 
$\rho_c(z)$, of the universe} equilibration timescales are longer, accretion is ongoing, 
and hydrostatic equilibrium (HSE) is a poor approximation. 
Several numerical simulations show that the fractional contribution
 from non-thermal pressure increases with radius \citep{shaw2010,battaglia2012,nelson2014}. 
For all three studies, non thermal pressure fractions between 15\% and 30\% are found at ($R \sim R_{500}$)
for redshifts $0 < z < 1$. However, the intermediate radii, between the core and outer regions of the 
galaxy cluster, offer a region where self-similar scalings derived from HSE can be used to describe simulations 
and observations \citep[e.g.][]{kravtsov2012}. Moreover, both simulations and observations find low
cluster-to-cluster scatter in pressure profiles within this intermediate radial range \citep[e.g.][]{borgani2004,
nagai2007,arnaud2010,bonamente2012,planck2013a,sayers2013}.

While many telescopes capable of making SZE observations are already operational or are being built, most have
angular resolutions of one arcminute or larger. The MUSTANG camera \citep{dicker2008}
on the 100 meter Robert C. Byrd Green Bank Telescope \citep[GBT, ][]{jewell2004} with its angular resolution of 9\asec 
(full-width, half-maximum FWHM) and sensitivity up to $1$\amin, set by MUSTANG's instantaneous field of view (FOV)
is one of only a few SZE instruments with sub-arcminute resolution.
To probe a wider range of scales we complement our MUSTANG data with SZE data from Bolocam \citep{glenn1998}. 
Bolocam is a 144-element bolometer
array on the Caltech Submillimeter Observatory (CSO) with a beam FWHM of 58\asecs at 140 GHz and circular FOV with 8\amins 
diameter, which is well matched to the angular size of $R_{500}$ ($\sim 4$\amin) for both of the clusters in our sample. 

%Surveys of galaxy clusters can constrain cosmological parameters such as the matter
%density of the universe, $\Omega_m$, the matter power spectrum normalization, $\sigma_8$, dark energy density $\Omega_{\Lambda}$
%and the equation of state for dark energy, $w$. Constraints are often limited by the accuracy of mass estimations of the
%galaxy clusters. It is 
%\citep[e.g.][]{carlstrom2002}.
