\documentclass[iop,numberedappendix,apj]{emulateapj}
%\documentclass[iop,numberedappendix,apj]{aastex}
%\documentclass{article}
%\usepackage{emulateapj}
%\pdfoutput=1
\usepackage{color}
\usepackage{amssymb}
%\PassOptionsToPackage{hyphens}{url}
\usepackage{hyperref}
%\usepackage{breakurl}
\def\UrlBreaks{\do\/\do-}
\usepackage{natbib}
\usepackage{graphicx}
\usepackage{epsfig}
%\usepackage[hyphens]{url}
%\usepackage{csquotes}
%\usepackage{lscape}
\usepackage{afterpage}

\usepackage[tbtags]{amsmath}
\usepackage{hyperref,xcolor}
\hypersetup{colorlinks,linkcolor={blue!50!black},citecolor={blue!50!black},urlcolor={blue!80!black}}
\setlength{\tabcolsep}{0.04in} 
%\usepackage{epsfig}
%\usepackage{fullpage}
%\usepackage{hyperref}
%%%%%%%%%%%%%%%%%%%%%%%%%%%%%%%%%%%%%%%%%%%%%%%%%%%%%%%%%%%%%%%%%%%%%%%%%%%%%%%
%%% NOTES ON COMPILING / PRINTING THIS DOCUMENT
%%% do the latex <filename>
%%% dvips -O 0cm,2.0cm <filename.dvi>
%%% dvips -O 0cm,2.0cm clash_pressures.dvi

\newcommand {\apgt} {\ {\raise-.5ex\hbox{$\buildrel>\over\sim$}}\ }
\newcommand {\aplt} {\ {\raise-.5ex\hbox{$\buildrel<\over\sim$}}\ }
\newcommand{\dmod}{\overrightarrow{d}_{mod}}
\newcommand{\dvec}{\overrightarrow{d}}
\newcommand{\avec}{\overrightarrow{a}}
\newcommand{\asec}{$^{\prime \prime}$}
\newcommand{\asecs}{$^{\prime \prime}\ $}
\newcommand{\amin}{$^{\prime}$}
\newcommand{\amins}{$^{\prime}\ $}
\newcommand{\sigT}{\mbox{$\sigma_{\mbox{\tiny T}}$}}
\newcommand{\Tcmb}{\mbox{$T_{\mbox{\tiny CMB}}$}}
\newcommand{\kB}{\mbox{$k_{\mbox{\tiny B}}$}}
\newcommand{\kBT}{\mbox{$k_{\mbox{\tiny B}}T_{\mbox{\tiny e}}$}}
\newcommand{\nH}{\mbox{$n_{\mbox{\tiny H}}$}}
\newcommand{\NH}{\mbox{$N_{\mbox{\tiny H}}$}}
\newcommand{\LameH}{\mbox{$\Lambda_{e \mbox{\tiny H}}$}}
\newcommand{\Lamee}{\mbox{$\Lambda_{ee}$}}
\newcommand{\rhogas}{\mbox{$\rho_{\mbox{\scriptsize gas}}$}}
\newcommand{\rhotot}{\mbox{$\rho_{\mbox{\scriptsize tot}}$}}
\newcommand{\Mgas}{\mbox{$M_{\mbox{\scriptsize gas}}$}}
\newcommand{\Mtot}{\mbox{$M_{\mbox{\scriptsize tot}}$}}
\newcommand{\Mvir}{\mbox{$M_{\mbox{\scriptsize vir}}$}}
\newcommand{\Yint}{\mbox{$Y_{\mbox{\scriptsize int}}$}}
\newcommand{\Ycyl}{\mbox{$Y_{\mbox{\scriptsize cyl}}$}}
\newcommand{\Ysph}{\mbox{$Y_{\mbox{\scriptsize sph}}$}}
\newcommand{\fgas}{\mbox{$f_{\mbox{\scriptsize gas}}$}}
\newcommand{\LCDM}{\mbox{$\Lambda$CDM}}
\newcommand{\Pe}{\mbox{$P_{\mbox{\scriptsize e}}$}}
\newcommand{\msun}{$M_{\odot}$}
\newcommand{\etal}{{\it et al.}}
\newcommand{\mJy}{\,{\rm mJy} }
\newcommand{\um}{\,\mu {\rm m} }
\newcommand{\mJySr}{\,{\rm MJy/Sr} }
\newcommand{\mJyBm}{\,{\rm mJy/Bm} }
\newcommand{\mK}{\,{\rm mK} }
\newcommand{\K}{\,{\rm K} }
\newcommand{\uJy}{\,{\rm \mu Jy} }
\newcommand{\uK}{\,{\rm \mu K} }
\newcommand{\kHz}{\, {\rm kHz} }
\newcommand{\eg}{{\it e.g.}}
\newcommand{\ie}{{\it i.e.}}
\newcommand{\etc}{{\it etc.}}
\newcommand{\aips}{{\tt AIPS++}}
\newcommand{\nusp}{\nu_{sp}}
\newcommand{\ghz}{{\, \rm GHz}}
\newcommand{\db}{{\, \rm dB}}
\newcommand{\degsqr}{\, {\rm deg^2}}
\newcommand{\Tew}{\mbox{$T_{\mathrm{ew}}$}}
\newcommand{\Tspec}{\mbox{$T_{\mathrm{spec}}$}}
\newcommand{\chandra}{{\it Chandra}}
\newcommand{\asca}{{ASCA}}
\newcommand{\wmap}{{WMAP}}
\newcommand{\rosat}{{ROSAT}}
\newcommand{\xmm}{{XMM-{\it Newton}}}
\newcommand{\planck}{{\it Planck}}
\newcommand{\hubble}{{\it Hubble}}
\newcommand{\rxj}{RX J1347.5-1145}
\newcommand{\clj}{CL J1226.9+3332}
\newcommand{\macsa}{MACS J0647.7+7015}
\newcommand{\macsb}{MACS J1206.2-0847}
\newcommand{\macsc}{MACS J0717.5+3745}
\newcommand{\macsd}{MACS J1423.8+2404}
\newcommand{\macse}{MACS J0329.6-0211}
\newcommand{\macsf}{MACS J0429-0253}
\newcommand{\macsg}{MACS J0744.9+3927}
\newcommand{\macsh}{MACS J1149+2223}
\newcommand{\macsi}{MACS J1115+0130}
\newcommand{\Tx}{\mbox{$T_{\mbox{\tiny X}}$}}
\newcommand{\te}{\mbox{$T_{\mbox{\tiny e}}$}}
\newcommand{\mec}{\mbox{$m_{\mbox{\tiny e}} c^2$}}
\newcommand{\dene}{\mbox{$n_{\mbox{\tiny e}}$}}
\newcommand{\denesq}{\mbox{$n^2_{\mbox{\tiny e}}$}}
\newcommand{\yx}{\mbox{$Y_{\mbox{\tiny X}}$}}
\newcommand{\ysze}{\mbox{$Y_{\mbox{\tiny tSZE}}$}}
\newcommand{\sx}{\mbox{$S_{\mbox{\tiny X}}$}}
\newcommand{\Itsz}{\mbox{$I_{\mbox{\tiny tSZE}}$}}
\newcommand{\Iksz}{\mbox{$I_{\mbox{\tiny kSZE}}$}}
\newcommand{\chisq}{\mbox{$\chi^{2}$}}
\newcommand{\chired}{\mbox{$\chi^{2}_{red}$}}

\defcitealias{nagai2007}{N07}
\defcitealias{sayers2013}{S13}
\defcitealias{arnaud2010}{A10}
\defcitealias{planck2013a}{P13}
\defcitealias{cavagnolo2009}{C09}
\defcitealias{baldi2014}{B14}
%\defcitealias{baldi2016}{B16}
\defcitealias{bulbul2010}{B10}
\defcitealias{vikhlinin2006}{V06}

\newcommand{\quotes}[1]{``#1''}

\slugcomment{}
\shortauthors{Romero \etal}
\shorttitle{Joint SZE Map Fitting with MUSTANG and Bolocam}
%altaffilmark{#}

\begin{document}

\title{Galaxy Cluster Pressure Profiles as Determined by Sunyaev Zel'dovich Effect 
  Observations with MUSTANG and Bolocam II: Joint Analysis of Fourteen Clusters}
\author{
%  Order TBD ,
  Charles E. Romero\altaffilmark{1,2},
  Brian S. Mason\altaffilmark{3},
  Jack Sayers\altaffilmark{4}
  Tony Mroczkowski\altaffilmark{5,6},
  Craig Sarazin\altaffilmark{7},
  Megan Donahue\altaffilmark{8},
  Alessandro Baldi\altaffilmark{8}
  Tracy E. Clarke\altaffilmark{6},
  Alexander H.\ Young\altaffilmark{9},
  Jonathan Sievers\altaffilmark{10},
  Simon R. Dicker\altaffilmark{11},
  Erik D.\ Reese\altaffilmark{12},
  Nicole Czakon \altaffilmark{4,13},
  Mark Devlin\altaffilmark{11},
  Phillip M.\ Korngut\altaffilmark{4},
  Sunil Golwala\altaffilmark{4}
} 
\date{\today}

%%%%%%%%%%%%%%%%%%%%%%%%%%%%%%%%%%%%%%%%%%%%%%%%%%%%%%%%%%%%%%%%%%%%%%%%%%%%%%%
\altaffiltext{1}{Institut de Radioastronomie Millim\'{e}trique
300 rue de la Piscine, Domaine Universitaire
38406 Saint Martin d'H\`{e}res, France} 
\altaffiltext{2}{Author contact: \email{romero@iram.fr}}
\altaffiltext{3}{National Radio Astronomy Observatory, 520 Edgemont Rd.,
Charlottesville, VA 22903, USA}
\altaffiltext{4}{Department of Physics, Math, and Astronomy,
  California Institute of Technology, Pasadena, CA 91125, USA}
\altaffiltext{5}{European Organization for Astronomical Research in the Southern hemisphere, 
Karl-Schwarzschild-Str. 2, D-85748 Garching b. M\"unchen, Germany} 
\altaffiltext{6}{U.S.\ Naval Research Laboratory,
  4555 Overlook Ave SW, Washington, DC 20375, USA}
\altaffiltext{7}{Department of Astronomy, University of Virginia,
  P.O. Box 400325, Charlottesville, VA 22904, USA}
\altaffiltext{8}{Physics and Astronomy Department, Michigan State University,
  567 Wilson Rd., East Lansing, MI 48824, USA}
\altaffiltext{9}{MIT Lincoln Laboratories}
\altaffiltext{10}{Astrophysics \& Cosmology Research Unit, University of KwaZulu-Natal,
  Private Bag X54001, Durban 4000, South Africa}
\altaffiltext{11}{Department of Physics and Astronomy, University of
  Pennsylvania, 209 South 33rd Street, Philadelphia, PA, 19104, USA}
\altaffiltext{12}{Department of Physics, Astronomy, and Engineering, 
  Moorpark College, 7075 Campus Rd., Moorpark, CA 93021, USA} 
\altaffiltext{13}{Academia Sinica, 128 Academia Road, Nankang, Taipei 115, Taiwan}
%\begin{document}

%%%%%%%%%%%%%%%%%%%%%%%%%%%%%%%%%%%%%%%%%%%%%%%%%%%%%%%%%%%%%%%%%%%%%%%%%%%%%%%

\begin{abstract}
We present pressure profiles of galaxy clusters determined from high resolution 
Sunyaev-Zel'dovich (SZ) effect observations of fourteen clusters, which span the
redshift range $ 0.25 < z < 0.89$. 
%We compare our pressure profiles to those from the X-ray data presented by the ACCEPT 
%collaboration, both under the assumption of spherical symmetry. 
The procedure simultaneously fits spherical cluster models to MUSTANG and Bolocam data. In this analysis, 
we adopt the generalized NFW parameterization of pressure profiles to produce our models.
Our constraints on ensemble-average pressure profile parameters, in this study $\gamma$, $C_{500}$, and $P_0$,
are consistent with those in previous studies, but for individual clusters we find discrepancies 
with the X-ray derived pressure profiles from the ACCEPT2 database. We investigate potential sources of these 
discrepancies, especially cluster geometry, electron temperature of the intracluster medium, and substructure.
We find that the ensemble mean profile for all clusters in our sample is described by the parameters: 
$[\gamma,C_{500},P_0] = [0.3_{-0.1}^{+0.1}, 1.3_{-0.1}^{+0.1}, 7.9_{-0.1}^{+0.1}]$, for cool core clusters: 
$[\gamma,C_{500},P_0] = [0.6_{-0.1}^{+0.1}, 0.9_{-0.1}^{+0.1}, 3.6_{-0.1}^{+0.1}]$,
and for disturbed clusters: 
$[\gamma,C_{500},P_0] = [0.0_{-0.1}^{+0.1}, 1.5_{-0.2}^{+0.1},12.6_{-0.3}^{+0.3}]$.
Three of the fourteen clusters have clear substructure in our SZ observations, while an additional
two clusters exhibit potential substructure.
\end{abstract}

\keywords{galaxy clusters: general --- galaxy clusters}

\maketitle

%%%%%%%%%%%%%%%%%%%%%%%%%%%%%%%%%%%%%%%%%%%%%%%%%%%%%%%%%%%%%%%%%%%%%%%%%%%%%%%
\section{Introduction}
\label{sec:intro}
%%%%%%%%%%%%%%%%%%%%%%%%%%%%%%%%%%%%%%%%%%%%%%%%%%%%%%%%%%%%%%%%%%%%%%%%%%%%%%%

%\textcolor{red}{Trimming my own notes now.}

Galaxy clusters are the largest gravitationally bound objects in the universe and thus serve as excellent cosmological probes 
and astrophysical laboratories. Within a galaxy cluster, the gas in the intracluster medium (ICM) constitutes 90\% of the
baryonic mass \citep{vikhlinin2006b} and is directly observable in the X-ray due to bremsstrahlung emission. 
At millimeter and sub-millimeter wavelengths, the ICM is observable via the Sunyaev-Zel'dovich (SZ) effect 
\citep{sunyaev1972}: the inverse Compton scattering of cosmic microwave background (CMB) photons off of
the hot ICM electrons. The thermal SZ is observed as an intensity decrement relative to the CMB at wavelengths longer 
than $\sim$1.4 mm (frequencies less than $\sim$220 GHz). The amplitude of the thermal SZ is proportional to the integrated
line-of-sight electron pressure, and is often parameterized as Compton $y$: $y = (\sigma_T / m_e c^2) \int P_e dl$, where
$\sigma_T$ is the Thomson cross section, $m_e$ is the electron mass, $c$ is the speed of light, and $P_e$ is the electron
pressure.
%At longer radio wavelengths, if relativistic electrons are present, parts of the ICM may emit synchrotron emission.

%\textcolor{red}{[I need to revise this paragraph.]}
Cosmological constraints derived from galaxy cluster samples are generally limited by the accuracy of mass calibration of 
galaxy clusters \citep[e.g.][]{hasselfield2013, reichardt2013}, which is often calculated via a scaling relation with 
respect to some integrated observable quantity. Scatter in the scaling relations will then depend on the regularity of 
clusters and the adopted integration radius of the clusters. Determining pressure profiles of galaxy clusters provides an 
assessment of the relative impact and frequency of various astrophysical processes in the ICM and can refine the choice of 
integration radius of galaxy clusters to reduce the scatter in scaling relations.

In the core of a galaxy cluster, some observed astrophysical processes include shocks and cold fronts 
\citep[e.g.][]{markevitch2007}, sloshing \citep[e.g.][]{fabian2006}, and X-ray cavities \citep{mcnamara2007}. 
It is also theorized that helium sedimentation should occur, most noticeably in low redshift, dynamically-relaxed 
clusters \citep{abramopoulos1981, gilfanov1984} 
and recently the expected helium enhancement via sedimentation has been numerically simulated \citep{peng2009}. 
This would result in an offset between X-ray and SZ derived pressure profiles if not accounted for correctly.

At large radii ($R \gtrsim R_{500}$),\footnote{$R_{500}$ is the radius at which the enclosed average mass density is 
500 times the critical density, $\rho_c(z)$, of the universe} equilibration timescales are longer, accretion is ongoing, 
and hydrostatic equilibrium (HSE) can be a poor approximation. Several numerical simulations show that the fractional contribution
from non-thermal pressure increases with radius \citep{shaw2010,battaglia2012,nelson2014}. 
For all three studies, non thermal pressure fractions between 15\% and 30\% are found at ($R \sim R_{500}$)
for redshifts $0 < z < 1$. Additionally, clumping is expected to increase with radius \citep{kravtsov2012}, and is expected to
increase the scatter of pressure profiles at large radii \citep{nagai2011} as well as biasing X-ray derived gas density high,
and thus X-ray derived thermal pressure low \citep{battaglia2015}.

%However, 
By contrast, the intermediate region, between the core and outer regions of the galaxy cluster, 
is often the best region to apply self-similar scaling relations derived from HSE \citep[e.g.][]{kravtsov2012}. 
Moreover, both simulations and observations find low
cluster-to-cluster scatter in pressure profiles within this intermediate radial range \citep[e.g.][]{borgani2004,
nagai2007,arnaud2010,bonamente2012,planck2013a,sayers2013}.

In recent years, the SZ community has often adopted the pressure profile
presented in \citet{arnaud2010} (hereafter, A10), who derive their pressure profiles from X-ray data from the 
REXCESS sample of 31 nearby ($z < 0.2$) clusters out to $R_{500}$ and numerical simulations for larger radii. The
adoption of the A10 pressure profile allows for the extraction of an integrated observable quantity which,
via scaling relations, can then be used to determine the mass of the clusters. In this paper, we use high resolution
SZ data to test the validity of this pressure profile in our sample of 14 clusters at intermediate redshifts.

There are many existing facilities capable of making SZ observations, but most have
angular resolutions of one arcminute or larger. The MUSTANG camera \citep{dicker2008}
on the 100 meter Robert C. Byrd Green Bank Telescope \citep[GBT, ][]{jewell2004} with its angular resolution of 9\asec 
(full-width, half-maximum FWHM) is one of only a few SZ effect instruments with sub-arcminute resolution.
However, MUSTANG's instantaneous field of view (FOV) of 42\asecs means that it is not sensitive to scales over $\sim1$\amin. 
To probe a wider range of scales we complement our MUSTANG data with SZ data from Bolocam \citep{glenn1998}. 
Bolocam is a 144-element bolometer
array on the Caltech Submillimeter Observatory (CSO) with a beam FWHM of 58\asecs at 140 GHz and circular FOV with 8\amins 
diameter, which is well matched to the angular size of $R_{500}$ ($\sim 4$\amin) of the clusters in our sample. 

This paper is organized as follows. In Section~\ref{sec:obs} we describe the MUSTANG and Bolocam observations and reduction. 
In Section~\ref{sec:jointfitting} we review the method used to jointly fit pressure profiles to MUSTANG and Bolocam data. We
present results from the joint fits in Section~\ref{sec:pp_constraints} and compare our results to X-ray derived pressures 
in Section~\ref{sec:xray_comp}. 
Throughout this paper we assume a $\Lambda$CDM cosmology with $\Omega_m = 0.3$, $\Omega_{\lambda} = 0.7$, and $H_0 = 70$ 
km s$^{-1}$ Mpc$^{-1}$. For the remainder of the paper we denote the electron pressure as $P$, electron density as $n_e$, 
and electron temperature as $T$. The errors we report are $1\sigma$ (68.5\% confidence) unless otherwise noted.
%consistent with the 9-year \emph{Wilkinson Microwave Anisotropy Probe} (WMAP) results reported in \cite{hinshaw2013}.

%%%%%%%%%%%%%%%%%%%%%%%%%%%%%%%%%%%%%%%%%%%%%%%%%%%%%%%%%%%%%%%%%%%%%%%%%%%%%%%
\section{Observations and Data Reduction}
\label{sec:obs}
%%%%%%%%%%%%%%%%%%%%%%%%%%%%%%%%%%%%%%%%%%%%%%%%%%%%%%%%%%%%%%%%%%%%%%%%%%%%%%%

\subsection{Sample}

Our cluster sample is based primarily on the Cluster Lensing And Supernova survey with Hubble (CLASH) sample
%, which is a 524-orbit multi-cycle treasury program 
\citep{postman2012}.
%One of its main goals is to ``measure the profiles and substructures of dark matter 
%in galaxy clusters with unprecedented precision and resolution'' \citep{postman2012}. 
The CLASH sample has 25 massive galaxy clusters, 20 of which are selected from X-ray data 
(from \emph{Chandra X-ray Observatory}, hereafter \emph{Chandra}), and 5 based on exceptional lensing strength. 
These clusters have the following properties: 
$0.187 < z < 0.890$, $5.5 < k_B T$ (keV)$ < 15.5$, and $6.7 \times 10^{44} < L_{bol}$ 
(erg s$^{-1}$) $<90.8$. Thus, these clusters are large enough that we should expect to detect 
them with MUSTANG with a reasonable 
amount of time on the sky (on average, $<$25 hours per cluster).

%While these clusters are not a complete sample, many already have SZ effect observations from the Sunyaev-Zel'dovich 
%Array (SZA), AMiBA, or Bolocam, making them well studied, and deserving of high resolution SZ effect measurements. 
%The wealth of observations on these clusters will allow us to constrain pressure and 
%mass profiles of clusters as well as the impact of substructure. Additionally, we will be able to assess discrepancies
%between X-ray derived properties, shown in Table~\ref{tbl:cluster_properties} 
%and compare to SZ derived properties. 

\begin{figure*}[!h]
  \centering
  \begin{tabular}{cccc}
   \epsfig{file=figures/MBO_Contours_a1835_xray_14_Feb_2016.eps,width=0.25\linewidth,clip=}   &
   \epsfig{file=figures/MBO_Contours_a611_xray_14_Feb_2016.eps,width=0.25\linewidth,clip=}    &
   \epsfig{file=figures/MBO_Contours_m1115_xray_14_Feb_2016.eps,width=0.25\linewidth,clip=}    &
   \epsfig{file=figures/MBO_Contours_m0429_xray_14_Feb_2016.eps,width=0.25\linewidth,clip=}    \\
   \epsfig{file=figures/MBO_Contours_m1206_xray_14_Feb_2016.eps,width=0.25\linewidth,clip=}    &
   \epsfig{file=figures/MBO_Contours_m0329_xray_14_Feb_2016.eps,width=0.25\linewidth,clip=}    &
   \epsfig{file=figures/MBO_Contours_rxj1347_xray_14_Feb_2016.eps,width=0.25\linewidth,clip=}    &
   \epsfig{file=figures/MBO_Contours_m1311_xray_14_Feb_2016.eps,width=0.25\linewidth,clip=}    \\
   \epsfig{file=figures/MBO_Contours_m1423_xray_14_Feb_2016.eps,width=0.25\linewidth,clip=}    &
   \epsfig{file=figures/MBO_Contours_m1149_xray_14_Feb_2016.eps,width=0.25\linewidth,clip=}    &
   \epsfig{file=figures/MBO_Contours_m0717_xray_14_Feb_2016.eps,width=0.25\linewidth,clip=}    &
   \epsfig{file=figures/MBO_Contours_m0647_xray_14_Feb_2016.eps,width=0.25\linewidth,clip=}    \\
   \epsfig{file=figures/MBO_Contours_m0744_xray_14_Feb_2016.eps,width=0.25\linewidth,clip=}    &
   \epsfig{file=figures/MBO_Contours_clj1226_xray_14_Feb_2016.eps,width=0.25\linewidth,clip=}    &
     &
  \end{tabular}
  \caption{MUSTANG maps of the clusters in our sample. Pale contours are MUSTANG contours;
    blue contours are Bolocam. Both start at $3\sigma$ decrement, with $1\sigma$ intervals for MUSTANG
    and $2\sigma$ intervals for Bolocam.
    Red contours are X-ray surface brightness contours at arbitrary levels. The red asterisk is the 
    ACCEPT centroid; the blue asterisk is the Bolocam centroid.}
  \label{fig:mustang_maps_sample}
\end{figure*}

\begin{deluxetable*}{lllllllllll}
\tabletypesize{\scriptsize}
\tablecolumns{10}
\tablewidth{0pt} 
\tablecaption{Cluster properties \label{tbl:cluster_properties}}
\tablehead{ 
    \colhead{Cluster} & \colhead{$z$} & \colhead{$M_{500}$} & \colhead{$P_{500}$} & \colhead{$R_{500}$} & \colhead{$T_x^a$} 
              & \colhead{$T_x^b$} & \colhead{$T_{mg}$} & \colhead{Dynamical} & \colhead{$\Delta r_{X,SZ}$} \\
              \colhead{}  & \colhead{} & \colhead{($10^{14} M_{\odot}$)} & \colhead{(keV/cm$^{3}$)} & \colhead{(kpc)} & 
              \colhead{(keV)} & \colhead{(keV)} & \colhead{(keV)} & \colhead{state} & \colhead{(\asec)}
}
\startdata
    \textbf{Abell 1835}  & 0.253 & 12   & 0.00594   & 1490   & 9.0  & 10.0 & 7.49 & CC      & 6.8   \\
    \textbf{Abell 611}   & 0.288 & 7.4  & 0.00445   & 1240   & 6.8  & --   & 6.71 & --      & 18.7  \\
    \textbf{MACS1115}    & 0.355 & 8.6  & 0.00545   & 1280   & 9.2  & 9.14 & 7.04 & CC      & 34.8  \\
    \textbf{MACS0429}    & 0.399 & 5.8  & 0.00448   & 1100   & 8.3  & 8.55 & 5.56 & CC      & 18.7  \\
    \textbf{MACS1206}    & 0.439 & 19   & 0.01059   & 1610   & 10.7 & 11.4 & 10.0 & --      & 6.9   \\
    \textbf{MACS0329}    & 0.450 & 7.9  & 0.00596   & 1190   & 6.3  & 5.85 & 5.64 & CC \& D & 14.8  \\
    \textbf{RXJ1347}     & 0.451 & 22   & 0.01171   & 1670   & 10.8 & 13.6 & 9.86 & CC      & 9.6   \\
    \textbf{MACS1311}    & 0.494 & 3.9  & 0.00399   & 930    & 6.0  & 6.36 & 5.18 & CC      & 27.7  \\
    \textbf{MACS1423}    & 0.543 & 6.6  & 0.00612   & 1090   & 6.9  & 6.81 & 5.50 & CC      & 19.8  \\
    \textbf{MACS1149}    & 0.544 & 19   & 0.01228   & 1530   & 8.5  & 8.76 & 7.70 & D       & 6.0   \\
    \textbf{MACS0717}    & 0.546 & 25   & 0.01490   & 1690   & 11.8 & 10.6 & 9.06 & D       & 32.4  \\
    \textbf{MACS0647}    & 0.591 & 11   & 0.00923   & 1260   & 11.5 & 12.6 & 8.06 & --      & 6.9   \\
    \textbf{MACS0744}    & 0.698 & 13   & 0.01199   & 1260   & 8.1  & 8.90 & 6.85 & D       & 4.9   \\
    \textbf{CLJ1226}     & 0.888 & 7.8  & 0.01184   & 1000   & 12.0 & 11.7 & 11.3 & --      & 15.3  
%    \hline
%    Abell 383            & 0.187 & 4.7  & 0.00285   & 1110   & 5.4  & 5.47 & --   & CC      & --    \\
%    Abell 209            & 0.206 & 13   & 0.00564   & 1530   & 8.2  & 8.69 & --   & --      & --    \\
%    Abell 1423           & 0.213 & 8.7  & 0.00445   & 1350   & 5.8  & 6.61 & --   & --      & --    \\
%    Abell 2261           & 0.224 & 14   & 0.00632   & 1590   & 6.1  & 8.09 & --   & CC      & --    \\
%    RXJ2129              & 0.234 & 7.7  & 0.00423   & 1280   & 6.3  & 7.78 & --   & CC      & --    \\
%    MS 2137              & 0.313 & 4.7  & 0.00342   & 1060   & 4.7  & --   & --   & CC      & --    \\
%    RXC J2248            & 0.348 & 22   & 0.01014   & 1760   & 10.9 & 11.5 & --   & --      & --    \\
%    MACS1931             & 0.352 & 9.9  & 0.00595   & 1340   & 7.5  & 7.92 & --   & CC      & --    \\
%    MACS1532             & 0.362 & 9.5  & 0.00589   & 1310   & 6.8  & 6.47 & --   & CC      & --    \\
%    MACS1720             & 0.387 & 6.3  & 0.00465   & 1140   & 7.9  & 6.50 & --   & CC      & --    \\
%    MACS0416             & 0.397 & 9.1  & 0.00625   & 1270   & 8.2  & 8.14 & --   & --      & --    \\
%    MACS2129             & 0.570 & 11   & 0.00903   & 1250   & 8.6  & 8.11 & --   & D       & --    
\enddata
\tablecomments{$z$, $M_{500}$, and $T_X^a$ are taken from \citet{mantz2010}:  $T_X^a$ is calculated from a 
  single spectrum over $0.15 R_{500} < r < R_{500}$ for each cluster. $T_X^b$ is from \citet{morandi2015},
  and is calculated over $0.15 R_{500} < r < 0.75 R_{500}$.  $T_{mg}$ is a fitted gas mass weighted temperature,
  (Section~\ref{sec:temp_profiles}) determined by fitting the ACCEPT2 \citep{baldi2014} temperature profiles to 
  the gas mass weighted profile found in \citet{vikhlinin2006}. 
  The dynamical states: cool core (CC) and disturbed (D) are taken from (and defined in) \citet{sayers2013}. 
  %The bolded clusters are the 14 clusters in our sample.
  $\Delta r_{X,SZ}$ denotes the offset between the ACCEPT and Bolocam centroids.}
\end{deluxetable*}

Of the 25 clusters in the CLASH sample, four are too far south to be observed with MUSTANG from Green Bank, WV.
Of the remaining 21, we were able to observe fourteen given the available good weather and their limited visibility
during the observational campaign from 2009 to 2014.
Abell 209 was observed, but was relatively noisy and showed no trace of any detection. Our final sample
includes thirteen CLASH clusters. We also include Abell 1835, a cluster of similar mass and redshift as the CLASH
clusters, which was observed under the program GBT/09A-052. These clusters (Table~\ref{tbl:cluster_properties})
were also observed with Bolocam, and have been analyzed in \citet{sayers2012, sayers2013,czakon2015}. The centroid differences
between the \emph{Archive of Chandra Cluster Entropy Profile Tables} \citep[ACCEPT][]{cavagnolo2009}) 
and Bolocam ($\Delta r_{X,SZ}$) are also listed as  in Table~\ref{tbl:cluster_properties}. The total integration times of
MUSTANG and Bolocam observations of our sample is listed in Table~\ref{tbl:cluster_obs}. 

\begin{deluxetable*}{lllllllllll}
\tabletypesize{\footnotesize}
\tablecolumns{10}
\tablewidth{0pt} 
\tablecaption{Bolocam and MUSTANG observational properties. \label{tbl:cluster_obs}}
\tablehead{ 
    \colhead{Cluster} & \colhead{$z$} & \colhead{R.A.} & \colhead{Decl.} & 
              \colhead{$t_{obs,B}$} & \colhead{Noise$_{B}$} & \colhead{A10$_{B}$} & 
              \colhead{$t_{obs,M}$} & \colhead{Noise$_{M}$} & \colhead{A10$_{M}$}    \\
            & \colhead{} & \colhead{(J2000)} & \colhead{(J2000)} &  
              \colhead{(hours)} & \colhead{$\mu K_{CMB}$-amin} & \colhead{($\sigma$)} &
              \colhead{(hours)} & \colhead{$\mu$Jy/bm}        & \colhead{($\sigma$)}
}
\startdata
    \textbf{Abell 1835}  & 0.253 & 14:01:01.9 & +02:52:40 & 14.0 & 16.2 & 28.9 & 8.6  & 53.4 & 10.0  \\
    \textbf{Abell 611}   & 0.288 & 08:00:56.8 & +36:03:26 & 18.7 & 25.0 & 13.9 & 12.0 & 46.2 & 1.73  \\
    \textbf{MACS1115}    & 0.355 & 11:15:51.9 & +01:29:55 & 15.7 & 22.8 & 16.3 & 10.0 & 56.4 & 8.66  \\
    \textbf{MACS0429}    & 0.399 & 04:29:36.0 & -02:53:06 & 17.0 & 24.1 & 13.2 & 11.6 & 47.2 & -0.02 \\
    \textbf{MACS1206}    & 0.439 & 12:06:12.3 & -08:48:06 & 11.3 & 24.9 & 28.7 & 13.3 & 42.5 & 8.89  \\
    \textbf{MACS0329}    & 0.450 & 03:29:41.5 & -02:11:46 & 10.3 & 22.5 & 17.4 & 13.1 & 39.9 & 8.63  \\
    \textbf{RXJ1347}     & 0.451 & 13:47:30.8 & -11:45:09 & 15.5 & 19.7 & 45.3 & 1.9  & 276. & 8.90  \\
    \textbf{MACS1311}    & 0.494 & 13:11:01.7 & -03:10:40 & 14.2 & 22.5 & 11.3 & 10.6 & 64.5 & 0.71  \\
    \textbf{MACS1423}    & 0.543 & 14:23:47.9 & +24:04:43 & 21.7 & 22.3 & 11.8 & 11.2 & 35.7 & 6.15  \\
    \textbf{MACS1149}    & 0.544 & 11:49:35.4 & +22:24:04 & 17.7 & 24.0 & 22.0 & 13.9 & 32.7 & -1.47 \\
    \textbf{MACS0717}    & 0.546 & 07:17:32.1 & +37:45:21 & 12.5 & 29.4 & 31.3 & 14.6 & 27.1 & 3.05  \\
    \textbf{MACS0647}    & 0.591 & 06:47:49.7 & +70:14:56 & 11.7 & 22.0 & 24.1 & 16.4 & 20.3 & 11.3  \\
    \textbf{MACS0744}    & 0.698 & 07:44:52.3 & +39:27:27 & 16.3 & 20.6 & 17.8 & 7.6  & 48.5 & 7.67  \\
    \textbf{CLJ1226}     & 0.888 & 12:26:57.9 & +33:32:49 & 11.8 & 22.9 & 13.7 & 4.9  & 85.6 & 9.43 
\enddata
\tablecomments{Subscripts $_{B}$ and $_{M}$ denote Bolocam and MUSTANG properties respectively. Noise$_{B}$
  and $t_{obs,B}$ are those reported in \citet{sayers2013}. Noise$_{M}$ is calculated on MUSTANG maps with 
  $10$\asecs smoothing, in the central arcminute. $t_{obs}$ are the integration times (on source) for the 
  given instruments. A10$_B$ and A10$_M$ values indicate the significance (in $\sigma$) of $P_0$ when we
  fit a spherical A10 \citep{arnaud2010} profile (see Section~\ref{sec:bulk_ICM}).}
\end{deluxetable*}

%The clusters were observed with
%MUSTANG over the projects AGBT08A\_056, AGBT09A\_052, AGBT09C\_059, AGBT10A\_056, AGBT10C\_017, AGBT10C\_026, AGBT10C\_042, 
%AGBT10C\_031, AGBT11A\_009, and AGBT11B\_001.


%%%%%%%%%%%%%%%%%%%%%%%%%%%%%%%%%%%%%%%%%%%%%%%%%%%%%%%%%%%%%%%%%%%%%%%%%%%%%%%
\subsection{MUSTANG Observations and Reduction}
\label{sec:musobs}
%%%%%%%%%%%%%%%%%%%%%%%%%%%%%%%%%%%%%%%%%%%%%%%%%%%%%%%%%%%%%%%%%%%%%%%%%%%%%%%

MUSTANG is a 64 pixel array of Transition Edge Sensor (TES) bolometers arranged in an $8 \times 8$ array
located at the Gregorian focus on the 100 m GBT. Operating at 90 GHz (81--99~GHz),
MUSTANG has an angular resolution of 9\asec and pixel spacing of 0.63$f \lambda$ resulting in a FOV
of 42\asec. More detailed information about the instrument can be found in \citet{dicker2008}.

Our observations and data reduction are described in detail in \citet{romero2015a}, and we briefly review them
here. Absolute flux calibrations are based on the planets Mars, Uranus, and Saturn; nebulae; or the star Betelgeuse 
($\alpha_{Ori}$). At least one of these flux calibrators was observed at least once per night, and we find our 
calibration is accurate to a 10\% RMS uncertainty. We also observe bright point sources every half hour
to track our pointing and beam shape. To observe the target galaxy clusters, we employ Lissajous daisy scans 
with a $3\arcmin$ radius and in many of the clusters we broadened our coverage with a hexagonal pattern of 
daisy centers (with $1\arcmin$ offsets). 
For most clusters, the coverage (weight) drops to 50\% of its peak value at a radius of $1.3\arcmin$.
%'

Processing of MUSTANG data is performed using a custom IDL pipeline. Raw data is recorded as time ordered data (TOD)
from each of the 64 detectors. An outline of the data processing for each scan on a galaxy cluster is as follows:
  
  (1) We define a pixel mask from the nearest preceding CAL scan; unresponsive detectors are masked out.
  The CAL scan provides us with unique gains to be applied to each of the responsive detectors.

  (2) A common mode template, polynomial, and sinusoid are fit to the data and then subtracted. The common mode is
  calculated as the arithmetic mean of the TOD across detectors.

  (3) After the common mode and polynomial subtraction each scan is subjected to spike (glitch), skewness, and Allan 
  variance tests and are flagged according to the following criteria. Glitches are flagged as $4\sigma$ excursions based
  on the median absolute deviation; The skewness threshold for flagging is 0.4. Flags based on Allan variance require
  the variance over a two second interval to be greater than 9 times the variance between each integration.

  (4) Individual detector weights are calculated as $1/ \sigma_i^2$, where $\sigma_i$ is the RMS of the non-flagged
  TOD for that detector. 

  (5) Maps are produced by gridding the TOD in 1\asec pixels in Right Ascension (R.A.) and Declination (Dec). A weight 
  map is produced in addition to the signal map.

\begin{figure}
  \begin{center}
  \includegraphics[width=0.5\textwidth]{figures/Transfer_Function_All_30_Mar_2016_long_v2.eps}
%  \includegraphics[width=0.5\textwidth]{figures/Transfer_Function_All_28_Jan_2016_rms.eps}
%  \includegraphics[width=0.5\textwidth]{figures/Transfer_Function_All_12_Mar_2016_rms.eps}
  \end{center}
  \caption{Effective average transfer function of our MUSTANG data reduction over our sample. 
    The variations between cluster are less than 3\%. For each cluster, attenuation is
    calculated based on simulated observations of 25 fake skies. The plotted one-dimensional
    transfer function is the weighted average of the transfer functions
    of individual clusters. The error bars show the scatter among cluster transfer functions. 
    The transfer functions (transmission) of individual clusters are calculated as the square
    root of the ratio of the one dimensional power spectra of the observed fake sky
    and input fake sky. We have labelled the relevant angular wavenumbers for the FOV and FWHM.}
  \label{fig:xfer_all}
\end{figure}

\subsection{Bolocam Observations and Reduction}
\label{sec:bolocamredox}

Bolocam is a 144-element camera that was a facility instrument on the Caltech Submillimeter Observatory (CSO) from
2003 until 2012. Its field of view is 8\amins in diameter, and at 140 GHz it has a resolution of 58\asecs FWHM
(\citet{glenn1998,haig2004}). The clusters were observed with a Lissajous pattern that results in a tapered
coverage dropping to 50\% of the peak value at a radius of roughly 5\amin, and to 0 at a radius of 10\amin.
The Bolocam maps used in this analysis are $14\arcmin \times 14\arcmin$. The Bolocam data 
\footnote{Bolocam data is publicaly available at 
\href{http://irsa.ipac.caltech.edu/data/Planck/release\_2/ancillary-data/bolocam/}
{http://irsa.ipac.caltech.edu/data/Planck/release\_2/ancillary-data/} 
\href{http://irsa.ipac.caltech.edu/data/Planck/release\_2/ancillary-data/bolocam/}{bolocam/}.} 
are the same as those used in \citet{czakon2015} and \citet{sayers2013}; the details of the reduction are 
given therein, along with \citet{sayers2011}. 
%Bolocam observed Abell 1835 for 14.0 hours resulting in a noise of 16.2 $\mu K_{CMB}$-arcminute, and observed 
%MACS 0647 for 11.4 hours resulting in a noise of 22.0 $\mu K_{CMB}$-arcminute.
%In addition to the data maps, for each cluster 1000 noise maps are also provided, which
%included relevant sources of instrumental, atmospheric, and astronomical noise. 
The reduction and calibration is similar to that used for MUSTANG, and Bolocam achieves a 
5\% calibration accuracy and 5\asecs pointing accuracy.

%%%%%%%%%%%%%%%%%%%%%%%%%%%%%%%%%%%%%%%%%%%%%%%%%%%%%%%%%%%%%%%%%%%%%%%%%%%%%%%
\section{Joint Map Fitting Technique}
\label{sec:jointfitting}
%%%%%%%%%%%%%%%%%%%%%%%%%%%%%%%%%%%%%%%%%%%%%%%%%%%%%%%%%%%%%%%%%%%%%%%%%%%%%%%

\subsection{Overview}
\label{sec:jf_overview}

The joint map fitting technique used in this paper is described in detail in \citet{romero2015a}. We review
it briefly here. The general approach follows that of a least squares fitting procedure, which assumes that
we can make a model map as a linear combination of model components. 

This linear combination can be written as:
\begin{equation}
  \vec{d}_m = \mathbf{A} \vec{a}_m,
  \label{eqn:model_array}
\end{equation}
where $d_m$ is the total model, each column in $\mathbf{A}$ is a filtered model component (Section~\ref{sec:components}), 
and $\vec{a}_m$ is 
an array of amplitudes of the components. There are up to four types of components for which we fit: 
a bulk component, point source(s), residual component(s), and a mean level. Of these, we produce a
sky model for the bulk component and point source to be filtered. The residual component is calculated 
directly as a filtered component.

We wish to fit $\vec{d}_m$ to our data, $\vec{d}$, and allow for a calibration offset between Bolocam and
MUSTANG data. We therefore define our data vector as:
\begin{equation}
  \vec{d} = [ \vec{d}_{B}, k \vec{d}_{M}, k ] ,
  \label{eqn:data_arr}
\end{equation}
where $\vec{d}_{B}$ is the Bolocam data, $\vec{d}_{M}$ is the MUSTANG data, and $k$ is the calibration offset of
MUSTANG relative to Bolocam, to which we apply an 11.2\% Gaussian prior derived from the MUSTANG and Bolocam
calibration uncertainties.

We use the $\chi^2$ statistic as our goodness of fit:
\begin{equation}
  \chi^2 = (\overrightarrow{d} - \overrightarrow{d}_m)^T \mathbf{N}^{-1} (\overrightarrow{d} - \overrightarrow{d}_m),
  \label{eqn:chi_sq}
\end{equation}
where $\mathbf{N}$ is the covariance matrix; however, because we wish to fit for $k$ in addition to the 
amplitude of model components, we no longer have completely linearly independent variables, and thus we 
employ MPFIT \citep{markwardt2009} to solve for these variables. Confidence intervals are derived from 
$\chi^2$ values over the parameter space searched (Section~\ref{sec:param_space}).

%%% Suggestion to remove this section. Do so?
%\subsection{Simulated Observations}
%\label{sec:jf_filtering}

%Simulated observations are performed by converting an input sky into input TOD and processing the simulated TOD
%as the true TOD are processed (Section~\ref{sec:musobs}). The time requirements for simulated observations are  
%reduced by storing the necessary telescope and detector information from the true TOD in an IDL structure. 
%Due to the time requirements to cover the
%necessary parameter space (Section~\ref{sec:param_space}), TOD were produced with a fraction of the scans (termed
%short TOD) on a given cluster, where care was taken to ensure the same coverage (relative weight distribution) 
%in the map as the full observation. 
%The filtering is observed to be the same between full TOD and short TOD (Figure~\ref{fig:long_vs_short_qv}). 

%\begin{figure}[!h]
%  \centering
%  \includegraphics[width=0.5\textwidth]{figures/Long_vs_short_qv_bestfit_m0647_22_Jun_2015_v2.eps}
%  \caption{MUSTANG simulated observations of a fitted model to MACS 0647. 
%    The scatter in the short TOD is $\lesssim 3$\% 
%    of the peak Compton $y$. In absolute terms, this translates to roughly $2\times 10^{-5}$ in Compton $y$. 
%    Typical pixel noise in maps is 7 to 8 times greater.}
%  \label{fig:long_vs_short_qv}
%\end{figure}

\subsection{Components}
\label{sec:components}

In order to produce component maps, it is necessary to account for the response of both instruments and
imaging pipeline filter functions. For Bolocam, we use the transfer function provided. For MUSTANG, we
perform simulated observations, processing the sky models in the same manner that real data is processed.

\subsubsection{Bulk ICM}
\label{sec:bulk_ICM}

 As in \citet{romero2015a}, the bulk component is taken to be a 
spherically symmetric 3D electron pressure profile as parameterized by a generalized Navarro, Frenk,
and White profile \citep[hereafter, gNFW][]{navarro1997,nagai2007}:
\begin{equation}
  \Tilde{P} = \frac{P_0}{(C_{500} X)^{\gamma} [1 + (C_{500} X)^{\alpha}]^{(\beta - \gamma)/\alpha}}
  \label{eqn:gnfw}
\end{equation}
where $X = R / R_{500}$, and $C_{500}$ is the concentration parameter; one can also write ($C_{500} X$) as
($R / R_s$), where $R_s = R_{500}/C_{500}$. $\Tilde{P}$ is the electron pressure in units of the characteristic
pressure $P_{500}$. This pressure profile is integrated along the line of sight to produce 
a Compton $y$ profile, given as 
\begin{equation}
  y(r) = \frac{P_{500} \sigma_{T}}{m_e c^2} \int_{-\infty}^{\infty} \Tilde{P}(r,l) dl
  \label{eqn:compton_y}
\end{equation}
where $R^2 = r^2 + l^2$, $r$ is the projected radius, and $l$ is the distance from the center of the cluster
along the line of sight. Once integrated, $y(r)$ is gridded as $y(\theta)$ and is realized as two maps with
the same astrometry as the MUSTANG and Bolocam data maps (pixels of 1\asecs and 20\asecs on a side, respectively). 
%From here, we produce two model maps: one for Bolocam and one for MUSTANG. 
In each case, we convolve the Compton $y$ map by the appropriate beam shape. For Bolocam we use a Gaussian with FWHM
$= 58$\asec, and for MUSTANG we use the double Gaussian, representing the GBT main beam and stable error beam
\cite{romero2015a}.

\subsubsection{Point Sources}
\label{sec:ptsrcs}

Point sources are treated in the same manner as in \citet{romero2015a}. All compact sources in our sample
are well modelled as a point source. We clearly detect point sources in Abell 1835, MACS 1115, MACS 0429, 
MACS 1206, RXJ1347, MACS 1423, and MACS 0717 in the MUSTANG maps. A point source is identified by NIKA 
\citep{adam2015} in CLJ1226, which is posited to be a submillimeter galaxy (SMG) behind the cluster. That
point source is distinct from the point source seen in \citet{korngut2011}, which is not evident in our map
and has a lower fitted significance than the one identified by NIKA. 
The point source in MACS 0717 is due to a foreground elliptical galaxy \citep{mroczkowski2012}.
All of the remaining point sources (six) are coincident (within 3\asecs of reported coordinnates) with the BCGs 
of their respective clusters \citep[][]{crawford1999,donahue2015}. 
%%%%%%% Suggestion to rewrite this text:
Moreover, of these six BCGs, four of them exhibit \quotes{unambiguous UV excess} \citep{donahue2015}. 
The remaining two are Abell 1835 and MACS 1206. The UV excess in MACS 1206 may be due to lensed
background systems \citep{donahue2015}. Abell 1835 is not in the CLASH sample and thus was not
included in \citet{donahue2015}. However, it was observed by \citet{odea2010} and found to have a 
far UV flux corresponding to a star formation rate of 11.7 $M_{\odot}$ per year, which fits within the SFR range 
(5 - 80 $M_{\odot}$ yr$^{-1}$) of the UV excess BCGs found in \citep{donahue2015}.
%%%%%%%%%%%%%%%%%%%%%%%%%%%%%%%%%%%%%%
For the Bolocam images, the point sources in Abell 1835, MACS 0429, RXJ1347, and MACS 1423
have been subtracted based on an extrapolation of a power law fit to the 1.4 GHz NVSS \citep{condon1998}
and 30 GHz SZA \citep{mroczkowski2009} measurements \citep{sayers2012}. The flux densities for
the point sources fitted are shown in Table~\ref{tbl:sample_ptsrc}.

\begin{deluxetable}{c c c c c}
\tabletypesize{\footnotesize}
\tablecolumns{5}
\tablewidth{0pt} 
\tablecaption{Point source flux densities \label{tbl:sample_ptsrc}}
\tablehead{ 
    \colhead{Cluster} & \colhead{R.A. (J2000)} & \colhead{Dec (J2000)} & 
    \colhead{$S_{90}$ (mJy)}  & \colhead{$S_{140}$ (mJy)}
}
\startdata
Abell 1835  & 14:01:02.07  &  +2:52:47.52  & $1.37 \pm 0.08$ & $0.7 \pm 0.2$ \\
MACS 1115   & 11:15:51.82  &  +1:29:56.82  & $1.04 \pm 0.11$ & --            \\  
MACS 0429   & 04:29:35.97  &  -2:53:04.74  & $7.67 \pm 0.84$ & $6.0 \pm 1.8$ \\
MACS 1206   & 12:06:12.11  &  -8:48:00.85  & $0.75 \pm 0.08$ & --            \\  
RXJ1347     & 13:47:30.61  & -11:45:09.48  & $7.40 \pm 0.58$ & $4.0 \pm 1.2$ \\  
MACS 1423   & 14:23:47.71  & +24:04:43.66  & $1.36 \pm 0.13$ & $0.7 \pm 0.2$ \\  
MACS 0717   & 07:17:37.03  & +37:44:24.00  & $2.08 \pm 0.25$ & --            \\   
CLJ1226     & 12:27:00.01  & +33:32:42.00  & $0.36 \pm 0.11$ & --   
\enddata
  \tablecomments{$S_{90}$ is the best fit flux density to MUSTANG, and $S_{140}$ is the assumed flux density in
  the Bolocam maps (at 140 GHz). The location of the point source is reported from the fitted centroid to the
  MUSTANG data. The conversion from mJy to the equivalent uK$_{CMB}$ is given as: 
  $S_{140} (\text{mJy/bm}) \sim S_{140} / 20 (\mu\text{K}_{CMB})$.}
\end{deluxetable}

\subsubsection{Residual Components}
\label{sec:residuals}

Residual components are selected primarily as $4 \sigma$ decrements within the central arcminute of smoothed
MUSTANG residual maps, which are not well fitted by a bulk model. 
We fit residual components for MACS 1206, RXJ 1347, and MACS 0744.
Although we do not fit for residual components in Abell 611 and MACS 1115, we report properties of potential
residual components for these two clusters. We do not fit the residual component for Abell 611 because the peak
significance is not $4\sigma$ and the bulk cluster model appears to account for much of the decrement. For MACS 1115, 
the residual component is outside the central arcminute and does not affect our fit.

To model the shape of residual component, we fit a two dimensional Gaussian to the selected pixels 
(those below $-4\sigma$). This Gaussian is then fit to the unsmoothed MUSTANG data map 
(in units of Compton $y$) with only its amplitude is allowed to vary to obtain the results presented
in Table\ref{tbl:resid_comps}.
%To create the residual component, we first select the feature of interest based on the MUSTANG signal-to-noise (SNR) map
%of the cluster \citep[see][]{romero2015a}. All pixels below $-3 \sigma$ pertaining to the feature are selected, and the 
%shape is determined by fitting a two dimensional Gaussian. This Gaussian is then fit to the unsmoothed MUSTANG data map 
%(in units of Compton $y$), where only its amplitude is allowed to vary.

\begin{deluxetable*}{c | c c c c c c c}
\tabletypesize{\footnotesize}
\tablecolumns{5}
\tablewidth{0pt} 
\tablecaption{Parameters of Residual Components from MUSTANG \label{tbl:resid_comps}}
\tablehead{
Cluster & RA      & Dec     & Modeled Peak y    & FWHM$_A$ & FWHM$_B$ & $\theta$ & Fitted Peak y \\
        & (J2000) & (J2000) & ($10^{-5}$) & (\asec) &  (\asec) & (deg.) & ($10^{-5}$)     
}
\startdata
Abell 611 &  8:00:56.20 & 36:03:00.08 &  8.4  &  20.7 &  35.3 &   160 & --        \\ 
MACS 1115 & 11:15:56.66 &  1:30:02.82 &  14   &  17.8 &  28.8 &   138 & --        \\ 
MACS 1206 & 12:06:12.91 & -8:47:33.48 &  7.6  &  23.5 &  23.5 &  -115 & $3.6 \pm 0.7$   \\ 
RXJ1347   & 13:47:31.06 &-11:45:18.38 &  42   &  12.2 &  30.1 &   -52 & $52  \pm 9$     \\ 
MACS 0744 &  7:44:52.22 & 39:27:28.71 &  11   &  17.0 &  23.5 &     1 & $9.0 \pm 2.8$    
\enddata
\tablecomments{Residual components modeled with a two dimensional Gaussian with associated. $\theta$ is 
  measured CCW (going east) from due north.}
\end{deluxetable*}

\subsubsection{Mean Level}
\label{sec:mean_level}

%%% JSayers: Bolocam too? Yes...if done, both are done. But I removed the mean level fit in the simple sense.
%%% I don´t think I fit out a Bolocam mean level, because it was so low.
Similar to \citet{czakon2015}, we wish to account for a mean level (signal offset) in the MUSTANG maps.
We do not wish to fit for a mean level simultaneously as a bulk component given the degeneracies. Therefore,
to determine the mean level independent of the other components, we create a MUSTANG noise map
% from time-flipped TOD 
and calculate the mean within the inner arcminute for each cluster. This mean is then subtracted before 
the other components are fit. 

\subsection{Parameter Space}
\label{sec:param_space}

As in \citet{romero2015a}, we fix MUSTANG's centroid, but allow Bolocam's pointing to vary by $\pm 10$\asecs 
in RA and Dec with a prior on Bolocam's radial pointing accuracy with an RMS uncertainty of $5$\asec. Our 
approach to find the absolute calibration offset between Bolocam and MUSTANG is the same as in
\citet{romero2015a} (see also Section~\ref{sec:jf_overview}). 

In \citet{romero2015a}, we performed a grid search over $\gamma$ and $C_{500}$, marginalizing over $P_0$,
where $\alpha$ and $\beta$ are fixed to values determined from \citetalias{arnaud2010}. To determine the
impact of our choice of fixed $\alpha$ and $\beta$, we explored how the profile shapes change when different,
fixed, values of $\alpha$ and $\beta$ are adopted. In all cases, we find the pressure profile shapes are in very 
good agreement with one another and that the differences in $\chi^2$ values are statistically consistent. Thus, 
our fits are not sensitive to the exact choice of $\alpha$ and $\beta$.

We search over $0 < \gamma < 1.3$ in steps of $\delta \gamma = 0.1$, and over $0.1 < C_{500} < 3.3$ in steps of 
$\delta C_{500} = 0.1$. This choice of parameter space searched is determined by computation requirements (largely in
filtering maps) and covering asufficient range of values. To create models in finer steps than $\delta \gamma$ 
and $\delta C_{500}$, we interpolate filtered model maps from nearest neighbors from the grid of original 
filtered models. 

\subsubsection{Centroid Choice}
\label{sec:centroids}

The default centroids used when gridding our bulk ICM component are the ACCEPT centroids. Given the offsets
between ACCEPT and Bolocam centroids (Table~\ref{tbl:cluster_properties}), we perform a second set of
fits where we grid the bulk ICM component using the Bolocam centroids and we do not find significant changes in
the fitted gNFW parameters (Section~\ref{sec:pp_constraints}). The ACCEPT centroid are taken to be the
X-ray peaks unless the centroiding algorithm produced a centroid more than 70 kpc from the X-ray peak, in which
case they adopt that centroid \citep{cavagnolo2008a}. 
%We do not find significant changes in
%the fitted gNFW parameters (Section~\ref{sec:pp_constraints}), 

%%%%%%%%%%%%%%%%%%%%%%%%%%%%%%%%%%%%%%%%%%%%%%%%%%%%%%%%%%%%%%%%%%%%%%%%%%%%%%%
\section{SZ Pressure Profile Constraints}
\label{sec:pp_constraints}
%%%%%%%%%%%%%%%%%%%%%%%%%%%%%%%%%%%%%%%%%%%%%%%%%%%%%%%%%%%%%%%%%%%%%%%%%%%%%%%

We have constrained the gNFW parameters $P_0$, $C_{500}$, and $\gamma$ for fourteen individual clusters and present 
these constraints in Table~\ref{tbl:pressure_profile_results}. Given that we find minimal differences between the 
fitted parameters using either the ACCEPT or Bolocam centroids, we report the results using the ACCEPT centroids.
%We find that six of the fourteen clusters are best fit by $\gamma = 0$. 
We find that six of our sample of fourteen have a best fit $\gamma = 0$, where we do not allow $\gamma <0$. 
We find that our range of $C_{500}$ is sufficient, and that it is generally found to be $0.5 < C_{500} < 2.0$. 

%For each cluster, we compare our SZ-derived pressure profiles with X-ray derived pressure profiles from ACCEPT2 
%\citep{baldi2014}. Specifically, 
We are further interested in comparing our pressure profile constraints, individually, and as a sample, to previous
constraints. To compare to the pressure profiles from ACCEPT2 \citep{baldi2014},
we fit gNFW profiles to the deprojected pressure profiles of our cluster sample (Section~\ref{sec:ellgeo}). 
We adopt B14 to refer to the ensemble pressure profiles fit to ACCEPT2 data for our sample of 14 clusters. 
Individually, we find discrepancies in pressure profiles, but as an ensemble there is relatively good agreement. 
Moreover, the average pressure profile for the 14 clusters has parameter values which are very similar 
to those found using X-ray data in \citet{arnaud2010}. This can also be seen in Figure~\ref{fig:pp_sets}, 
where the A10 and B14 pressure profiles are generally consistent to the profile from this work (R16),
where deviations are $<30$\% over $0.03 R_{500} < r < R_{500}$ for A10 and $<50$\% for B14. 
While all 14 clusters in this work are in \citet{sayers2013} (hereafter S13), we note that they find a consistently higher 
average pressure profile. Furthermore, the average pressure profile found by \citet{planck2013a} (hereafter P13) is also
higher than our average profiles at both large and small radii. In Figure~\ref{fig:pp_sets} we also include
a comparison to the pressure profile determined from simulations in \citet{nagai2007}, denoted as N07.

%\textcolor{red}{[I think I want to expand on this - perhaps postulating why we
%find these trends.]}

%%%%%%%%%%%%%%%%%%%%%%%%%%%%%%%%%%%%%%%%%%%%%%%%%%%%%%%%%%%%%%%%%%%%%%%%%%%%%%%%%%%%%%%%%%%%%%%%%%%%%%%%%%%
%%%                                                SOME FIGURES                                         %%%
%%%%%%%%%%%%%%%%%%%%%%%%%%%%%%%%%%%%%%%%%%%%%%%%%%%%%%%%%%%%%%%%%%%%%%%%%%%%%%%%%%%%%%%%%%%%%%%%%%%%%%%%%%%

\begin{figure}[!h]
  \centering
  \begin{tabular}{cc}
     \epsfig{file=figures/JF_Conf_Intervals_Disturbed_23_Jan_2016.eps,width=0.50\linewidth,clip=} &
     \epsfig{file=figures/JF_Conf_Intervals_Cool_core_23_Jan_2016.eps,width=0.50\linewidth,clip=} \\
%     \epsfig{file=figures/JF_Conf_Intervals_Well_behaved_23_Jan_2016.eps,width=0.50\linewidth,clip=} &
     \epsfig{file=figures/JF_Conf_Intervals_All_23_Jan_2016.eps,width=0.50\linewidth,clip=} &
   \end{tabular}
  \caption{Confidence intervals over all disturbed clusters (upper left panel), 
    cool-core clusters (upper right panel), and the entire sample (lower left panel).
    Cool core clusters include: Abell 1835, MACS 1115, MACS 0429, MACS 0329, RXJ 1347, 
    MACS 1311 and MACS 1423. Disturbed clusters include: MACS 0329, MACS 1149, MACS 0717, and
    MACS 0744.}
           %Well behaved clusters include: Abell 1835, MACS 1115, MACS 1206, RXJ 1347,
%           MACS 0647, MACS 0744, and CLJ 1226. Well behaved clusters are identified above.
  \label{fig:ensemble_cis}
\end{figure}

%%% Remove references to well behaved clusters.

\begin{figure}
  \begin{center}
%  \includegraphics[width=0.5\textwidth]{figures/profile_sets_plot_v2.eps}
%  \includegraphics[width=0.5\textwidth]{figures/profile_sets_plot_v2_23_Jan_2016.eps}
%  \includegraphics[width=0.5\textwidth]{figures/profile_sets_plot_v2_29_Mar_2016.eps}
%  \includegraphics[width=0.5\textwidth]{figures/profile_sets_plot_v2_18_Jul_2016.eps}
  \includegraphics[width=0.5\textwidth]{figures/profile_sets_plot_v2_27_Jul_2016.eps}
  \end{center}
  \caption{Pressure profiles from this (R16) and other works. We observe that for our
    fourteen clusters, the ACCEPT2 data \citepalias{baldi2014} falls below R16, whereas
    \citetalias{arnaud2010,planck2013a}, and \citetalias{sayers2013} show higher pressure 
    at large radii. The pressure profile \citetalias{nagai2007} also agrees
    well with our work, but shows a steeper inner profile.}
  \label{fig:pp_sets}
\end{figure}

% Now I want to compare to A10, P13, S13? OK...this is done.
%%% What else to say?

While our average pressure profiles are in excellent agreement with the previously derived pressure profiles
in the region $0.1 R_{500} < r < R_{500}$, we see deviations at small and large radii. It is not too surprising
that our fits agree with A10 at large radii, as we have fixed $\alpha$ and $\beta$ to the A10 values. Despite our
fourteen clusters being included in the BOXSZ sample \citep{sayers2013}, we see that S13 shows higher pressure
at all radii. 
S13, fixing the slope of $\gamma$, present a higher pressure at small radii than found here, where the MUSTANG
data provide stronger constraints on the pressure gradients in the cluster core and suggest they are often weaker
than previously thought. 
%We note that S13 presents a higher pressure at small radii than found in this work, where the addition of MUSTANG 
%data contributes significantly to constraining the pressure at small radii (towards smaller pressures). 
At larger radii, our restriction of $\alpha$ and $\beta$ again explain our reduced pressure relative
to S13.

%Figure~\ref{fig:ppr_ensembles} shows the ratios of the pressure profiles derived from this work to those from 
%other works, when clusters are characterized by dynamical type. We calculate these average ratios by weighting 
%the ratios of individual clusters, where the specific fitted gNFW pressure profile for
%each cluster is taken for ACCEPT2, but for the other sets (A10, P13, and S13), we assume the gNFW profile 
%found for all clusters in their sample. In fitting profiles to ACCEPT2, we impose the same restriction on 
%$\alpha$ and $\beta$ (fixing them to A10 values). In this manner, this imposition no longer biases our results
%at large radii, and we consistently see higher pressures at large radii in the SZ as compared to X-ray (ACCEPT2) 
%data. 

We further consider the ratio of cluster pressure profiles from our work ($P_{SZ}$) to the pressure profiles from 
other works. For comparisons with A10, P13, and S13, we take $P_{A10}$, $P_{P13}$, and $P_{S13}$ to be the gNFW
profile which each respective work had fit to their entire sample. For any of these sets (A10, P13, or S13), 
the ratio $P_{SZ} / P_{set}$ is calculated for each cluster, where only $P_{SZ}$ changes for each cluster. To
compare $P_{SZ}$ to ACCEPT2 ($P_{XRAY}$), we fit a gNFW profile to ACCEPT2 data (Section~\ref{sec:ellgeo}) for
each cluster, and thus compare unique $P_{SZ}$ amd $P_{XRAY}$ pressure profiles for each cluster. These ratios are
shown in Figure~\ref{fig:ppr_ensembles}, where the shaded regions are influenced both by statistical errors and
scatter.

%\textcolor{red}{I also started thinking about \emph{XMM-Newton} vs. \emph{Chandra} data. However,
%\citet{donahue2014} would indicate that \emph{XMM-Newton} data tends to be lower in $n_e$ and $T_e$, which is
%\textbf{opposite} what we see in C09 vs. A10.}

\begin{figure}
  \begin{center}
%  \includegraphics[width=0.5\textwidth]{figures/PPRs_ensembles_All_24_Oct_2015_v2.eps}
%  \includegraphics[width=0.5\textwidth]{figures/PPRs_ensembles_All_11_Jan_2016.eps}
  \begin{tabular}{cc}
    \epsfig{file=figures/PPRs_ensembles_All_XRAY_scalerr_v0_18_Jul_2016.eps,width=0.50\linewidth,clip=}   &
    \epsfig{file=figures/PPRs_ensembles_All_A10_scalerr_v0_1_Apr_2016.eps,width=0.50\linewidth,clip=}  \\
    \epsfig{file=figures/PPRs_ensembles_All_P12_scalerr_v0_1_Apr_2016.eps,width=0.50\linewidth,clip=}   &
    \epsfig{file=figures/PPRs_ensembles_All_S13_scalerr_v0_1_Apr_2016.eps,width=0.50\linewidth,clip=}  
  \end{tabular}
  \end{center}
  \caption{Pressure ratios as compared to different sets, plotted as the central 68\% confidence intervals. 
    The ensemble pressure ratios relative to ACCEPT2 
    ($P_{XRAY}$) are calculated per cluster and weighted by the error in the ratio (per radial bin). 
    For the other pressure ratios, 
    the ratio is again calculated per individual cluster, but the comparison pressure profile is the gNFW profile
    for the entire dataset, respectively (i.e. A10, P13, or S13). These ratios are weighted in the same manner.}
  \label{fig:ppr_ensembles}
\end{figure}

%%%%%%%%%%%%%%%%%%%%%%%%%%%%%%%%%%%%%%%%%%%%%%%%%%%%%%%%%%%%%%%%%%%%%%%%%%%%%%%
\subsection{Fits with $\gamma = 0$}
%\subsection{Potential Sources of Error}
\label{sec:pp_error}
%%%%%%%%%%%%%%%%%%%%%%%%%%%%%%%%%%%%%%%%%%%%%%%%%%%%%%%%%%%%%%%%%%%%%%%%%%%%%%%

%%% Look at Jack´s comments. 
We find that 6 of our 14 clusters have pressure profiles, for which $\gamma = 0$, the limit we imposed as a prior, 
produces the best fits. There is no clear segregation based on dynamical state or presence of central point source.
Here, we consider two effects which could spuriously bias the cluster central pressures: our choice of centroid
and the under-subtraction of a central point source.

%Several of our fits results in which $\gamma = 0$, the limit we imposed as a prior.

As it stands, finding slopes in the cores of galaxy clusters that are fit with $\gamma = 0$ is not unprecedented; 
\citetalias{arnaud2010} find six of their 31 analyzed clusters in the REXCESS sample have $\gamma=0$, where all gNFW parameters 
except $\beta$ were fit for individual clusters. They find a similar range in $C_{500}$ as we do, They fit for $\alpha$, 
which is fit by the range $0.3 < \alpha < 2.5$ over their sample. While \citetalias{arnaud2010} is a local ($z < 0.2$) sample,
\citet{mantz2016} find $\gamma=-0.01$, using \emph{Chandra} data, for their sample of 40 galaxy clusters of $0.07 < z < 1.10$.
Moreover, in an analysis of X-ray (\emph{Chandra}) data from observations of 80 clusters, 
\citet{mcdonald2014} find $\gamma=0$ for their low redshift ($0.3 < z < 0.6$), non-cool-core clusters, and similarly shallow 
inner pressure profile slope ($\gamma=0.05$) for the high redshift ($0.6 < z < 1.2$) non-cool-core clusters.
Although these previous studies indicate that $\gamma = 0$ is relatively common, we explore whether systematics related to
either our data or analysis methods may produce this results in our fits.
%Given our analysis and care to not be biased towards $\gamma=0$ and the precedence for other studies to find $\gamma=0$ in
%either individual clusters or cluster ensembles, we conclude that our study is not systematically biased towards $\gamma = 0$.

Shallow slopes in the cores of clusters could be suggestive of a centroid offset either between MUSTANG and Bolocam or
between SZ and X-ray data. Given the MUSTANG and Bolocam pointing accuracies (2\asec and 5\asec, respectively),
it is unlikely that the centroid offsets between MUSTANG and Bolocam are driving the fits to shallow slopes. 
The difference betweenv SZ (Bolocam) centroids and ACCEPT centroids (Table~\ref{tbl:cluster_properties}) 
are large relative to pointing accuracies and thus potentially more important. However, when we adopt Bolocam's centroid 
%(\textcolor{red}{[Is this worth another figure? I'm OK without another figure for this.]}) 
we find negligible change to the SZ pressure profile as compared to adopting the ACCEPT centroid. 

Individually, the Bolocam and MUSTANG data sets yield consistent fits with each other, where changes in best fit 
parameters generally occur along the shallow gradient in confidence intervals (i.e. along the degeneracy
between  $C_{500}$, and $\gamma$).

Additionally, we consider the impact of the assumed flux densities of point sources in the Bolocam maps. 
%There are four clusters where Bolocam assumes a point source flux density: Abell 1835, MACS 0429, RXJ 1347, and MACS 1423. 
There are four clusters (Abell 1835, MACS 0429, RXJ 1347, and MACS 1423) where it was necessary to assume a 140 GHz flux
density in order to analyze the Bolocam data. 
The conversion for $S_{140}$ values from mJy to the equivalent $\mu K_{CMB-amin}$ 
is $\sim20$, which puts the uncertainties of these point sources at 6, 52, 35, and 6 $\mu K_{CMB-amin}$ respectively. 
From Table~\ref{tbl:cluster_obs}, we see that the noise in the Bolocam maps of these clusters are 16.2, 24.1, 19.7, 
and 22.3 $\mu K_{CMB-amin}$ respectively. Thus, for MACS 0429 and RXJ 1347, which have point source flux density 
uncertainties larger than the measurement noise, the assumed point sources could have a noticeable impact on the cluster
fits. However, we must also consider the strength of cluster detections (Table~\ref{tbl:cluster_obs}), 
relative to the point source significances (Table~\ref{tbl:sample_ptsrc}). With the point sources being equally significant
between RXJ 1347 and MACS 0429, but the cluster decrement in RXJ 1347 being detected much more significantly %($>3\times$)
than in MACS 0429, the errors in the assumed point source flux density will impact RXJ 1347 less. 
Therefore, we are left with only MACS 0429 where
we believe that the treatment of the point source may affect our results non-trivially.

Only two (MACS 0429 and MACS 1423) of these four clusters are fit by notably low $\gamma$ values.
In addition, for the remaining four clusters in which MUSTANG detects a point source, the Bolocam maps assume no 
point source contamination. Of these remaining clusters, only MACS 0717 is fit by a notably low $\gamma$, and that is best 
attributed to the dynamics of the cluster (Section~\ref{sec:results_m0717}). 

We also consider that our treatment of point sources in the MUSTANG maps may leave residual emission from point sources,
either due to our fitting procedure or the assumption that our assumed point source has a non-trivial extent. Within 
our fitting procedure, we find little correlation (degeneracy) between the fitted amplitudes of the point source and bulk
ICM, and do find a moderate correlation between the fitted point source amplitude and $\gamma$ \citep{romero2015}, but
this should not bias the fitted point source amplitude. Instead, a compact source which is treated as a point source
would lead to an underestimation of the flux density of a source, and cannot be excluded.

%The sample in \citet{arnaud2010} is a local ($z < 0.2$), flux limited sample, and for their analysis, 
%they have excluded two clusters (a supercluster, Abell 901/902, 
%and a bimodal cluster, RXC J2157.4-0747) from the full REXCESS sample of 33 clusters. \citet{sayers2013} 
%determined pressure profile parameters over a sample of 45 clusters with the redshift range $0.15 < z < 0.89$, 
%where most (60\%) lie between $0.35 < z < 0.59$. They fit profiles to the stack of deprojected pressure 
%profiles, restricting $C_{500}$ to the A10 value, and fit for the other gNFW parameters. 

%%% Add more references!!!

%%%%%%%%%%%%%%%%%%%%%%%%%%%%%%%%%%%%%%%%%%%%%%%%%%%%%%%%%%%%%%%%%%%%%%%%%%%%%%%
%\subsection{Potential Sources of Bias}
%\label{sec:pp_bias}
%%%%%%%%%%%%%%%%%%%%%%%%%%%%%%%%%%%%%%%%%%%%%%%%%%%%%%%%%%%%%%%%%%%%%%%%%%%%%%%

%\textcolor{red}{This subsection would depend on if we think my confidence intervals are in fact biased...}

%%%%%%%%%%%%%%%%%%%%%%%%%%%%%%%%%%%%%%%%%%%%%%%%%%%%%%%%%%%%%%%%%%%%%%%%%%%%%%%%%%%%%%%%%%%%%%%%%%%%%%%%%%%
%%%                  Table of just the ensembles (and A10 to compare against)                           %%%
%%%%%%%%%%%%%%%%%%%%%%%%%%%%%%%%%%%%%%%%%%%%%%%%%%%%%%%%%%%%%%%%%%%%%%%%%%%%%%%%%%%%%%%%%%%%%%%%%%%%%%%%%%%
%
% %\begin{deluxetable*}{l | l l l l l l l l l l l}
%\tabletypesize{\footnotesize}
%\tablecolumns{6}
%\tablewidth{0pt} 
%\tablecaption{Fitted Pressure Profiles for ensembles \label{tbl:pressure_profile_results}}
%\tablehead{
%Sample & $P_0$ & $C_{500}$ & $\alpha$ & $\beta$ & $\gamma$ \\ 
%      & (Mpc) & ($10^{-5}$ Mpc$^2$) & $10^{-3}$ keV cm$^{-3}$         
%}
%\startdata
%All          &  $7.94  \pm 0.10$ & $1.3_{-0.1}^{+0.1}$ & 1.05 & 5.49 & $0.3_{-0.1}^{+0.1}$ \\ 
%Cool Core    &  $3.55  \pm 0.06$ & $0.9_{-0.1}^{+0.1}$ & 1.05 & 5.49 & $0.6_{-0.1}^{+0.1}$ \\
%Disturbed    &  $12.56 \pm 0.29$ & $1.5_{-0.2}^{+0.1}$ & 1.05 & 5.49 & $0.0^{+0.1}$       \\ 
%Well behaved &  $5.34 \pm 0.08$ & $1.2_{-0.1}^{+0.1}$ & 1.05 & 5.49 & $0.5_{-0.1}^{+0.1}$  \\ 
%\hline
%All (A10)    &  $8.403 h_{70}^{-3/2}$ & 1.177 & 1.0510 & 5.4905 & 0.3081 \\
%Cool core (A10) &  $3.249 h_{70}^{-3/2}$ & 1.128 & 1.2223 & 5.4905 & 0.7736 \\
%Disturbed (A10) &  $3.202 h_{70}^{-3/2}$ & 1.083 & 1.4063 & 5.4905 & 0.3798 
%\enddata
%\tablecomments{We have assumed A10 values of $\alpha$ and $\beta$.
%    The findings from A10 are reproduced in the last three rows. The $h_{70}$ dependence is included for explicit replication
%    of A10 results; all $P_0$ values have this dependence (the assumed cosmologies are the same). Well behaved clusters are
%    identified in the next section.}
%\end{deluxetable*}
 % Not really intending to use this.
%
%%%%%%%%%%%%%%%%%%%%%%%%%%%%%%%%%%%%%%%%%%%%%%%%%%%%%%%%%%%%%%%%%%%%%%%%%%%%%%%%%%%%%%%%%%%%%%%%%%%%%%%%%%%
%%%    END OF THAT TABLE!    NOW LET'S PUT IN THE TABLE WE WANT INSTEAD                                 %%%
%%%%%%%%%%%%%%%%%%%%%%%%%%%%%%%%%%%%%%%%%%%%%%%%%%%%%%%%%%%%%%%%%%%%%%%%%%%%%%%%%%%%%%%%%%%%%%%%%%%%%%%%%%%

%%\begin{landscape}

\begin{deluxetable*}{l|lllllllllll}
\tabletypesize{\scriptsize}
\tablecolumns{12}
\tablewidth{\columnwidth} 
\tablecaption{Summary of Fitted Pressure Profiles \label{tbl:pressure_profile_results}}
\tablehead{
\colhead{Cluster} & \colhead{$R_{500}^a$} & \colhead{$Y_{sph}(R_{500})$} & \colhead{$P_{500}^a$} & 
        \colhead{$P_0$} & \colhead{$C_{500}$} & \colhead{$\alpha$} & \colhead{$\beta$} & \colhead{$\gamma$} & 
                  \colhead{$k$} & \colhead{$\tilde{\chi}^2$} &  \colhead{d.o.f.}        \\ 
      & \colhead{(Mpc)} & \colhead{($10^{-5}$ Mpc$^2$)} & \colhead{$10^{-3}$ keV cm$^{-3}$} & \colhead{}  & 
      \colhead{} & \colhead{}  & \colhead{}  & \colhead{}  &   \colhead{}    
}
\startdata
Abell 1835 & 1.49 & $22.50_{-4.49}^{+4.12}$ &  5.94 & $2.15 \pm 0.07$ & $0.77_{-0.17}^{+0.23}$ & 
1.05 & 5.49 & $0.78_{-0.13}^{+0.12}$ & 1.08 & 0.99 & 12880   \\ 
Abell 611  & 1.24 &  $8.14_{-2.21}^{+3.68}$ &  4.45 & $35.43 \pm 2.46$ & $2.00_{-0.30}^{+0.40}$ & 
1.05 & 5.49 & $0.00^{+0.15}$ & 0.96 & 1.02 & 12882   \\ 
MACS 1115  & 1.28 & $20.53_{-3.52}^{+3.84}$ &  5.45 & $0.67 \pm 0.04$ & $0.35_{-0.10}^{+0.15}$ & 
1.05 & 5.49 & $0.87_{-0.27}^{+0.18}$ & 1.11 & 1.04 & 12875   \\ 
MACS 0429  & 1.10 & $19.85_{-3.74}^{+4.00}$ &  4.48 & $11.01 \pm 0.77$ & $0.59_{-0.09}^{+0.11}$ & 
1.05 & 5.49 & $0.00^{+0.15}$ & 1.00 & 1.03 & 12875   \\ 
MACS 1206  & 1.61 & $43.24_{-8.27}^{+8.19}$ & 10.59 & $2.39 \pm 0.10$ & $0.74_{-0.14}^{+0.16}$ & 
1.05 & 5.49 & $0.51_{-0.16}^{+0.14}$ & 1.09 & 1.01 & 12874   \\ 
MACS 0329  & 1.19 & $12.91_{-2.37}^{+2.93}$ &  5.93 & $9.30 \pm 0.50$ & $1.18_{-0.28}^{+0.72}$ & 
1.05 & 5.49 & $0.41_{-0.41}^{+0.19}$ & 1.03 & 0.99 & 12876   \\ 
RXJ1347    & 1.67 & $37.69_{-5.11}^{+5.78}$ & 11.71 & $3.24 \pm 0.08$ & $1.18_{-0.48}^{+1.02}$ & 
1.05 & 5.49 & $0.80_{-0.70}^{+0.30}$ & 1.15 & 0.99 & 12880   \\ 
MACS 1311  & 0.93 & $10.16_{-1.73}^{+1.79}$ &  3.99 & $2.75 \pm 0.22$ & $0.35_{-0.05}^{+0.15}$ & 
1.05 & 5.49 & $0.41_{-0.41}^{+0.34}$ & 0.98 & 1.00 & 12881   \\ 
MACS 1423  & 1.09 &  $8.47_{-2.07}^{+2.53}$ &  6.12 & $22.39 \pm 1.71$ & $1.58_{-0.48}^{+0.22}$ & 
1.05 & 5.49 & $0.00^{+0.35}$ & 1.04 & 0.98 & 12876   \\ 
MACS 1149  & 1.53 & $42.77_{-5.67}^{+4.99}$ & 12.28 & $5.50 \pm 0.25$ & $0.83_{-0.03}^{+0.07}$ & 
1.05 & 5.49 & $0.00^{+0.05}$ & 0.87 & 1.00 & 13584   \\ 
MACS 0717  & 1.69 & $43.44_{-8.00}^{+9.28}$ & 14.90 & $21.28 \pm 0.68$ & $1.97_{-0.37}^{+0.53}$ & 
1.05 & 5.49 & $0.00^{+0.25}$ & 0.48 & 1.04 & 12876   \\ 
MACS 0647  & 1.26 & $26.22_{-4.72}^{+5.37}$ &  9.23 & $2.78 \pm 0.11$  & $0.70_{-0.20}^{+0.30}$ & 
1.05 & 5.49 & $0.60_{-0.20}^{+0.15}$ & 1.14 & 1.01 & 12876   \\ 
MACS 0744  & 1.26 & $12.59_{-2.29}^{+3.18}$ & 11.99 & $13.15 \pm 0.81$ & $1.71_{-0.21}^{+0.29}$ & 
1.05 & 5.49 & $0.00^{+0.15}$ & 0.90 & 1.02 & 12875   \\ 
CLJ1226    & 1.00 &  $9.03_{-1.60}^{+2.03}$ & 11.84 & $19.29 \pm 1.25$ & $1.90_{-0.50}^{+0.60}$ & 
1.05 & 5.49 & $0.29_{-0.29}^{+0.36}$ & 0.92 & 1.03 & 12875   \\ 
\hline
All          &  --    &  --    &  --    &  $7.94  \pm 0.10$ & $1.3_{-0.1}^{+0.1}$ & 1.05 & 5.49 & $0.3_{-0.1}^{+0.1}$ & -- & -- & -- \\ 
Cool Core    &  --    &  --    &  --    &  $3.55  \pm 0.06$ & $0.9_{-0.1}^{+0.1}$ & 1.05 & 5.49 & $0.6_{-0.1}^{+0.1}$ & -- & -- & -- \\
Disturbed    &  --    &  --    &  --    &  $12.56 \pm 0.29$ & $1.5_{-0.2}^{+0.1}$ & 1.05 & 5.49 & $0.0^{+0.1}$       & -- & -- & -- \\ 
Well behaved &  --    &  --    &  --    &  $5.34 \pm 0.08$ & $1.2_{-0.1}^{+0.1}$ & 1.05 & 5.49 & $0.5_{-0.1}^{+0.1}$  & -- & -- & -- \\ 
\hline
All (A10)    &  --    &  --    &  --    &  $8.403 h_{70}^{-3/2}$ & 1.177 & 1.0510 & 5.4905 & 0.3081 & -- & -- & -- \\
Cool core (A10) &  --    &  --    &  --    &  $3.249 h_{70}^{-3/2}$ & 1.128 & 1.2223 & 5.4905 & 0.7736 & -- & -- & -- \\
Disturbed (A10) &  --    &  --    &  --    &  $3.202 h_{70}^{-3/2}$ & 1.083 & 1.4063 & 5.4905 & 0.3798 & -- & -- & --
\enddata
\tablecomments{Results from our pressure profile analysis. $Y_{sph}$ is calculated using the tabulated value of $R_{500}$.
    $^a$Values of $R_{500}$ and $P_{500}$ are taken from \citet{sayers2013}. We have assumed A10 values of $\alpha$ and $\beta$.
    The findings from A10 are reproduced in the last three rows. The $h_{70}$ dependence is included for explicit replication
    of A10 results; all $P_0$ values have this dependence (the assumed cosmologies are the same).}
\end{deluxetable*}

%\end{landscape}
 % reference: {tbl:pressure_profile_results}
%%% Maybe \include does something I don't want it to?
\begin{deluxetable*}{l|llllllllllll}
\tabletypesize{\footnotesize}
\tablecolumns{13}
%\tablewidth{\columnwidth} 
\tablewidth{0pt} 
\tablecaption{Summary of Fitted Pressure Profiles \label{tbl:pressure_profile_results}}
\tablehead{
\colhead{Cluster} & \colhead{$R_{500}^a$} & \colhead{$Y_{cyl}(R_{500})$} & \colhead{$Y_{sph}(R_{500})$} & \colhead{$10^3 P_{500}^a$} & 
        \colhead{$P_0$} & \colhead{$C_{500}$} & \colhead{$\alpha$} & \colhead{$\beta$} & \colhead{$\gamma$} & 
                  \colhead{$k$} & \colhead{$\tilde{\chi}^2$} &  \colhead{d.o.f.}        \\ 
      & \colhead{(Mpc)} & \colhead{($10^{-5}$ Mpc$^2$)} & \colhead{($10^{-5}$ Mpc$^2$)} & \colhead{keV cm$^{-3}$} & \colhead{}  & 
      \colhead{} & \colhead{}  & \colhead{}  & \colhead{}  &   \colhead{}    
}
\startdata
Abell 1835 & 1.49 & $26.75_{-6.15}^{+6.05}$ & $21.81_{-4.49}^{+4.12}$ &  5.94 & $2.15 \pm 0.07$ & $0.77_{-0.17}^{+0.23}$ & 
1.05 & 5.49 & $0.78_{-0.13}^{+0.12}$ & 1.08 & 0.99 & 12880   \\ 
Abell 611  & 1.24 & $9.67_{-2.57}^{+4.85}$ & $8.73_{-2.21}^{+3.68}$ &  4.45 & $35.43 \pm 2.46$ & $2.00_{-0.30}^{+0.40}$ & 
1.05 & 5.49 & $0.00^{+0.15}$ & 0.96 & 1.02 & 12882   \\ 
MACS 1115  & 1.28 & $30.28_{-6.30}^{+7.32}$ & $20.10_{-3.52}^{+3.84}$ &  5.45 & $0.67 \pm 0.04$ & $0.35_{-0.10}^{+0.15}$ & 
1.05 & 5.49 & $0.87_{-0.27}^{+0.18}$ & 1.11 & 1.04 & 12875   \\ 
MACS 0429  & 1.10 & $30.41_{-6.88}^{+7.72}$ & $19.57_{-3.74}^{+4.00}$ &  4.48 & $11.01 \pm 0.77$ & $0.59_{-0.09}^{+0.11}$ & 
1.05 & 5.49 & $0.00^{+0.15}$ & 1.00 & 1.03 & 12875   \\ 
MACS 1206  & 1.61 & $61.52_{-12.63}^{+12.49}$ & $48.16_{-8.27}^{+8.19}$ & 10.59 & $2.39 \pm 0.10$ & $0.74_{-0.14}^{+0.16}$ & 
1.05 & 5.49 & $0.51_{-0.16}^{+0.14}$ & 1.09 & 1.01 & 12874   \\ 
MACS 0329  & 1.19 & $13.38_{-2.99}^{+3.83}$ & $11.86_{-2.37}^{+2.93}$ &  5.93 & $9.30 \pm 0.50$ & $1.18_{-0.28}^{+0.72}$ & 
1.05 & 5.49 & $0.41_{-0.41}^{+0.19}$ & 1.03 & 0.99 & 12876   \\ 
RXJ1347    & 1.67 & $42.47_{-6.81}^{+8.29}$ & $37.80_{-5.11}^{+5.78}$ & 11.71 & $3.24 \pm 0.08$ & $1.18_{-0.48}^{+1.02}$ & 
1.05 & 5.49 & $0.80_{-0.70}^{+0.30}$ & 1.15 & 0.99 & 12880   \\ 
MACS 1311  & 0.93 & $17.18_{-3.49}^{+3.80}$ & $10.08_{-1.73}^{+1.79}$ &  3.99 & $2.75 \pm 0.22$ & $0.35_{-0.05}^{+0.15}$ & 
1.05 & 5.49 & $0.41_{-0.41}^{+0.34}$ & 0.98 & 1.00 & 12881   \\ 
MACS 1423  & 1.09 & $10.35_{-2.73}^{+4.00}$ & $8.89_{-2.07}^{+2.53}$ &  6.12 & $22.39 \pm 1.71$ & $1.58_{-0.48}^{+0.22}$ & 
1.05 & 5.49 & $0.00^{+0.35}$ & 1.04 & 0.98 & 12876   \\ 
MACS 1149  & 1.53 & $56.87_{-9.00}^{+8.04}$ & $41.62_{-5.67}^{+4.99}$ & 12.28 & $5.50 \pm 0.25$ & $0.83_{-0.03}^{+0.07}$ & 
1.05 & 5.49 & $0.00^{+0.05}$ & 0.87 & 1.00 & 13584   \\ 
MACS 0717  & 1.69 & $64.50_{-10.18}^{+12.42}$ & $55.72_{-8.00}^{+9.28}$ & 14.90 & $21.28 \pm 0.68$ & $1.97_{-0.37}^{+0.53}$ & 
1.05 & 5.49 & $0.00^{+0.25}$ & 0.48 & 1.04 & 12876   \\ 
MACS 0647  & 1.26 & $34.06_{-7.76}^{+10.21}$ & $26.33_{-4.72}^{+5.37}$ &  9.23 & $2.78 \pm 0.11$  & $0.70_{-0.20}^{+0.30}$ & 
1.05 & 5.49 & $0.60_{-0.20}^{+0.15}$ & 1.14 & 1.01 & 12876   \\ 
MACS 0744  & 1.26 & $15.10_{-3.01}^{+4.50}$ & $13.20_{-2.29}^{+3.18}$ & 11.99 & $13.15 \pm 0.81$ & $1.71_{-0.21}^{+0.29}$ & 
1.05 & 5.49 & $0.00^{+0.15}$ & 0.90 & 1.02 & 12875   \\ 
CLJ1226    & 1.00 & $10.50_{-1.94}^{+2.65}$ & $9.46_{-1.60}^{+2.03}$ & 11.84 & $19.29 \pm 1.25$ & $1.90_{-0.50}^{+0.60}$ & 
1.05 & 5.49 & $0.29_{-0.29}^{+0.36}$ & 0.92 & 1.03 & 12875   \\ 
\hline
All          &  --    &  --    &  --    &  --    &  $7.94  \pm 2.10$ & $1.3_{-0.1}^{+0.1}$ & 1.05 & 5.49 & $0.3_{-0.1}^{+0.1}$ & -- & -- & -- \\ 
Cool Core    &  --    &  --    &  --    &  --    &  $3.55  \pm 1.53$ & $0.9_{-0.1}^{+0.1}$ & 1.05 & 5.49 & $0.6_{-0.1}^{+0.1}$ & -- & -- & -- \\
Disturbed    &  --    &  --    &  --    &  --    &  $12.56 \pm 2.07$ & $1.5_{-0.2}^{+0.1}$ & 1.05 & 5.49 & $0.0^{+0.1}$       & -- & -- & -- \\ 
% Well behaved &  --    &  --    &  --    &  --    &  $5.34 \pm 0.08$ & $1.2_{-0.1}^{+0.1}$ & 1.05 & 5.49 & $0.5_{-0.1}^{+0.1}$  & -- & -- & -- \\ 
\hline
All (A10)    &  --    &  --    &  --    &  --    &  $8.403 h_{70}^{-3/2}$ & 1.18 & 1.05 & 5.49 & 0.31 & -- & -- & -- \\
Cool core (A10) &  --    &  --    &  --    &  --    &  $3.249 h_{70}^{-3/2}$ & 1.13 & 1.22 & 5.49 & 0.78 & -- & -- & -- \\
Disturbed (A10) &  --    &  --    &  --    &  --    &  $3.202 h_{70}^{-3/2}$ & 1.08 & 1.41 & 5.49 & 0.38 & -- & -- & --
\enddata
\tablecomments{Results from our pressure profile analysis. $Y_{sph}$ is calculated using the tabulated value of $R_{500}$.
    $^a$Values of $R_{500}$ and $P_{500}$ are taken from \citet{sayers2013}. We have assumed A10 values of $\alpha$ and $\beta$.
    The findings from A10 are reproduced in the last three rows. The $h_{70}$ dependence is included for explicit replication
    of A10 results; all $P_0$ values have this dependence (the assumed cosmologies are the same).}
\end{deluxetable*}

%%%%%%%%%%%%%%%%%%%%%%%%%%%%%%%%%%%%%%%%%%%%%%%%%%%%%%%%%%%%%%%%%%%%%%%%%%%%%%%%%%%%%%%%%%%%%%%%%%%%%%%%%%%
%%%                                             END OF THAT TABLE!                                      %%%
%%%%%%%%%%%%%%%%%%%%%%%%%%%%%%%%%%%%%%%%%%%%%%%%%%%%%%%%%%%%%%%%%%%%%%%%%%%%%%%%%%%%%%%%%%%%%%%%%%%%%%%%%%%

%%%%%%%%%%%%%%%%%%%%%%%%%%%%%%%%%%%%%%%%%%%%%%%%%%%%%%%%%%%%%%%%%%%%%%%%%%%%%%%%%%%%%%%%%%%%%%%%%%%%%%%%%%%
%%%                                                NEXT SECTION                                         %%%
%%%%%%%%%%%%%%%%%%%%%%%%%%%%%%%%%%%%%%%%%%%%%%%%%%%%%%%%%%%%%%%%%%%%%%%%%%%%%%%%%%%%%%%%%%%%%%%%%%%%%%%%%%%

\section{Integrated Compton $Y$ Scaling Relations}

%\textcolor{red}{I need to introduce integrated Y a bit more I think.}

We calculate integrated Compton $Y$ values at $R_{500}$ due to the expected minimal scatter at intermediate 
radii \citep[e.g.][]{kravtsov2012}. We use $R_{500}$ derived from X-ray observations \citep{mantz2010}, 
and calculate $Y_{sph}$, given by:
\begin{equation}
  Y_{sph}(R) = \frac{\sigma_T}{m_e c^2} \int_0^R P(r') 4 \pi r'^2 dr' 
  \label{eqn:ysph}
\end{equation}
and $Y_{cyl}$, which is given by:
\begin{equation}
  Y_{cyl}(R) = \frac{\sigma_T}{m_e c^2} \int_0^R  2 \pi r dr \int_r^{R_b} \frac{2 r' P(r') dr'}{\sqrt{r'^2 - r^2}},
  \label{eqn:ycyl}
\end{equation}
where we adopt $R_b = R_{500}$ as in \citetalias{arnaud2010}.
The error bars on $Y_{sph}(R_{500})$ and $Y_{cyl}(R_{500})$ are found by calculating the respective quantities 
from the pressure profile fits over the 1000 noise realizations, and taking the values encompassing the middle 68\%. 
We take $M_{500}$ from \citet{mantz2010}, who arrive at $M_{500}$ in the following steps: (1) calculate $R_{500}$ using 
a ratio of $R_{500} / R_{2500} \sim 2.3$ assuming a NFW profile with concentration parameter $c=4$, (2) calculate 
$M_{gas,500}$, the total gas mass enclosed in $R_{500}$ from deprojected gas mass (non-parametric) profiles, (3) 
determine the gas mass fraction, $f_{gas}(r_{500})$, by fitting a power law model to $f_{gas}(r)$ from simulations, and (4)
calculate $M_{500}$ as $(M_{gas}(r_{500})) ((1+B) f_{gas}(r_{500}))^{-1}$, where $B = 0.03 \pm 0.06$ is a systematic fractional 
bias. \citet{mantz2010} note that the dominant source of systematic uncertainty associated with $M_{500}$ comes from the 
uncertainty in the assumed $f_{gas}(r_{2500}) = 0.1104$, which was used in calibrating $f_{gas}(r_{500}) \approx 0.115$.

We compare our $Y_{sph}(R_{500}) - M_{500}$ relation to that of A10 in Figure~\ref{fig:ysph_scaling}.
The $Y_{sph} - M_{500}$ scaling relation calculated in \citetalias{arnaud2010} is given as:
\begin{equation}
%  h(z)^{-2/3} Y_{sph}(x R_{500}) = A_x \left{[}\frac{M_{500}}{3 \times 10^{14} h_{70}^{-1} M_{\odot}} \right{]}^{\alpha} ,
  h(z)^{-2/3} Y_{sph}(x R_{500}) = A_x \left[ \frac{M_{500}}{3 \times 10^{14} h_{70}^{-1} M_{\odot}} \right] ^{\alpha} ,
  \label{eqn:ysph_scaling}
\end{equation}
where $\alpha = 1.78$, $A_X = 2.925 \times 10^{-5} I(x) h_{70}^{-1}\text{Mpc}^2$, and $I(1) = 0.6145$.
and find six of fourteen clusters which are more than $2 \sigma$ in $Y_{sph}$ from the scaling relation.
When we consider the mass uncertainty, that number drops to three. While our sample size is small, the
tendency of cool core clusters to lie above the scaling relation and of disturbed clusters to lie below the
scaling relation is interesting. Regardless of cluster type, our sample does show a more shallow $Y_{sph,500}-M_{500}$ slope
($1.06\pm0.13$) than the predicted self similar slope ($5/3$) or $1.78$ found in \citetalias{arnaud2010}. 
This is consistent with the slope found for the BOXSZ sample by \citet{czakon2015} for $Y_{cyl,2500} - M_{2500}$ of $1.06\pm0.12$.

% I can talk about a shallower slope...but then I should fit a slope to our data. Ugh...OK, that probably is good to do.
% Also Y_cyl - M_500.

%\textcolor{red}{I am still thinking about potential biases / if the tendencies of CC clusters above the scaling relation and
%disturbed clusters below the scaling relation is significant.}

%\textcolor{red}{The bias of the (two) cool core clusters above the scaling relation is fundamentally caused by the shallower
%slope at large radii for these two clusters. The only way to achieve a shallow slope at larger radii with $\alpha$ and $\beta$ 
%fixed is then to drive $C_{500}$ to low values (and keep $\gamma$ relatively low too). We, in fact, see this. MACS 1311 and MACS
%0429, the two ``offending'' clusters, have two of the three lowest $C_{500}$ values, where MACS 1115 has the lowest, but is also
%fit with a relatively high $\gamma$ value. The question then becomes, what is driving the pressure high at larger radii? And
%subsequently, is it just coincidence that these two clusters are cool core clusters? 
%Or colloquially, \emph{what's up with that?}.
%Finally, with only four disturbed clusters, one of which is also a cool core cluster, it's tempting to see the reverse trend -
%that disturbed clusters are falling below the relation, and cool core clusters are falling above the scaling relation. I'm less
%inclined to make a point of this, but perhaps it's worth mentioning.}

%\textcolor{red}{I would suspect the source of elevated pressure at large radii is largely assymetries seen at these larger radii. 
%That is, looking at the maps with X-ray and Bolocam contours, we see extended pressure, 
%which is not strictly circular about the [either] centroid. A
%gain, on centroids, as discussed in Section~\ref{sec:pp_error}, there is generally not a significant difference in 
%fitted parameters, and specifically for these two clusters, there is visually \textbf{no} difference when comparing fits 
%(i.e. confidence intervals) using the ACCEPT centroid to the fits using the Bolocam centroid. 
%Thus, I have to return to the notion that there \textbf{is} some
%pressure that SZ observations are finding at large radii that the X-ray observations are not. This higher pressure is not seen
%globally (it's not symmetrically present), but arises assymetrically and could be due to locally elevated temperatures at large
%radii, to which X-rays are not sensitive.}

%\textcolor{red}{I wonder if this tendency is not seen in X-ray data as much simply because exposure times for cool core clusters
%may be restricted. That is, perhaps the cost to measure the pressure at these radii is too much for what proposers have expected
%to find, especially when they can constrain the pressure profile in the core very well in a much shorter time, relative to what
%would be required for these larger radii.}


%%%%%%%%%%%%%%%%%%%%%%%%%%%%%%%%%%%%%%%%%%%%%%%%%%%%%%%%%%%%%%%%%%%%%%%%%%%%%%%%%%%%%%%%%%%%%%%%%%%%%%%%%%%
%%%                                                SOME FIGURES                                         %%%
%%%%%%%%%%%%%%%%%%%%%%%%%%%%%%%%%%%%%%%%%%%%%%%%%%%%%%%%%%%%%%%%%%%%%%%%%%%%%%%%%%%%%%%%%%%%%%%%%%%%%%%%%%%

\begin{figure}[!ht]
  \begin{center}
%  \includegraphics[width=1.0\textwidth]{analysis_figures/ysph_scaling_relation_2sigma}
%  \includegraphics[width=0.5\textwidth]{figures/ysph-M_all_color_coded_unlabeled_connected.eps}
%  \includegraphics[width=0.5\textwidth]{figures/ysph-M_all_color_coded_unlabeled_accept_14_Jan_2016_connected.eps}
  \includegraphics[width=0.5\textwidth]{figures/ysph-M_all_color_coded_unlabeled_xray_18_Jul_2016_connected.eps}
  \end{center}
  \caption{$Y_{sph,SZ}(R_{500})$ as calculated in this work (Table~\ref{tbl:pressure_profile_results}),
  and $M_{500}$ as calculated from \citet{mantz2010}. The scaling relation (dashed line) and triangles
  are from \citet{arnaud2010} and \citet{pratt2010}. The diamonds are $Y_{sph,X}(R_{500})$ as calculated from the gNFW fits
  to the ACCEPT2 pressure profiles. MACS 1311 and MACS 0429 are the notable outliers above the scaling relation.}
% The labelled clusters are those whose $Y_{sph}(R_{500})$ values lie
%  more than $2\sigma$ from the scaling relation from \citet{arnaud2010}, the dot-dashed line.
  \label{fig:ysph_scaling}
\end{figure}

%%%%%%%%%%%%%%%%%%%%%%%%%%%%%%%%%%%%%%%%%%%%%%%%%%%%%%%%%%%%%%%%%%%%%%%%%%%%%%%
\section{Combining SZ and X-ray Data}
\label{sec:xray_comp}
%%%%%%%%%%%%%%%%%%%%%%%%%%%%%%%%%%%%%%%%%%%%%%%%%%%%%%%%%%%%%%%%%%%%%%%%%%%%%%%

The observed SZ and X-ray signal from galaxy clusters differ in their 
dependence upon the physical properties in the intracluster medium (ICM). This difference
has, in the past, been exploited to make calculations of the Hubble parameter, $H_0$, 
assuming spherical geometry of galaxy clusters. However, one can relax the spherical assumption
and use the differences in SZ and X-ray inferred quantities to calculate cluster elongation
along the line of sight, helium sedimentation, or (recalculate) the ICM electron temperature.
These are all degenerate (i.e. these cannot all be independently constrained). We investigate
each constraint individually, and conclude that differences in the SZ and X-ray spherically derived
pressure profiles are due to some combination cluster elongation and ICM temperature distribution.

We choose to compare our SZ data (primarily the pressure profiles) to the ACCEPT2 catalog, accessed
via private communication with Megan Donahue and Alessandro Baldi, which sought to add to the ACCEPT 
catalog sample size and properties reported \citep{baldi2014}. The ACCEPT2 catalog
includes any new public \emph{Chandra} observations since the ACCEPT catalog. We correct for the
difference in cosmologies assumed in our SZ analysis and that used in ACCEPT2.

When we assume the X-ray derived temperatures, we find most clusters to be elongated along the line of sight,
which is perhaps a result of the CLASH sample selection. If, on the other hand, we assume the clusters are
spherical, then our SZ data combined with ACCEPT2 data imply temperatures systematically higher than
X-ray derived temperatures (this augmentation occurs at larger radii). Such a trend could be indicative of
a systematic calibration offset in either (or both) dataset(s): the SZ calibration may be biased high, 
or the X-ray calibration may be biased low.

%with a geometric mean of $2.7$.
%%% Add some general discussion of SZ vs. ACCEPT2?

\subsection{Ellipsoidal Geometry}
\label{sec:ellgeo}

%%% Many comments from CLS.
The geometry of a cluster along the line of sight can be calculated by comparing SZ and
X-ray pressure profiles.  If we assume azimuthal symmetry in the plane of the sky with scale radius $\theta_{proj}$
and a scale radius along the line-of-sight of $\theta_{los}$, then we denote the elongation/compression 
along the line-of-sight with an axis ratio $c = \theta_{los} / \theta_{proj}$, where $c > 1$ implies that the cluster is 
longer along the line-of-sight than in the plane of the sky.
%The X-ray signal is proportional to $n_e^2 \Lambda_{ee}(T,Z)$,
The X-ray surface brightness is proportional to $\int n_e^2 \Lambda(T,Z) \, dl \propto \int (P/T)^2 \Lambda(T,Z) \, dl$,
where $Z$ is the abundance of heavy elements and $\Lambda$ is the X-ray cooling function, 
%while SZ signal is proportional to $n_e T$.
while the SZ signal is proportional to $\int P dl$ (Equation~\ref{eqn:compton_y}).
%We make the simplifying assumption that we can concern ourselves solely with the electron distributions
%as determined by SZ and X-ray observations ($n_{e,SZ}$ and $n_{e,X}$ respectively)
%so that we find $n_{e,SZ} / n_{e,X} = c^{1/2}$. If the temperature distribution determined by ACCEPT2 is
%assumed to be true, then we have $P_{SZ} / P_{X} = c^{1/2}$
The temperature $T$ can be derived from X-ray. Initially, we will assume that the cluster is spherically
symmetric, and derive the pressure profile from the X-ray observations (giving $P_{X}$), and from the SZ
observations (giving $P_{SZ}$). If the pressure profiles disagree, one explanation would be the elongation
of the cluster along the line-of-sight. In this case, the elongation is given by $c = (P_{SZ} / P_{X})^2$.


%We take $n_e$ as the true electron density, and let 
%it be a function of ellipsoidal radius, $E^2 = \frac{x^2}{a^2} + \frac{y^2}{b^2} + \frac{z^2}{c^2}$, 
%and $\rho = \sqrt{\frac{x^2}{a^2} + \frac{y^2}{b^2}}$ is the radius in the plane of the sky.
%We take our observable as $O(\rho) = \int_{-\infty}^{\infty} Q(\rho, z) dz$, with $Q(\rho, z)$ being our source function.
%For the spherical case (subscripted with $S$), we assume $a=b=c=1$. In finding the geometry along the line
%of sight, we allow $c$ to vary, but keep $a=b=1$. Given the our observable is fixed,
%we have $Q_s(\rho, z) = Q(\rho, z) / c$. For X-ray observations, our source function has the proportionality
%$Q_X \propto n_e^2 \Lambda_{ee}(T_e,Z)$, where $Z$ is the abundance of heavy elements, while for SZ, the source function 
%has the proportionality $Q_{SZ} \propto n_e T_e$. 

%We define $\eta = \frac{a}{c} = \frac{b}{c}$. 
%We define $\eta = c^{-1}$, thus prolate clusters (along the line of sight) have $\eta < 1$.
%Noting that $\eta$ should be independent of $r$,
%we see that it can be ascertained through the normalization of the electron density profiles,
%or similarly, the electron pressure profiles: $\frac{P_{e,SZ}}{P_{e,X}} = \eta^{-1/2}$. 
%We can sum this up as saying that a prolate cluster (along
%the line of sight, $\eta < 1$) will yield greater pressure as derived from SZ than from X-ray,
%whereas an oblate cluster (along the line of sight) will do the opposite.

To estimate the ellipticity of clusters, we fit the ACCEPT2 pressure profiles with
a gNFW pressure profile, with $\alpha$ and $\beta$ fixed at their A10 values: 1.05 and 5.49,
respectively. 
%These are tabulated in Table~\ref{tbl:accept_gnfw}. 
The resultant gNFW profile
is then integrated along the line of sight (LOS) to create a Compton $y$ map, and then filtered as 
discussed in \citet{romero2015a}. We use amplitude of the ACCEPT2 model fitted to SZ data
$P_{SZ}$ to estimate the geometry of the cluster.

The axis ratio is calculated as $c = P_{SZ}^{2}$, and its associated uncertainty
is calculated as 
$\sigma_{c}^2 = 4 P_{SZ}^{4} ((\sigma_{SZ}/P_{SZ})^2 + (\sigma_{A}/P_{A})^2)$, 
where $(\sigma_{SZ}/P_{SZ}) = 0.11$ and $(\sigma_{A}/P_{A}) = 0.10$ 
are the calibration uncertainties of Bolocam and ACCEPT2 respectively. 
Table~\ref{tbl:accept_gnfw} presents relevant fitted gNFW parameters used in calculating
the cluster geometry.


\begin{deluxetable*}{l | l l l | l l l | l l | l }
\tabletypesize{\footnotesize}
\tablecolumns{10}
\tablewidth{0pt} 
\tablecaption{ACCEPT2 gNFW Fitted Parameters and Comparison to SZ data \label{tbl:accept_gnfw}}
\tablehead{
    Cluster & $P_0$ & $C_{500}$ & $\gamma$ & $P_{SZ}$ & k & $P_{B}$ & $c$ & $\sigma_{c}$ & $\Delta P$ ($\sigma$)
}
%\begin{table}
%  \centering
%  \begin{tabular}{l l l l l | l l l | l l | l }
%    Cluster & $P_0$ & $C_{500}$ & $\gamma$ & $P_{SZ}$ & k & $P_{B}$ & $c$ & $\sigma_c$ & $\Delta P$ ($\sigma$) \\
%    \hline   
\startdata
  Abell 1835 &  8.1 &  1.4 & 0.44 & $0.83\pm0.03$ & 1.15 & $0.82\pm0.03$ & 0.69 & 0.16 &  0.34 \\
   Abell 611 & 15.3 &  0.9 & 0.62 & $1.31\pm0.09$ & 0.92 & $1.35\pm0.10$ & 1.71 & 0.45 &  0.32 \\
   MACS 1115 &  8.2 &  1.5 & 0.35 & $0.84\pm0.05$ & 1.14 & $0.80\pm0.05$ & 0.70 & 0.17 &  0.51 \\
   MACS 0429 &  6.0 &  1.0 & 0.71 & $1.30\pm0.11$ & 0.64 & $1.48\pm0.11$ & 1.70 & 0.47 &  1.12 \\
   MACS 1206 &  8.3 &  1.0 & 0.49 & $1.12\pm0.04$ & 1.01 & $1.11\pm0.04$ & 1.24 & 0.29 & -0.02 \\
   MACS 0329 & 18.6 &  1.2 & 0.59 & $1.61\pm0.09$ & 0.90 & $1.64\pm0.09$ & 2.58 & 0.64 &  0.30 \\
    RXJ 1347 & 18.3 &  2.4 & 0.40 & $0.95\pm0.02$ & 1.18 & $0.94\pm0.02$ & 0.89 & 0.20 &  0.26 \\
   MACS 1311 &  5.8 &  1.6 & 0.26 & $1.28\pm0.12$ & 0.85 & $1.40\pm0.12$ & 1.64 & 0.47 &  0.69 \\
    MACS1423 & 12.1 &  1.8 & 0.51 & $1.26\pm0.11$ & 0.81 & $1.39\pm0.12$ & 1.58 & 0.44 &  0.82 \\
   MACS 1149 &  3.6 &  0.9 & 0.23 & $1.24\pm0.06$ & 0.70 & $1.28\pm0.06$ & 1.54 & 0.37 &  0.55 \\
   MACS 0717 & 21.8 &  1.5 & 0.00 & $1.36\pm0.04$ & 0.71 & $1.39\pm0.04$ & 1.85 & 0.43 &  0.38 \\
   MACS 0647 &  6.2 &  0.9 & 0.54 & $1.29\pm0.05$ & 1.09 & $1.27\pm0.05$ & 1.67 & 0.40 & -0.28 \\
   MACS 0744 &  5.9 &  0.6 & 0.93 & $1.04\pm0.06$ & 0.94 & $1.05\pm0.06$ & 1.08 & 0.27 &  0.18 \\
    CLJ 1226 &  6.0 &  1.3 & 0.04 & $0.64\pm0.04$ & 1.15 & $0.60\pm0.04$ & 0.41 & 0.11 &  0.70 
\enddata
\tablecomments{gNFW fits to the ACCEPT2 pressure profiles. $P_{SZ}$ denotes the fitted
  amplitude (renormalization) of the ACCEPT2 model to the SZ data. $P_{B}$ denotes the
  fitted amplitude of the ACCEPT2 model to just Bolocam data. the gNFW parameters $\alpha$ and
  $\beta$ are fixed at A10 values of 1.05 and 5.49.  The elongation $c$ is the ratio between the scale radius
  along the line-of-sight and the projected scale radius (taken to be azimuthally symmetric in the plane of the sky). 
  Positive values in the column $\Delta P$ ($\sigma$) indicate that the
  core is more spherical than the extended cluster. $\Delta P$ is calculated as 
  $(P_{SZ} - P_{B})$, with uncertainty $(\sigma_{P_{SZ}}^2 + \sigma_{P_{B}}^2)^{1/2}$, and sign depending on the
  $P_{B}$ relative to 1.}
% The column $\Delta k$ ($\sigma$) lists the 
%  significances of a more spherical core, as compared to the outer regions. $\Delta k$ was calculated
%  as the difference between the $k$ in this table (column 5), and the values listed in
%  Table~\ref{tbl:pressure_profile_results}. A negative $\Delta k$ value
%  signifies that the core is more ellipsoidal than the outer regions.}
\end{deluxetable*}

The CLASH sample contains X-ray (20) and lensing (5) selected clusters and was not explicitly designed to
be orientation unbiased. It is, therefore, not too surprising that we find indications that many of the clusters 
in our sample are elongated along the line of sight ($c > 1$).
Abell 1835 is not in the CLASH sample, but is a notably well studied cool core cluster, i.e. it is the subject 
of many studies on the basis of its cool core.

This investigation has made the assumption that the geometry of a given cluster is globally consistent.
That is, one ellipsoidal geometry applies to all regions of the cluster. However, a cluster should appear 
more spherical towards the center, where baryons have
condensed \citep[e.g.][and references therein]{kravtsov2012}. Also, the DM and baryonic distributions
need not align (one need only look at the Bullet cluster \citep{Markevitch2004} for a dramatic example).
This is not a particular concern to this analysis as we are comparing quantities based on the baryonic
distribution, but would be more of a concern when including lensing. 

Across our sample, we find an average pressure ratio $\langle P_{SZ} \rangle = 1.14 \pm 0.25$, where a portion
($0.07$) of the uncertainty is from the scatter in the values. 
In elongation, we find $\langle c \rangle = 1.38 \pm 0.58$, where again,
$0.33$ is from the scatter. \citet{battaglia2012} find minor-to-major axis ratios $\simeq 0.85$-$0.9$ based on
pressure distributions at $\sim R_{500}$. This ratio has some dependence on cluster mass and redshift, where in
both cases the deviations from unity grow with increasing mass and with increasing redshift. As we have a 
heterogenous sample, there is no easy comparison with the expected scatter. As a starting point, we take the uncertainty
for the $z = 0.5$ bin to be $\sim 0.05$. This would then support potential values $0.8 < c < 1/0.8$, which
produces tension for our results.

However, working with a smaller sample size (16 clusters) and higher resolution, 
\citet{lau2011} find smaller values for minor-to-major axis gas ratios taken at $R_{500}$: 
at $z=0.6$, they find values of $0.80 \pm 0.04$ and $0.72 \pm 0.05$ for the gas in relaxed clusters 
(for two different modes of simulation). For all clusters at $z = 0.6$, these values are $0.69 \pm 0.05$ and
$0.66 \pm 0.04$ and correspond to elongations of 1.45 and 1.5, respectively. Accounting for our
uncertainties, only MACS 0329 and CLJ 1226 are outside (by $1\sigma$) of the range found in the literature.
%Noting that these last values correspond to major-to-minor axis ratios (elongations) 
%of 1.45 and 1.5, and our uncertainties, only MACS 0329 and CLJ 1226 are outside of range found in the literature.

We take the difference between $P_{B}$ and $P_{SZ}$ to be indicative that the core has a different elongation
than ICM at moderate to large radii. In particular, for $P_{B} < 1$, then $P_{SZ} > P_{B}$ is indicative of a
more spherical core and for $P_{B} > 1$, then a more spherical core will have $P_{SZ} < P_{B}$. The statistical
significances of core sphericity (relative to the region outside the center) are tabulated in 
Table~\ref{tbl:pressure_profile_results}. While none of our determinations individually are above $3\sigma$, 
it is nonetheless interesting to note the tendency for core sphericity.

%One way to infer a difference in geometries between the inner and outer regions is to use the (multiplicative) calibration
%offset between Bolocam and MUSTANG in our fits. In almost all cases, we find that $k$ tends to be inversely related to 
%$P_{B}$, which suggests that the central pressure distribution is more spherical than the outer pressure distribution, 
%as shown in Table~\ref{tbl:accept_gnfw}, under $\Delta k$($\sigma$). However, we must attempt to account for the true
%calibration offset. Taking our values from Table~\ref{tbl:pressure_profile_results} to be our best estimate of the true
%calibration offset, we calculate $\Delta k$($\sigma$) using the uncertainty prior of $12$\% and modulate the
%sign such that positive $\Delta k$($\sigma$) indicates that the core is more spherical than the cluster at large scales.

%A simple, but not very robust, estimate of this significance of the signature
%is found by comparing to the calibration offset values found in Table~\ref{tbl:pressure_profile_results}, and
%finding those clusters that show a preference in $\Delta k$ towards a more spherical center. Recalling that $k$ 
%has a prior on it of $12$\%, we can calculate significances shown in Table~\ref{tbl:accept_gnfw}.

%%%%%%%%%%%%%%%%%%%%%%%%%%%%%%%%%%%%%%%%%%%%%%%%%%%%%%%%%%%%%%

\subsection{Temperature profiles}
\label{sec:temp_profiles}

%%%%%%%%%%%%%%%%%%%%%%%%%%%%%%%%%%%%%%%%%%%%%%%%%%%%%%%%%%%%%%

If we assume a given geometry (known ellipticity), then instead of solving for the ellipticity, we can
derive a temperature profile, making use of the direct pressure constraints from SZ observations and the
electron density constraints from X-ray observations. That is, we calculate
\begin{equation}
  T_{SZ,X} = \frac{P_{SZ}}{n_{e,X}},
  \label{eqn:telec}
\end{equation}
where $P_{SZ}$ is the pressure derived from pressure profile fits to the SZ data (Section~\ref{sec:pp_constraints}) and
$n_{e,X}$ is the deprojected electron density derived from X-ray data by the ACCEPT2 collaboration.
For each bin, we assign radial values as the arithmetic mean of its radial bounds.
%(i.e. $r = (r_i + r_{i+1})/2$), where $r_i$ are the bounds between bins.
Binned values of $P_{SZ}$ are then calculated from the fitted gNFW profile for each radial value 
for the corresponding bins used for $n_{e,X}$.

%may lead to more accurate than that found solely by X-rays, depending on
%the quality of both data sets. 
%X-rays do not constrain electron temperature as well as electron density because
%of the scaling in emission, and foremost because of the photon count requirement for X-ray spectra used to derive 
%a temperature from X-rays, which effectively provides lower resolution temperature data, especially at larger radii.
%Thus, combining the SZ derived pressure and X-ray derived electron density makes use of the strength of each data
%set.

We consider two models to describe temperature profiles in the literature: those in
\citet{vikhlinin2006}, denoted as V06 and \citet{bulbul2010}, denoted as B10. The first is given as:
\begin{equation}
  T_{e,V06} = t_{cool} \times \frac{T_0}{(r/r_t)^a [1 + (r/r_t)^b]^{c/b}} \text{, where}
  \label{eqn:tvik06}
\end{equation}
\begin{equation}
  t_{cool} = \frac{[(r / r_{cool})^{a_{cool}} + T_{min}/T_0]}{(r / r_{cool})^{a_{cool}} + 1}.
  \label{eqn:cc_taper}
\end{equation}
Here $r_{cool}$ is a fitted parameter, indicated the radial scale of the cool core, $r_t$ is a transitional radius,
which has been called a scaling radius in other profiles (e.g. NFW). $T_{min}$ is the minimum temperature observed
within the cool core. The remaining parameters, $a_{cool}$, $a$, $b$, $c$, and $T_0$ are all fit for. The first term
in the V06 model (Equation~\ref{eqn:tvik06}) is denoted as the cool core taper:
The second term in the V06 model (Equation~\ref{eqn:tvik06}) has the form of a gNFW profile, where typically Greek letters
are used for the gNFW pressure profiles (Section~\ref{sec:components}). The letter equivalents
are $a = \gamma$, $b = \alpha$, and $c = \beta - \gamma$. Given that for our pressure profile analysis, 
we fix $\alpha$ and $\beta$ from A10 values, here we fix $b$ and $c$ in fitting for the V06 temperature model.
We calculate $c$ and $r_T= R_{500}/C_{500}$ 
based on the best fit $\gamma$ and $C_{500}$ values from our joint fits (Section~\ref{sec:components}). 
Thus, in the temperature profile, we fit for $T_0$, $r_{cool}$, and $a_{cool}$, and $a$.

The other temperature model \citep{bulbul2010} is given as:
\begin{equation}
  T_{e,B10} = T_0 \left[ \frac{1}{(\beta - 2)} \frac{(1+ r/r_s)^{\beta-2} - 1}{r/r_s (1 + r/r_s)^{\beta-2}}\right] t_{cool}
  \label{eqn:tbul10}
\end{equation}
where $T_0$ and parameters in $t_{cool}$ are the only parameters specific to the temperature profile (independent of
pressure or density profile). $\beta$ is a power law term in a generalized NFW profile proposed in \citet{bulbul2010}:
\begin{equation}
  \rho_{tot}(r) = \frac{\rho_i}{(r/r_s)(1+r/r_s)^{\beta}}
  \label{eqn:bulgnfw}
\end{equation}
This formulation of density allows for an analytical formulation of $P(r)$ under the assumption of hydrostatic 
equilibrium. With the assumption of a polytropic equation of state ($P = k \rho_g^{n+1}$, where $k$ is simply a
constant, and $n$ is the polytropic index), the temperature profile (Equation~\ref{eqn:tbul10}) can be derived.
It is worth noting that Equation~\ref{eqn:tbul10} does not diverge at $\beta = 2$.
% and can be calculated from L'Hopital's rule. 
As with $r_t$ in the V06 model, we fix $r_s = R_{500}/C_{500}$ based on our fitted value of $C_{500}$ 
(Section~\ref{sec:pp_constraints}).

We use MPFIT \citep{markwardt2009} to solve for the free parameters in the two temperature profile parameterizations. For 
the V06 model, we fit for  $r_{cool}$, $a_{cool}$, $T_0$, and $a$, while for the B10 model we fit for $r_{cool}$, $a_{cool}$, 
$T_0$, and $\beta$. We define our bins (annuli) based on those provided by ACCEPT2. To match to SZ instrument resolutions, 
we require bins be at least 5\asecs in width within the central arcminute, and beyond the central arcminute we require the 
bins be at least 30\asec. To meet this requirement we take the weighted average of an integer number of bins.

Our SZ and X-ray derived temperature profiles (Figure~\ref{fig:tprofs_all}) reveal, on average, 
larger temperatures than the spectroscopically derived temperatures from ACCEPT2. Generally, both the V06 and B10
models are easily fit to our data (Table~\ref{tbl:temperature_profile_results}), as both often have $\tilde{\chi}^2 < 1$.
We note that the V06 model typically performs worse than B10 in our fits (by the goodness of fit parameter, $\chi^2$. 
This may be due to fixing $b$ and $c$ (especially $c$) as these are fixed power laws that incorporate the
behavior of $n_e$ with radius. However, we are still fitting the same number of parameters in the two models, thus
the B10 model offers more flexibility for the same number of fitted parameters. 

As an additional means of comparison, we fit a profile derived in \citetalias{vikhlinin2006} for the gas mass weighted
temperature:
\begin{equation}
  \frac{T(r)}{T_{mg}} = 1.35 \frac{(x_r/0.045)^{1.9} + 0.45}{(x_r/0.045)^{1.9} + 1} \frac{1}{(1+(x_r/0.6)^2)^{0.45}},
  \label{eqn:tmg},
\end{equation}
where $x_r = r / R_{500}$. Thus, since we take $R_{500}$ as known, the shape of the profile is fixed. The values
fit to the ACCEPT2 temperatures are reported in Table~\ref{tbl:cluster_properties}. We fit $T_{mg}$ to our
$T_{SZ,X}$ profiles, and compute the ratio  $\eta = T_{mg,SZ,X} / T_{mg,X}$ of the two fitted gas mass weighted 
temperatures. We find that $\langle \eta \rangle = 1.38 \pm 0.31$. 

%\afterpage{
%\clearpage
\thispagestyle{empty}
\begin{figure}
  \centering
  \begin{tabular}{ccc}
   \epsfig{file=figures/T_profs_and_models_a1835_coarse_accept2_20_Jul_2016.eps,width=0.33\linewidth,clip=}   &
   \epsfig{file=figures/T_profs_and_models_a611_coarse_accept2_20_Jul_2016.eps,width=0.33\linewidth,clip=}   &
   \epsfig{file=figures/T_profs_and_models_m1115_coarse_accept2_20_Jul_2016.eps,width=0.33\linewidth,clip=}   \\
   \epsfig{file=figures/T_profs_and_models_m0429_coarse_accept2_20_Jul_2016.eps,width=0.33\linewidth,clip=}   &
   \epsfig{file=figures/T_profs_and_models_m1206_coarse_accept2_20_Jul_2016.eps,width=0.33\linewidth,clip=}   &
   \epsfig{file=figures/T_profs_and_models_m0329_coarse_accept2_20_Jul_2016.eps,width=0.33\linewidth,clip=}   \\
   \epsfig{file=figures/T_profs_and_models_rxj1347_coarse_accept2_20_Jul_2016.eps,width=0.33\linewidth,clip=}   &
   \epsfig{file=figures/T_profs_and_models_m1311_coarse_accept2_20_Jul_2016.eps,width=0.33\linewidth,clip=}   &
   \epsfig{file=figures/T_profs_and_models_m1423_coarse_accept2_20_Jul_2016.eps,width=0.33\linewidth,clip=}   \\
%%% OLD files: temperature_profiles_and_models_a611_5_Apr_2016.eps
%  \end{tabular}
%  \caption{Temperature Profiles. The green triangles are derived as $T_e = P_{e,SZ} / n_{e,X}$, and the shaded green
%    indicates $1\sigma$ uncertainties.The blue plus signs are from ACCEPT2. The solid lines are our fitted
%    Vikhlinin temperature models and the dashed lines are our fitted Bulbul temperature models.}
%  \label{fig:tprofs_1}
%\end{figure}
%\begin{figure}
%  \centering
%  \begin{tabular}{ccc}
   \epsfig{file=figures/T_profs_and_models_m1149_coarse_accept2_20_Jul_2016.eps,width=0.33\linewidth,clip=}   &
   \epsfig{file=figures/T_profs_and_models_m0717_coarse_accept2_20_Jul_2016.eps,width=0.33\linewidth,clip=}   &
   \epsfig{file=figures/T_profs_and_models_m0647_coarse_accept2_20_Jul_2016.eps,width=0.33\linewidth,clip=}   \\
   \epsfig{file=figures/T_profs_and_models_m0744_coarse_accept2_20_Jul_2016.eps,width=0.33\linewidth,clip=}   &
   \epsfig{file=figures/T_profs_and_models_clj1226_coarse_accept2_20_Jul_2016.eps,width=0.33\linewidth,clip=}  &
  \end{tabular}
  \caption{Temperature Profiles. The green triangles are derived as $T_{SZ,X} = P_{SZ} / n_{e,X}$, and the shaded green
    indicates $1\sigma$ uncertainties, including calibration uncertainties. The blue plus signs are X-ray 
    spectroscopically derived temperatures from ACCEPT2. The solid lines are our fitted
    Vikhlinin temperature models and the dashed lines are our fitted Bulbul temperature models.}
  \label{fig:tprofs_all}
\end{figure}

%}
\begin{deluxetable}{l | l l l | l }
\tabletypesize{\footnotesize}
\tablecolumns{4}
\tablewidth{0pt} 
\tablecaption{Summary of Fitted Temperature Profiles \label{tbl:temperature_profile_results}}
\tablehead{
Cluster & $\chi_{V06}^2$ & $\chi_{B10}^2$ & DOF & $\eta$
}
\startdata
    Abell 1835  & 3.02 & 0.20 & 5 & 1.22 \\
    Abell 611   & 0.52 & 1.09 & 3 & 1.62 \\
    MACS 1115   & 0.17 & 0.14 & 3 & 1.29 \\
    MACS 0429   & 0.42 & 0.74 & 1 & 0.90 \\
    MACS 1206   & 0.21 & 0.28 & 3 & 1.27 \\
    MACS 0329   & 4.11 & 0.44 & 3 & 2.05 \\
    RXJ 1347    & 16.6 & 4.71 & 5 & 1.21 \\
    MACS 1311   & 1.00 & 0.98 & 2 & 1.48 \\
    MACS 1423   & 0.12 & 0.97 & 2 & 1.57 \\
    MACS 1149   & 9.25 & 2.32 & 3 & 1.55 \\
    MACS 0717   & 51.9 & 70.9 & 5 & 1.17 \\
    MACS 0647   & 0.33 & 0.18 & 2 & 1.69 \\
    MACS 0744   & 2.20 & 1.00 & 3 & 1.27 \\
    CLJ 1226    & 0.38 & 0.53 & 0 & 0.96 
%%%%%%%%%%%%%%%%%%%%%%%%%%%%%%%%%%%%%%%%%%%%%%%%%%%%%%%%
%%% July 15, 2016:
%    Abell 1835  & 5.04 & 0.44 & 5  \\
%    Abell 611   & 1.90 & 3.34 & 4  \\
%    MACS 1115   & 3.94 & 3.11 & 4  \\
%    MACS 0429   & 0.66 & 1.06 & 1  \\
%    MACS 1206   & 9.58 & 0.94 & 5  \\
%    MACS 0329   & 9.21 & 0.98 & 3  \\
%    RX J1347    & 35.5 & 9.28 & 5  \\
%    MACS 1311   & 1.36 & 1.10 & 2  \\
%    MACS 1423   & 36.6 & 5.16 & 4  \\
%    MACS 0717   & 77.2 & 142  & 5  \\
%    MACS 0647   & 0.99 & 0.41 & 2  \\
%    MACS 0744   & 5.04 & 2.68 & 3  \\
%    CL J1226    & 0.93 & 0.99 & 0  
%%%%%%%%%%%%%%%%%%%%%%%%%%%%%%%%%%%%%%%%%%%%%%%%%%%%%%%%
%    Abell 1835  & 1.03  & 0.57  & 3  \\
%    Abell 611   & 0.68  & 1.52  & 3  \\
%    MACS1115    & 2.00  & 3.30  & 4  \\
%    MACS0429    & 0.63  & 0.67  & 1   \\
%    MACS1206    & 3.30  & 1.34  & 4  \\
%    MACS0329    & 8.31  & 3.51  & 4  \\
%    RXJ1347     & 9.65  & 3.24  & 4  \\
%    MACS1311    & 1.17  & 1.14  & 2   \\
%    MACS1423    & 0.52  & 1.73  & 2   \\
%    MACS1149    & 9.76  & 1.40  & 4  \\
%    MACS0717    & 44.1  & 52.7  & 4  \\
%    MACS0647    & 0.29  & 1.37  & 2   \\
%    MACS0744    & 6.55  & 1.98  & 3   \\
%    CLJ1226     & 10.8  & 5.04  & 2
%%%%%%%%%%%%%%%%%%%%%%%%%%%%%%%%%%%%%%%%%%%%%%%%%%%%%%%%
%    Abell 1835  & 6.86  & 1.14  & 10  \\
%    Abell 611   & 9.60  & 2.40  & 10  \\
%    MACS1115    & 17.9  & 7.28  & 11  \\
%    MACS0429    & 1.69  & 1.13  & 3   \\
%    MACS1206    & 20.8  & 1.73  & 12  \\
%    MACS0329    & 8.31  & 5.03  & 11  \\
%    RXJ1347     & 13.5  & 4.55  & 11  \\
%    MACS1311    & 5.00  & 3.24  & 6   \\
%    MACS1423    & 0.73  & 3.09  & 6   \\
%    MACS1149    & 27.5  & 6.34  & 14  \\
%    MACS0717    & 41.4  & 54.7  & 13  \\
%    MACS0647    & 8.25  & 1.51  & 6   \\
%    MACS0744    & 6.55  & 2.74  & 8   \\
%    CLJ1226     & 14.4  & 12.5  & 5   
%%%%%%%%%%%%%%%%%%%%%%%%%%%%%%%%%%%%%%%%%%%%%%%%%%%%%%%%
%%% OLD FITS:
%    Abell 1835  & 12.2  & 6.45  & 25  \\
%    Abell 611   & 11.2  & 3.52  & 25  \\
%    MACS1115    & 23.1  & 12.1  & 28  \\
%    MACS0429    & 4.96  & 4.10  & 12  \\
%    MACS1206    & 21.6  & 2.62  & 30  \\
%    MACS0329    & 13.7  & 10.2  & 28  \\
%    RXJ1347     & 19.2  & 10.4  & 28  \\
%    MACS1311    & 8.98  & 7.01  & 17  \\
%    MACS1423    & 3.75  & 6.76  & 17  \\
%    MACS1149    & 29.0  & 7.67  & 34  \\
%    MACS0717    & 55.8  & 58.9  & 32  \\
%    MACS0647    & 9.25  & 3.73  & 18  \\
%    MACS0744    & 14.5  & 10.7  & 21  \\
%    CLJ1226     & 21.8  & 18.9  & 15
\enddata
\tablecomments{The reported $\chi^2$ values show that the temperature error bars are generally quite large.
   Covariance is not taken into account.}
\end{deluxetable}

%Finally, as we look forward to the future of galaxy cluster surveys \citep[e.g. eRosita, SPT3G, ACTpol][]{
%borm2014,benson2014,thornton2016}, we expect temperature derivations from SZ intensity and X-ray surface
%brightness (density) to be more common. The temperatures we derive in this manner here are dominated by 
%uncertainties in the SZ measurements, both statistical and systematic (calibration uncertainties), which are
%roughly comparable to each other. The SZ uncertainties are roughly a factor of 3 larger than the X-ray
%electron density uncertainties. This still results in temperature uncertainties $\sim 2$ times larger
%than the statistical spectroscopic X-ray temperature uncertainties. While systematic errors on spectroscopic
%temperatures are not well quantified, it is clear that they can be notable \citep[e.g.][]{donahue2014}.
%If the systematic uncertainties on spectroscopic X-ray temperatures were 20\%, then we would have
%$\sigma_{T_{SZ,X}} \sim \sigma_{T_X}$.

%%% Why are temperature profiles useful cosmologically? Or are they?
%%% Motivate resolved temperature profiles... 
%With the advent of MUSTANG-2 \citep{dicker2014} and NIKA-2 \citep{calvo2016} soon to come online,  

\begin{deluxetable}{l | c c c}
\tabletypesize{\footnotesize}
\tablecolumns{4}
\tablewidth{0pt} 
\tablecaption{Temperature Uncertainties \label{tbl:temperature_uncertainties}}
\tablehead{
Cluster & Exp. Time. & $\bar{\phi}$ & $\tilde{\phi}$ \\
}
\startdata
    Abell 1835  & 19.5  & 1.57 & 1.34 \\
    Abell 611   & 36.1  & 2.40 & 2.42 \\
    MACS 1115   & 55.5  & 1.87 & 1.54 \\
    MACS 0429   & 23.2  & 2.08 & 1.74 \\
    MACS 1206   & 23.5  & 1.20 & 1.11 \\
    MACS 0329   & 69.4  & 3.84 & 3.77 \\
    RXJ 1347    & 67.7  & 2.42 & 2.48 \\
    MACS 1311   & 78.1  & 2.65 & 2.13 \\
    MACS 1423   & 134.1 & 3.51 & 3.88 \\
    MACS 1149   & 38.5  & 1.60 & 1.65 \\
    MACS 0717   & 79.1  & 2.72 & 2.73 \\
    MACS 0647   & 39.3  & 2.19 & 2.09 \\
    MACS 0744   & 89.6  & 2.90 & 3.12 \\
    CLJ 1226    & 74.2  & 1.22 & 0.87 
\enddata
\tablecomments{The exposure time is reported in ks. We report the ratio of uncertainties in temperature determinations: 
  $\phi = \sigma_{T_{SZ,X}} / \sigma_{T_X}$, and  $\bar{\phi}$ is the geometric mean of $\phi$,
   and $\tilde{\phi}$ is the median of $\phi$.}
\end{deluxetable}

\subsection{Discussion: Differences Between SZ- and X-ray- Derived Quantities}

In regards to our geometry analysis, we find our approach (fitting the modelled ACCEPT2 pressure profile to the SZ data) 
to be less biased than fitting the SZ-derived profile to the ACCEPT2 data. However, as Bolocam probes larger radii than
available in the X-ray data, these fits should be accompanied with some caution. In particular, the extrapolation of the 
gNFW fits to the ACCEPT2 data to large radii may exaggerate the differences between the two datasets. With this cautionary
note, our discussion is centered more on the temperature profiles.

Given the degeneracy between the geometry of the cluster and temperature, as calculated here, we are not able to 
conclusively dissentangle the primary physical cause of our discrepancies in pressure profiles. 
Accounting for geometry as calculated in Section~\ref{sec:ellgeo}, we find that there are some clusters that still 
appear discrepant with ACCEPT2, particularly those whose pressure falls below ACCEPT2's pressure towards the center, 
but above ACCEPT2's pressure at larger radii. Dividing these pressure profiles by ACCEPT2's electron density, we see
the same trend in temperature profile: lower temperatures in the center, and higher temperatures are larger radii 
relative to ACCEPT2.

Clumping is expected to increase with radius, and thus may account for some of the discrepancy between our inferred 
temperature and the X-ray spectroscopically derived temperatures. \citet{battaglia2015} find that clumping
is more pronounced for more massive clusters. For the most massive bin of clusters considered, which is most applicable to our
sample, the density clumping ($C_{2,\rho} = \langle \rho^2 \rangle / \langle \rho \rangle ^2$) at $R_{500}$ is roughly 1.2. 
Such a clumping factor can account for a roughly $\sim5$\% discrepancy between the SZ/X-ray and X-ray spectroscopically derived 
temperatures. However, the discrepancies we find are well within $R_{500}$, and therefore we do not consider that clumping
is a contributing factor to temperature discrepancies.

Merger activity may also address some of the discrepancies which we find. Interestingly, in RXJ 1347 and MACS 0744, 
where shocks have been observed \citep{kitayama2004,mason2010,korngut2011}, and residual components fit 
(Section~\ref{sec:jointfitting}), our temperature profiles are in good agreement with the spectroscopic X-ray temperature 
profiles. 

The triple merging cluster MACS 0717 (Section~\ref{sec:results_m0717}) does not present a clear shock 
in SZ or X-ray, but it may be that the merger activity is primarily along the line of sight. The notable enhancement of 
SZ-to-X-ray spectroscopic temperature in the center is undoubtedly due to merger activity, and bears credence as
other studies have found hot (roughly 20 keV in \citet{sayers2013,adam2016b}, and 34 keV in \citet{mroczkowski2012}) gas
in the region about subcluster C, which would contribute to temperature enhancements at small radii.
We note that in addition to X-ray emission being biased to denser (often cooler) regions, \emph{Chandra} is not sensitive
to higher energy photons and therefore constraints on gas hotter than
$k_B T \gtrsim 10$ keV are generally poor.
Furthermore, subcluster C is shown to be moving towards us, inducing a kinetic SZ (kSZ), a Doppler shift of the CMB due to the
bulk peculiar velocity of the ICM relative to the CMB, signal which would augment
the signal that we interpret as tSZ signal. While these analyses were performed about subcluster C, and not azimuthally 
about the cluster centroid, the difference in the our derived temperatures and the spectroscopic X-ray temperatures 
highlights the difference between a mass-weighted temperature (SZ+X-ray) versus a density-squared-weighted temperature 
(spectroscopic X-ray). Interestingly, because subcluster B is moving away there will be a decrease in the signal that we 
interpret as tSZ signal, which in principle would counteract the augmentation of subcluster C modulo the subcluster gas 
masses and distance from our assumed centroid. 

While the kSZ is relevant for MACS 0717, it has not, to our knowledge, been detected in any other individual cluster.
With typical cluster peculiar velocities \citep{hernandez2010}, the kSZ will be be significantly less than the tSZ,
especially in massive clusters. However, within mergers \citep[e.g.][]{sarazin2002}, the velocities can be an order of
magnitude greater, making the kSZ relevant.

%Another cluster, MACS 1149, suspected of undergoing a merger (Section~\ref{sec:results_m1149}), 
%has a enhancement in SZ/X-ray temperature and then tends towards 
%the spectroscopic temperature, but it is far less pronounced than in MACS 0717 and it occurs at a larger radius. 

While the temperature values we derive in CLJ 1226 are not significantly different than those in ACCEPT2, the slope is
reversed. We believe that this difference in slope accounts for a non-trivial change in the fitted pressure profile to
ACCEPT2 (Section~\ref{sec:ellgeo}), which drives the corresponding SZ-fitted normalizations ($P_{SZ}$ and $P_B$) low.

Within our sample, MACS 0429 has the greatest discrepancy, especially in terms of the pressure profile shapes. While
this may be due, in part, to an increase in temperature with radius, we do not contend that this increase is as dramatic
as the shown in Figure~\ref{fig:tprofs_all}.
It is possible that our point source treatment (Section~\ref{sec:pp_error}) is responsible for some of the discrepancy. 
However, this should primarily affect the inner pressure profile, and Bolocam is constraining the pressure at moderate 
to large radii to be well above that found in ACCEPT2. 

Having offered potential explanations for the origin of discrepancies within individual clusters, it is interesting to
contrast these discrepancies with some overall trends where there appears to be agreement. In particular, we noted good 
overall agreement in ensemble constraints of the pressure profile between our work and \citetalias{baldi2014}. Our
methodology for calculating the ensemble pressure profiles does not give equal weighting to every cluster, where clusters
with both good SZ and X-ray data (the most weight), such as RXJ 1347, exhibit smaller discrepancies. Alternatively,
we can average axis ratios and find $\langle c \rangle = 1.38 \pm 0.58$, or average the gas mass weighted temperature 
ratios: $\langle \eta \rangle = 1.38 \pm 0.31$. While these are computed in an unweighted manner across clusters, their
derivation depends on the functions being fitted, and how the weight is distributed in the SZ and X-ray data. 

%%% My text:
%Overall, we do not find internal inconsistencies when comparing SZ and X-ray data. However, our results do seem at odds
%with \citet{rumsey2016}, who study 10 CLASH clusters with AMI and compare to \emph{Chandra} data. When they compare to
%ACCEPT temperatures, they find a trend that $T_X$, the X-ray (\emph{Chandra}) temperature is higher than $T_{SZ}$ (AMI).
%They compare $T_{SZ}$ to $T_X$ (1) determined within an annulus about $r = 700$ kpc and (2) calculated as a mean temperature 
%within $0.15 R_{500} < r < R_{500}$. Of their sample of 10, there are 7 clusters in common with our sample.
%%% JS text:
In contrast to our results, which indicate on-average higher values of $T_{SZ}$ than $T_{X}$, we note that \citet{rumsey2016}
find the opposite trend when comparing SZ data from the Arcminute MicroKelvin Imager (AMI) with \emph{Chandra} X-ray data for
a highly overlapping subset of CLASH clusters with our sample. However, \citet{rumsey2016} use a much different technique to
constrain $T_{SZ}$, based solely on the SZ data with strong priors on cluster parameters such as $f_{gas}$. In addition, the 
potential systematics in the 15 GHz interferometric SZ data used by \citet{rumsey2016} are largely distinct from those
related to our higher frequency bolometric SZ images. As a results, it is not possible to make a direct comparison of the
results to better ascertaint the cause of the discrepancy.

%While the ICM of our clusters may deviate from being smooth and spherical, the potential deviations discussed here do not,
%in our estimation, account for some of the larger discrepancies. Therefore, we find it implausible that the temperature is in 
%some cases twice that found by ACCEPT2, especially those inferred values of $k_B T_e \gtrsim 15$ keV.

% a $\sim20$\% increase of temperature higher relative to ACCEPT2 at $R_{500}$ would be consistent with simulations.

%the square root of pressure clumping ($C_{2,P}$),
%given by $C_{2,P} = \langle P_{th}^2 \rangle / \langle P_{th} \rangle ^2$, is roughly 1.2, and the square root of mass
%density clumping ($C_{2,\rho}$), given by $C_{2,\rho} = \langle \rho^2 \rangle / \langle \rho \rangle ^2$, is roughly 1.1.

%Some cases, especially
%MACS 0717 and MACS 1149, are likely due to their known merging status. Others, like MACS 1115, a cool core cluster
%with no documented merger activity, with higher pressure at large radii in the SZ relative to X-rays could likely 
%be explained by the geometry in the plane of the sky. MACS 1115 shows a northern elongation in the Bolocam map that 
%is not present in the X-ray surface brightness maps (Figure~\ref{fig:mustang_maps_sample}), and MACS 0429 shows a 
%northwest-southeast elongation that is not evident in the X-ray surface brightness maps. Thus, there is a geometric 
%discrepancy between the SZ and X-ray, especially at moderate to large radii. 

%%%%%%%%%%%%%%%%%%%%%%%%%%%%%%%%%%%%%%%%%%%%%%%%%%%%%%%%%%%%%%%%%%%%%%%%%%%%%%%%%%%%%%%%%%%%%%%%%%%%%%%%%%%%%%%%
%%%%%%%%%%%%%%%%%%%%%%%%                    CONCLUSIONS!!!                           %%%%%%%%%%%%%%%%%%%%%%%%%%%
%%%%%%%%%%%%%%%%%%%%%%%%%%%%%%%%%%%%%%%%%%%%%%%%%%%%%%%%%%%%%%%%%%%%%%%%%%%%%%%%%%%%%%%%%%%%%%%%%%%%%%%%%%%%%%%%

\section{Conclusions}
\label{sec:conclusions}

We developed an algorithm to jointly fit gNFW pressure profiles to clusters observed via the SZ
effect with MUSTANG and Bolocam. We apply this algorithm to 14 clusters and find the profiles are 
consistent with a universal pressure profile found in \citet{arnaud2010}. Specifically, the 
pressure profile is of the form:
\begin{equation*}
  \Tilde{P} = \frac{P_0}{(C_{500} X)^{\gamma} [1 + (C_{500} X)^{\alpha}]^{(\beta - \gamma)/\alpha}},
%  \label{eqn:norm_gnfw}
\end{equation*}
where we fixed $\alpha$ and $\beta$ to values found in \citet{arnaud2010}. A comparison to previous
determinations of pressure profiles is shown in Figure~\ref{fig:pp_sets}. Within the radii where we
have the greatest constraints ($0.03 R_{500} \lesssim r \lesssim R_{500}$), the pressure profile from this
work is comparable to the other pressure profiles. This is further evidenced in the parameters themselves,
as seen in Table~\ref{tbl:pressure_profile_results}, especially in comparison to \citetalias{arnaud2010} parameter values.

Despite agreement for the ensemble, we found discrepancies between the SZ and X-ray derived
pressure profiles for individual clusters and investigated the potential to explain these discrepancies
as being due to cluster geometry and ICM temperature (Section~\ref{sec:xray_comp}). We investigate cluster
geometry by looking at the ratio between spherically derived pressure profiles as fit to SZ and X-ray data 
and we find that the clusters have axis ratios, $c$, tabulated in 
Table~\ref{tbl:accept_gnfw}, which are generally greater than unity, implying that most of these clusters are 
elongated along the line of sight. We extend this analysis to estimate the relative cluster geometry
in the core (from MUSTANG), compared to the larger scale ICM (from Bolocam) and we find some 
hint that the cores tend to be more spherical than the ICM at larger radii. 
%\textcolor{red}{larger scale ICM}. However, our assumption is that the
%cluster geometry is one (or two) ellipsoids and this only explains a scalar offset in the pressure profiles.

In contrast, discrepancies due to electron temperature can account for differences with cluster-centric radii.
To explain the discrepancies we see, the ICM temperatures would need to be higher than derived by 
spectroscopic X-ray data by ACCEPT2. The difference in temperatures generally increases towards larger radii.

We conclude that cluster geometry and ICM temperature appear critical in accounting for the differences
between SZ and X-ray derived pressure profiles. However, we do not believe that either of these (with
strict ellipsoids) are sufficient to explain the differences observed. We argue that deviations from
the ellipsoidal geometry, such as the geometry structures seen in MACS 1115
(Figure~\ref{fig:mustang_maps_sample}), will also be important in explaining the discrepancies observed.

As SZ and X-ray surveys discover many thousands of new clusters over the next few years, it will be imperative
to jointly analyse clusters, especially at higher redshift, to determine ICM temperatures. This study shows
the feasibility, alongside the challenges, of determining temperature profiles by joining SZ and X-ray datasets.

Finally, as we look forward to the future of galaxy cluster surveys \citep[e.g. eRosita, SPT3G, ACTpol][]{
borm2014,benson2014,thornton2016}, we expect temperature derivations from SZ intensity and X-ray surface
brightness (density) to be more common. The temperatures we derive in this manner here are dominated by 
uncertainties in the SZ measurements, both statistical and systematic (calibration uncertainties), which are
roughly comparable to each other. The SZ uncertainties are roughly a factor of 3 larger than the X-ray
electron density uncertainties. This still results in temperature uncertainties $\sim 2$ times larger
than the statistical spectroscopic X-ray temperature uncertainties. While systematic errors on spectroscopic
temperatures are not well quantified, it is clear that they can be notable \citep[e.g.][]{donahue2014}.
If the systematic uncertainties on spectroscopic X-ray temperatures were 20\%, then we would have
$\sigma_{T_{SZ,X}} \sim \sigma_{T_X}$.

%On the high resolution
%end, mapping the ICM temperatures can reveal (especially in its application to entropy) the dynamical history of
%clusters. With more sensitive SZ instruments \citep[e.g.][]{dicker2014,calvo2016} coming online soon, this is
%
%With MUSTANG-2 \citep{dicker2014} and NIKA-2 \citep{calvo2016} soon to come online, this endeavour
%will be commonplace.


%We tabulate two temperatures derived for the 
%clusters from ACCEPT \citep{cavagnolo2009} and \citet{morandi2015}: $T_X^1$ is calculated from a single spectrum over 
%$0.15 R_{500} < r < R_{500}$ for each cluster. $T_X^2$ is from \citet{morandi2015} and is calculated over 
%$0.15 R_{500} < r < 0.75 R_{500}$.  Additionally, we calculate $T_{mg}$, by fitting the ACCEPT 
%temperature profiles to the profile found in \citet{vikhlinin2006}. We then tabulate
%$T_{spec}$ as $T_{spec} = 1.11 \times T_{mg}$ given the ratio found in \citet{vikhlinin2006} between $T_{spec}$
%and $T_{mg}$. We may then compare either $T_x^1$ and $T_x^2$ to $T_{spec}$. Thus, we note that generally 
%$T_{spec} < T_{X}^1 \sim T_{X}^2$, which may indicate the ACCEPT pressure profiles may be biased low, relative 
%to the SZ pressure profiles by the deprojected (and interpolated) ACCEPT temperature profiles.

%These investigations of cluster geometry and electron temperatures may account for the discrepancy between
%SZ and X-ray pressure profiles. However, these investigations have been simplistic and are indicative of
%the global geometry and properties of the cluster (at all radii), with the exception that we have also found
%some indication of a difference in geometry between moderate-to-large radii and core. Yet, the discrepancy
%in pressure profiles is not simply a scalar offset, but that the shapes often differ, especially in that
%the SZ pressure profile is lower than X-ray in the cluster core, and SZ pressure is generally higher at
%large radii. Geometry in the plane of the sky may be able to account for this offset, as evidenced by
%MACS 1115, where an extended decrement to the north of the cluster center is seen by Bolocam, and drives the
%radial pressure up, whereas this extension does not appear in the X-ray surface brightness image.

%Finally, many of these clusters with discrepancies in pressure profiles have weak detections with 
%MUSTANG, and would benefit from additional high resolution observations. While this may not resolve the
%pressure difference at large radii, it could bring the pressure profile shape into greater alignment. 

\section*{Acknowledgements}

While at NRAO and UVA, support for CR was provided through the Grote Reber Fellowship at NRAO. 
Support for CR, PK, and AY was 
provided by the Sudent Observing Support (SOS) program. Support for TM is provided by the National Research 
Council Research Associateship Award at the U.S.\ Naval Research Laboratory. Basic research in radio 
astronomy at NRL is supported by 6.1 Base funding. JS was partially supported by a
Norris Foundation CCAT Postdoctoral Fellowship and by NSF/AST-1313447.

The National Radio Astronomy Observatory is a facility of the National Science Foundation which is operated
under cooperative agreement with Associated Universities, Inc. The GBT observations used in this paper were
taken under NRAO proposal IDs GBT/08A-056, GBT/09A-052, GBT/09C-020, GBT/09C-035, GBT/09C-059, GBT/10A-056, 
GBT/10C-017, GBT/10C-026, GBT/10C-031, GBT/10C-042, GBT/11A-001, and GBT/11B-009 and VLA/12A-340.
We  thank the GBT operators Dave Curry, Greg Monk, Dave Rose, Barry Sharp, and Donna Stricklin for their
assistance. 

The Bolocam observations presented here were obtained form the Caltech Submillimeter Observatory, which,
when the data used in this analysis were taken, was operated by the California Institute of Technology under
cooperative agreement with the National Science Foundation. Bolocam was constructed and commissioned using funds
from NSF/AST-9618798, NSF/AST-0098737, NSF/AST-9980846, NSF/AST-0229008, and NSF/AST-0206158. Bolocam observations
were partially supported by the Gordon and Betty Moore Foundation, the Jet Propulsion Laboratory Research and
Technology Development Program, as well as the National Science Council of Taiwan grant NSC100-2112-M-001-008-MY3.

Access to ACCEPT2 data was possible due to the gracious assistance of Rachael Salmon.


%%%%%%%%%%%%%%%%%%%%%%%%%%%%%%%%%%%%%%%%%%%%%%%%%%%%%%%%%%%%%%%%%%%%%%%%%%%%%%%%%%%%%%%%%%%%%%%%%%%%%%%%%%%%%%%%
%%%%%%%%%%%%%%%%%%%%%%%%                       APPENDIX                              %%%%%%%%%%%%%%%%%%%%%%%%%%%
%%%%%%%%%%%%%%%%%%%%%%%%%%%%%%%%%%%%%%%%%%%%%%%%%%%%%%%%%%%%%%%%%%%%%%%%%%%%%%%%%%%%%%%%%%%%%%%%%%%%%%%%%%%%%%%%

\appendix

%%%%%%%%%%%%%%%%%%%%%%%%%%%%%%%%%%%%%%%%%%%%%%%%%%%%%%%%%%%%%%%%%%%%%%%%%%%%%%%
\section{Notes on Individual Clusters}
\label{sec:ind_notes}
%%%%%%%%%%%%%%%%%%%%%%%%%%%%%%%%%%%%%%%%%%%%%%%%%%%%%%%%%%%%%%%%%%%%%%%%%%%%%%%

%%%%%%%%%%%%%%%%%%%%%%%%%%%%%%%%%%%%%%%%%%%%%%%%%%%%%%%%%%%%%%

\subsection{Abell 1835 (z=0.25)}
\label{sec:results_a1835}

%%%%%%%%%%%%%%%%%%%%%%%%%%%%%%%%%%%%%%%%%%%%%%%%%%%%%%%%%%%%%%

Abell 1835 is a well studied massive cool core cluster. The cool core was noted to have substructure in the central
10\asecs by \citet{schmidt2001}, and identified as being due the central AGN by \citet{mcnamara2006}. Abell 1835 has also
been extensively studied via the SZ effect \citep{reese2002,benson2004,bonamente2006,sayers2011,mauskopf2012}. The models adopted
were either beta models or generalized beta models, and tend to suggest a shallow slope for the pressure interior
to 10\asec. Previous analysis of Abell 1835 with MUSTANG data \citep{korngut2011} detected the SZ effect decrement, but not
at high significance, which is consistent with a featureless, smooth, broad signal. Our updated MUSTANG reduction
of Abell 1835, shown in Figure \ref{fig:mustang_maps_sample}, has the same features as in \citet{korngut2011}.

%%%%%%%%%%%%%%%%%%%%%%%%%%%%%%%%%%%%%%%%%%%%%%%%%%%%%%%%%%%%%%

\subsection{Abell 611 (z=0.29)}
\label{sec:results_a611}

%%%%%%%%%%%%%%%%%%%%%%%%%%%%%%%%%%%%%%%%%%%%%%%%%%%%%%%%%%%%%%

The MUSTANG map (Figure~\ref{fig:mustang_maps_sample}) shows an enhancement
south of the X-ray centroid, and the Bolocam map shows elongation towards the south-southwest. Weak lensing maps
are suggestive of a southwest-northeast elongation \citep{newman2009, zitrin2015}.
Using the density of galaxies, \citet{lemze2013} find a core and a halo which align with the elongation seen 
in the SZ. We note that AMI \citep{hurley-walker2012} and \citet{defilippis2005} also see such an elongation. 
\citet{defilippis2005} also calculate a line of sight elongation $c = 1.05 \pm 0.37$.
cluster in their sample and that the X-ray data presented from \citet{laroque2006} is very circular and uniform.
Despite being relaxed, Abell 611 is not listed as a cool core cluster (nor disturbed) \citep{sayers2013}.
%Recent work by Siegal et al., in prep.,
%\citet{siegel2016} 
%finds that both SZ (Bolocam) and X-ray (\emph{Chandra}) data are well described
%by a spherical ICM supported entirely by thermal pressure.

In an analysis of the dark matter distribution, \citet{newman2009} find that the core (logarithmic) slope of the
cluster is shallower than an NFW model, with $\beta_{DM} = 0.3$, where the dark matter distribution has been characterized
by yet another generalization of the NFW profile:
\begin{equation}
  \rho(r) = \frac{\rho_0}{(r/r_s)^{\beta_{tot}}(1 + r/r_s)^{3-\beta_{tot}}}
\end{equation}
They find the distribution of dark matter within Abell 611 to be inconsistent with a single NFW model. 

%%%%%%%%%%%%%%%%%%%%%%%%%%%%%%%%%%%%%%%%%%%%%%%%%%%%%%%%%%%%%%

\subsection{MACS 1115 (z=0.36)}
\label{sec:results_m1115}

%%%%%%%%%%%%%%%%%%%%%%%%%%%%%%%%%%%%%%%%%%%%%%%%%%%%%%%%%%%%%%

MACS 1115 is listed as a cool core cluster \citep{sayers2013}. It is among seven CLASH clusters that show
unambiguous ultraviolet (UV) excesses attributed to unabsorbed star formation rates of 5-80 $M_{\odot} $yr$^{-1}$
\citep{donahue2015}. MUSTANG detects a point source in MACS 1115, which is coincident with its BCG. 
%The NVSS, at 1.4 GHz, finds the flux of the point source to be $16.2$ mJy.
MACS 1115 is fit by a fairly steep inner pressure profile slope to the SZ data (Table~\ref{tbl:pressure_profile_results}).
Adopting the Bolocam centroid, the inner pressure profile slope is notably reduced, yet the goodness of fit is
not significantly changed. In particular, the Bolocam image shows a north-south elongation (particularly to the
north of the centroids). In contrast, weak and strong lensing \citep{zitrin2015} show a more southeast-northwest
elongation.

%%%%%%%%%%%%%%%%%%%%%%%%%%%%%%%%%%%%%%%%%%%%%%%%%%%%%%%%%%%%%%

\subsection{MACS 0429 (z=0.40)}
\label{sec:results_m0429}

%%%%%%%%%%%%%%%%%%%%%%%%%%%%%%%%%%%%%%%%%%%%%%%%%%%%%%%%%%%%%%

MACS 0429 has been well studied in the X-ray \citep{schmidt2007,comerford2007,maughan2008,allen2008,mann2012}
MACS 0429 is identified as a cool core cluster \citep[cf.][]{mann2012,sayers2013}. The bright point source in 
the MUSTANG image is the cluster BCG, which is noted as having an excesses UV emission \citep{donahue2015}.
Of the point sources observed by MUSTANG, this has the shallowest spectral index between 90 GHz and 140 GHz
of $\alpha_{\nu} = 0.55$.
%At 90 GHz, we find the flux density as $7.67 \pm 0.84$ mJy. The point source subtracted from the Bolocam data
%is a $6.0 \pm 1.8$ mJy source at 140 GHz. At 1.4 GHz, NVSS finds the point source to have a flux density of
%$138.8 \pm 4.2$ mJy \citep{condon1998}. 

Despite MACS 0429's stature as a cool core cluster, its pressure profile
(Table~\ref{tbl:pressure_profile_results}) is surprisingly shallow in the core, and shows elevated pressure relative to
X-ray derived pressure at moderate radii. The offset between the Bolocam centroid \citep{sayers2013} and ACCEPT
\citep{cavagnolo2009} centroid is 100 kpc, which is notably larger than the X-ray-optical separations of the cluster
peaks and centroids reported in \citet{mann2012} of 12.8 and 19.5 kpc respectively. %\citet{siegel2016}
Siegel et al.\ (in prep.)\
report an excess in SZ pressure (Bolocam) relative to X-ray (\emph{Chandra}) pressure at moderate to large radii.

%\afterpage{
%\clearpage
%\thispagestyle{empty}
%\begin{figure}
%  \centering
%  \includegraphics[width=0.85\textwidth]{analysis_figures/MACS0429_flux_figure_with_centroid_ptsub_mnsub_9_Jul_2015}
%  \includegraphics[width=0.85\textwidth]{analysis_figures/MBO_Contours_m0429_lens_22_Jan_2015.eps}
%  \caption{MACS 0429}
%  \label{fig:macs_0429params}
%\end{figure}
%\clearpage
%}

%%%%%%%%%%%%%%%%%%%%%%%%%%%%%%%%%%%%%%%%%%%%%%%%%%%%%%%%%%%%%%

\subsection{MACS 1206 (z=0.44)}
\label{sec:results_m1206}

%%%%%%%%%%%%%%%%%%%%%%%%%%%%%%%%%%%%%%%%%%%%%%%%%%%%%%%%%%%%%%

MACS 1206 has been observed extensively \citep[e.g.][]{ebeling2001,ebeling2009,gilmour2009,umetsu2012,
zitrin2012a,biviano2013,sayers2013}. It is not categorized as a cool core or a disturbed cluster
\citep{sayers2013}. Using weak lensing data from Subaru, \citet{umetsu2012} find that the major-minor 
axis ratio of projected mass is $\gtrsim 1.7$ at $1\sigma$. They infer that this high ellipticity and 
alignment with the BCG, optical, X-ray, and LSS shapes are suggestive that the major axis is aligned 
close to the plane of the sky. In \citet{young2014}, substructure is identified that corresponds to an 
optically-identified subcluster, which may either be a merging subcluster, or a foreground cluster. 
In this analysis, the SZ signal observed by MUSTANG is well modelled by a residual component (coincident 
with the subcluster) and a spherical bulk ICM component. We note that the Bolocam contours of MACS 1206
do not exhibit much ellipticity. We do find that MACS 1206 has a major-minor axis 
ratio of $1.24 \pm 0.29$ (Table~\ref{tbl:accept_gnfw}), where the major axis is along the line of sight.

%The point source was found to have a flux density of $0.77 \pm 0.06$ mJy with the best fit model in
%\citet{young2014}. In this analysis, we find it to have a flux density of $0.75 \pm 0.08$ mJy. A proposal 
%has been accepted for \emph{XMM-Newton} observations of this substructure (PI: Sarazin).

%\afterpage{
%\clearpage
%\thispagestyle{empty}
%\begin{figure}
%  \centering
%  \includegraphics[width=0.85\textwidth]{analysis_figures/cres/JF_Conf_Intervals_2params_both_default_speedy_9_Feb_2015_m1206.eps}
%  \includegraphics[width=0.85\textwidth]{analysis_figures/cres/PPP_arnaud_v3_log-log_30_Mar_2015_m1206.eps}
%  \caption{MACS 1206}
%  \label{fig:macs_1206params}
%\end{figure}
%\clearpage
%}

%%%%%%%%%%%%%%%%%%%%%%%%%%%%%%%%%%%%%%%%%%%%%%%%%%%%%%%%%%%%%%

\subsection{MACS 0329 (z=0.45)}
\label{sec:results_m0329}

%%%%%%%%%%%%%%%%%%%%%%%%%%%%%%%%%%%%%%%%%%%%%%%%%%%%%%%%%%%%%%

MACS 0329 has the distinction of being listed as both a cool core and disturbed cluster. Although it has
been classified as relaxed \citep{schmidt2007}, subtructure has been noted \citep{maughan2008}, and it earns
its cool core and disturbed classifications based on central weighting of X-ray luminsoity and comparing
centroid offsets between optical and X-ray data \citep{sayers2013}. The elongation of the weak lensing and
strong lensing are towards the northwest and southeast of the centroid.

MACS 0329 has two systems with multiple images: one at $z = 6.18$ and the other at $z = 2.17$. The Einstein
radii for these two systems are $r_E = 34$\asecs and $r_E = 28$\asec, respectively \citep{zitrin2012b}, which is
noted as being typical for relaxed, well-concentrated lensing clusters. 


%\afterpage{
%\clearpage
%\thispagestyle{empty}
%\begin{figure}
%  \centering
%  \includegraphics[width=0.85\textwidth]{analysis_figures/cres/JF_Conf_Intervals_2params_both_default_speedy_9_Feb_2015_m0329.eps}
%  \includegraphics[width=0.85\textwidth]{analysis_figures/cres/PPP_arnaud_v3_log-log_23_Feb_2015_m0329.eps}
%  \caption{MACS 0329}
%  \label{fig:macs_0329params}
%\end{figure}
%\clearpage
%}

%%%%%%%%%%%%%%%%%%%%%%%%%%%%%%%%%%%%%%%%%%%%%%%%%%%%%%%%%%%%%%

\subsection{RXJ1347 (z=0.45)}
\label{sec:results_rxj1347}

%%%%%%%%%%%%%%%%%%%%%%%%%%%%%%%%%%%%%%%%%%%%%%%%%%%%%%%%%%%%%%

RXJ1347 is one of the most luminous X-ray clusters, and has been well studied in radio, SZ, lensing, optical
spectroscopy, and X-rays \citep[e.g.][]{schindler1995,allen2002, pointecouteau1999,komatsu2001,kitayama2004,
gitti2007b,ota2008,bradac2008,miranda2008}. X-ray contours have long suggested RXJ1347 is a relaxed system
\citep[e.g.][]{schindler1997}, and it is classified as a cool core cluster \citep[e.g.][]{mann2012,sayers2013}. 

Indeed, the first sub-arcminute SZ observations \citep{komatsu2001,kitayama2004} saw an enhancement to
the southeast of the cluster X-ray peak, which was suggested as being due to shock heating. This enhancement
was confirmed by MUSTANG \citep{mason2010}. Further measurements were made with CARMA \citep{plagge2013},
which find the 9\% of the thermal energy in the cluster is in sub-arcminute substructure. Most recently,
\citet{kitayama2016} has observed this cluster with ALMA to a resolution of 5\asec.
At low radio frequencies \citep[][237 MHz and 614 MHz]{ferrari2011},
\citep[][1.4 GHz]{gitti2007a} find evidence for a radio mini-halo in the core of RXJ1347. The cosmic ray electrons
are thought to be reaccelerated because of the shock and sloshing in the cluster \citep{ferrari2011}.

We observe a point source (coincident with the BCG) with flux density of $7.40 \pm 0.58$ mJy. Previous analysis of 
the MUSTANG data found the point source flux density as 5 mJy \citep{mason2010}. The difference in the flux 
densities is likely accounted by (1) the different modeling of point sources; primarily that we filter the double 
Gaussian, (2) we simultaneously fit the components, and (3) we assume a steeper profile in the core than the beta 
model assumed in \citet{mason2010}. Lower frequency radio observations found the flux density of the source to be 
$10.81 \pm 0.19$ mJy at 28.5 GHz \citep{reese2002}, and $47.6 \pm 1.9$ mJy at 1.4 GHz \citep{condon1998}. The BCG 
is observed to have a UV excess\citep{donahue2015}. 

Despite the classification of being a cool core cluster, it is also observed that there are hot regions, intially
constrained as $k_BT > 10$ keV \citep[e.g.][]{allen2002,bradac2008}, and more recently constrained to even hotter 
temperatures \citep[$k_BT > 20$ keV][]{johnson2012}, indicative of an unrelaxed cluster. \citet{johnson2012} also 
interpret the two cold fronts as being due to sloshing, where a subscluster has returned for a second passage.

Several previous studies have found similar evidence for compression along the line of sight in this cluster
\citep[e.g.][and references therein]{plagge2013}. However, the compression we find in this study is less 
severe as in \citet{plagge2013}.
%\afterpage{
%\clearpage
%\thispagestyle{empty}
%\begin{figure}
%  \centering
%  \includegraphics[width=0.85\textwidth]{analysis_figures/cres/JF_Conf_Intervals_2params_both_default_speedy_9_Feb_2015_rxj1347.eps}
%  \includegraphics[width=0.85\textwidth]{analysis_figures/cres/PPP_arnaud_v3_log-log_3_Feb_2015_rxj1347.eps}
%  \caption{RXJ1347}
%  \label{fig:rxj1347params}
%\end{figure}
%\clearpage
%}

%%%%%%%%%%%%%%%%%%%%%%%%%%%%%%%%%%%%%%%%%%%%%%%%%%%%%%%%%%%%%%

\subsection{MACS 1311 (z=0.49)}
\label{sec:results_m1311}

%%%%%%%%%%%%%%%%%%%%%%%%%%%%%%%%%%%%%%%%%%%%%%%%%%%%%%%%%%%%%%


MACS 1311 is listed as a cool core cluster \citep[e.g.][]{sayers2013}, and appears to have quite circular
contours in the X-ray and lensing images, yet has evidence for some disturbance, given its classification
in \citet{mann2012}. However, the SZ contours from Bolocam show some enhancement  to the west, and has
a notable centroid shift ($27.7$\asec, 167 kpc) westward from the X-ray centroid. When fitting pressure profiles
to this cluster, it appears that the enhanced SZ pressure at moderate radii ($r \sim 100$\asec) is due
to this enhancement, especially when noting that we use the X-ray centroid. Adopting the Bolocam centroid
does not change the pressure profile much, and we still observe a pressure enhancement at moderate radii.
In contrast, in their analysis, %\citet{siegel2016}
Siegel et al.\ (in prep.)\ find that X-ray (\emph{Chandra}) and SZ (Bolocam) data 
are in good agreement with a spherical ICM model which is supported primarily with thermal pressure.

%\afterpage{
%\clearpage
%\thispagestyle{empty}
%\begin{figure}
%  \centering
%  \includegraphics[width=0.85\textwidth]{analysis_figures/cres/JF_Conf_Intervals_2params_both_default_speedy_3_May_2015_m1311.eps}
%  \includegraphics[width=0.85\textwidth]{analysis_figures/cres/PPP_arnaud_v3_log-log_26_Apr_2015_m1311.eps}
%  \caption{MACS 1311}
%  \label{fig:macs_1311params}
%\end{figure}
%\clearpage
%}

%%%%%%%%%%%%%%%%%%%%%%%%%%%%%%%%%%%%%%%%%%%%%%%%%%%%%%%%%%%%%%

\subsection{MACS 1423 (z=0.54)}
\label{sec:results_m1423}

%%%%%%%%%%%%%%%%%%%%%%%%%%%%%%%%%%%%%%%%%%%%%%%%%%%%%%%%%%%%%%

MACS 1423 is a cool core cluster \citep{mann2012,sayers2013}. While the Bolocam contours are quite concentric,
and suggestive of a relaxed cluster, the centroid is still offset from the X-ray peak by an appreciable angle 
($19.8$\asec, 126 kpc). Similar to MACS 1311, the pressure is slightly less than the ACCEPT2 X-ray derived pressure in the
core, and slightly greater at moderate radii. While this is expected for a centroid offset, we find that adopting
the Bolocam centroid again yields no substantial difference in the SZ pressure profile. Both our analysis and
that of %\citet{siegel2016} 
Siegel et al.\ (in prep.)\ find good agreement between SZ and X-ray pressure profiles. We observe a point source 
(the cluster BCG) with flux density of $1.36 \pm 0.13$ mJy, which is also observed to have a UV excess 
\citep{donahue2015}. 

%\afterpage{
%\clearpage
%\thispagestyle{empty}
%\begin{figure}
%  \centering
%  \includegraphics[width=0.85\textwidth]{analysis_figures/cres/JF_Conf_Intervals_2params_both_default_speedy_9_Feb_2015_m1423.eps}
%  \includegraphics[width=0.85\textwidth]{analysis_figures/cres/PPP_arnaud_v3_log-log_24_Feb_2015_m1423.eps}
%  \caption{MACS 1423}
%  \label{fig:macs_1423params}
%\end{figure}
%\clearpage
%}

%%%%%%%%%%%%%%%%%%%%%%%%%%%%%%%%%%%%%%%%%%%%%%%%%%%%%%%%%%%%%%

\subsection{MACS 1149 (z=0.54)}
\label{sec:results_m1149}

%%%%%%%%%%%%%%%%%%%%%%%%%%%%%%%%%%%%%%%%%%%%%%%%%%%%%%%%%%%%%%

MACS 1149 is classified as a disturbed cluster \citep[e.g.][]{mann2012,sayers2013}, and lensing studies have found
that a single DM halo does not describe the cluster well, but rather at least four large-scale DM hales are used to
describe the cluster \citep{smith2009}. A large radial velocity dispersion \citep[1800 km s$^{-1}$][]{ebeling2007} is 
observed, indicative of merger activity along the line of sight. X-ray, SZ, and lensing (particularly 
strong lensing) all show elongation in the northwest-southeast direction. We see a $3\sigma$ feature to the east of
the centroids, but it is not clear that this is associated with any particular feature.

The SZ derived pressure profile roughly matches the shape of the X-ray derived pressure profile, 
with the SZ pressure consistently greater than the X-ray pressure.We calculate
that the axis along the line of sight is $1.54 \pm 0.37$ (Section~\ref{sec:ellgeo}) times greater than the axes in the plane
of the sky. Although we do not find previous analysis of the elongation in the plane of the sky, we would certainly
expect this given (1) the inferred merger activity along the line of sight, and (2) the lensing strength of the cluster.

%\afterpage{
%\clearpage
%\thispagestyle{empty}
%\begin{figure}
%  \centering
%  \includegraphics[width=0.85\textwidth]{analysis_figures/cres/JF_Conf_Intervals_2params_both_default_speedy_9_Feb_2015_m1149.eps}
%  \includegraphics[width=0.85\textwidth]{analysis_figures/cres/PPP_arnaud_v3_log-log_22_Jan_2015_m1149.eps}
%  \caption{MACS 1149}
%  \label{fig:macs_1149params}
%\end{figure}
%\clearpage
%}

%%%%%%%%%%%%%%%%%%%%%%%%%%%%%%%%%%%%%%%%%%%%%%%%%%%%%%%%%%%%%%

\subsection{MACS 0717 (z=0.55)}
\label{sec:results_m0717}

%%%%%%%%%%%%%%%%%%%%%%%%%%%%%%%%%%%%%%%%%%%%%%%%%%%%%%%%%%%%%%

%\begin{figure}
%  \centering
%  \includegraphics[width=0.85\textwidth]{analysis_figures/M0717_mroczkowski_fig1.eps}
%  \caption{From \citet{mroczkowski2012}.}
%  \label{fig:m0717_mroczkowski}
%\end{figure}

Despite MACS 1149's impressive merging activity, MACS 0717 is thought to be the most disturbed massive cluster at $z> 0.5$
\citep{ebeling2007}, which appears to be accreting matter along a 6-Mpc-long filament \citep{ebeling2004}, and has the
largest known Einstein radius \citep[$\theta_e \sim 55$\asec;][]{zitrin2009}. Four distinct components are identified
from X-ray and optical analyses \citep{ma2009}, and the lensing analyses \citep{zitrin2009,limousin2012} find agreement
in the location of these four mass peaks with those from the X-ray and optical. 

%%% Rework (subclusters, but without figure...?)
%The four components are labelled in Figure~\ref{fig:m0717_mroczkowski}. 
There are four identified subclusters \citep[labeled A through D][]{ma2009}. They find that subcluster C is the
most massive component, while subcluster A is the least massive, and subclusters B and D are likely remnant cores. The
velocities of the components from spectroscopy are found to be $(v_A, v_B, v_C, v_D) = (+278_{-339}^{+295},+3238_{-242}^{+252},
-733_{-478}^{+486},+831_{-800}^{+843})$ km s$^{-1}$ \citep{ma2009}. The first indication of detection of the kSZ signal
towards these subclusters was presented in \citet{mroczkowski2012}, with a subsequent paper from \citet{sayers2013}
having the first significant detection and derived cluster velocities. Most recently, \citet{adam2016b} has mapped the 
kSZ signal and derived model-dependent subcluster velocities. 

MACS 0717 has also been observed at 610 MHz with the Giant Metrewave Radio Telescope (GMRT) which reveals both a radio
halo and a radio relic \citep{vanweeren2009}. This is interpreted as likely being due to a diffuse shock acceleration
(DSA).

We observe a foreground radio galaxy, well outside the cluster centered, which we model as a point source here, 
with flux density of $2.08 \pm 0.25$ mJy at 90 GHz. 
This was previously reported with an integrated flux density of $2.8 \pm 0.2$ mJy and an extended shape 
14.\asec4 $\times$ 16.\asec1 \citep{mroczkowski2012}. However, an improved beam modeling has allowed us to model the 
foreground galaxy given a known beam shape. It is also worth noting that the MUSTANG data itself has been processed 
slightly differently from that presented in \citet{mroczkowski2012}; in this work the map is produced with a common 
calculated as the mean across detectors, whereas in \citet{mroczkowski2012} the common mode was calculated as the 
median across detectors.

%\afterpage{
%\clearpage
%\thispagestyle{empty}
%\begin{figure}
%  \centering
%  \includegraphics[width=0.85\textwidth]{analysis_figures/cres/JF_Conf_Intervals_2params_both_default_speedy_9_Feb_2015_m0717.eps}
%  \includegraphics[width=0.85\textwidth]{analysis_figures/cres/PPP_arnaud_v3_log-log_12_Mar_2015_m0717.eps}
%  \caption{MACS 0717}
%  \label{fig:macs_0717params}
%\end{figure}
%\clearpage
%}

%%%%%%%%%%%%%%%%%%%%%%%%%%%%%%%%%%%%%%%%%%%%%%%%%%%%%%%%%%%%%%

\subsection{MACS 0647 (z=0.59)}
\label{sec:results_m0647}

%%%%%%%%%%%%%%%%%%%%%%%%%%%%%%%%%%%%%%%%%%%%%%%%%%%%%%%%%%%%%%

MACS 0647 is at $z = 0.591$ and is classified as neither a cool core nor a disturbed cluster \citep{sayers2013}. 
It was included in the CLASH sample due to its strong lensing properties \citep{postman2012}.
Gravitational lensing \citep{zitrin2011}, X-ray surface brightness \citep{mann2012}, 
and SZ effect (MUSTANG, see Figure \ref{fig:mustang_maps_sample}, and Bolocam) maps all
show elongation in an east-west direction. 
In the joint analysis presented here, we see that the spherical model provides an adequate fit to both datasets and we note 
that the spherical assumption allows for a easier interpretation of the mass profile of the cluster.

%%%%%%%%%%%%%%%%%%%%%%%%%%%%%%%%%%%%%%%%%%%%%%%%%%%%%%%%%%%%%%

\subsection{MACS 0744 (z=0.70)}
\label{sec:results_m0744}

%%%%%%%%%%%%%%%%%%%%%%%%%%%%%%%%%%%%%%%%%%%%%%%%%%%%%%%%%%%%%%

MACS 0744 is neither classified as a cool core cluster nor a disturbed cluster \citep{mann2012,sayers2013}, but qualifies
as a relaxed cluster \citep{mann2012}. There is a dense X-ray core, and a doubly peaked red sequence of galaxies as found
by \citet{kartaltepe2008}. The gas is also found to be rather hot: $k_B T = 17.9_{-3.4}^{+10.8}$ keV, as determined by combining
SZ and X-ray data \citep{laroque2003}. 

The data presented here is the same as in \citet{korngut2011}, but has been processed differently: again, the primary difference
is in the treatment of the common mode. Additionally, \citet{korngut2011} optimize over the low-pass filtering of the common mode
and do not implement a correction factor for the SNR map. The surface brightness significance of the shock feature is the same, 
but is perhaps less
bowed than the kidney bean shape seen previously.  The excess found in \citet{korngut2011}
marked the first clear detection of a shock in the SZ that had not been previously been known from X-ray observations. 
\citet{korngut2011}
reanalyze the X-ray data with the knowledge of the shocked region from MUSTANG, and calculate the Mach number of the shock
based on (1) the shock density jump, (2) stagnation condition between the pressures at the edge of the cold front and just
ahead of the shock, and (3) temperature jump across the shock, and find Mach numbers between 1.2 and 2.1, with a velocity of 
$1827_{-195}^{+267}$ km s$^{-1}$. The shocked region (region II in \citet{korngut2011}) is well modelled with $19.7$ keV gas.

%\afterpage{
%\clearpage
%\thispagestyle{empty}
%\begin{figure}
%  \centering
%  \includegraphics[width=0.85\textwidth]{analysis_figures/cres/JF_Conf_Intervals_2params_both_default_speedy_9_Feb_2015_m0744.eps}
%  \includegraphics[width=0.85\textwidth]{analysis_figures/cres/PPP_arnaud_v3_log-log_26_Feb_2015_m0744.eps}
%  \caption{MACS 0744}
%  \label{fig:macs_0744params}
%\end{figure}
%\clearpage
%}

%%%%%%%%%%%%%%%%%%%%%%%%%%%%%%%%%%%%%%%%%%%%%%%%%%%%%%%%%%%%%%

%%%%%%%%%%%%%%%%%%%%%%%%%%%%%%%%%%%%%%%%%%%%%%%%%%%%%%%%%%%%%%

\subsection{CLJ 1226 (z=0.89)}
\label{sec:results_clj1226}

%%%%%%%%%%%%%%%%%%%%%%%%%%%%%%%%%%%%%%%%%%%%%%%%%%%%%%%%%%%%%%

CLJ 1226 is a well studied high redshift cluster \citep[e.g.][]{mroczkowski2009,bulbul2010,adam2015}. 
\citet{adam2015} find a point source at RA 12:12:00.01 and Dec +33:32:42 with a flux density of 
$6.8 \pm 0.7 \text{ (stat.)} \pm 1.0 \text{ (cal.)}$ mJy at 260 GHz and $1.9 \pm 0.2 \text{ (stat.)}$ at 150 GHz. 
This is not the same point source seen in \citet{korngut2011}, which is reported as a point source
with $4.6\sigma$ significance in surface brightness, and can be fit in our current analysis as a point source 
with a flux density of $0.33 \pm 0.13$ mJy. A short VLA filler observation (VLA-12A-340, D-array, at 7 GHz) 
was performed to follow up this potential source. To a limit of $\sim 50 {\rm \mu Jy}$ nothing is seen, 
other than the clearly spatially distinct radio source associated with the BCG at the cluster center 
(1 mJy at 7 GHz and 3.2 mJy in NVSS). In contrast, the point source found in \citet{adam2015} is fit to our 
data with a flux density of $0.36 \pm 0.11$ mJy. Given the slight increase in significance of the point source
from \citet{adam2015}, we adopt that point source location for our pressure profile analysis of CLJ 1226.
 
%%% K09 flux is 0.34 +/- 0.13 mJy in our maps. Now I've written it in. Jan 2016.

In the previous analysis of the MUSTANG data, \citet{korngut2011} find a ridge of significant substructure after 
subtracting a bulk SZ profile (N07, fitted to SZA data). They find that this ridge, southwest of the cluster
center, alongside X-ray profiles, are consistent with a merger scenario. However, in this work, we do not find
any significant substructure after fitting a bulk component.
%\textcolor{red}{[I will do an analysis with the point source found in Korngut+2011. The question is if I already 
%have the point source modelled (and to find it).}

\section{Data Products}

We have made MUSTANG data products for the sample of clusters analysed in this paper available at: 
\protect{\url{https://safe.nrao.edu/wiki/bin/view/GB/Pennarray/MUSTANG_CLASH}}. Links to accompanying
Bolocam and ACCEPT data are available from this website as well. In particular, we have publicized the final
data maps, noise maps, and signal-to-noise (SNR) maps used in this analysis, as well as transfer functions
for individual clusters. Further documentation is available on the website.

%%%%%%%%%%%%%%%%%%%%%%%%%%%%%%%%%%%%%%%%%%%%%%%%%%%%%%%%%%%%%%%%%%%%%%%%%%%%%%%%%%%%%%%%%%%%%%%%%%%%%%%%%%%%%%%%
%%%%%%%%%%%%%%%%%%%%%%%%                    BIBLIOGRAPHY!!!                          %%%%%%%%%%%%%%%%%%%%%%%%%%%
%%%%%%%%%%%%%%%%%%%%%%%%%%%%%%%%%%%%%%%%%%%%%%%%%%%%%%%%%%%%%%%%%%%%%%%%%%%%%%%%%%%%%%%%%%%%%%%%%%%%%%%%%%%%%%%%


\bibliographystyle{apj}
\bibliography{mycluster}
\label{references}

\end{document}


