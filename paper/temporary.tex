%%%%%%%%%%%%%%%%%%%%%%%%%%%%%%%%%%%%%%%%%%%%%%%%%%%%%%%%%%%%%%%%%%%%%%%%%%%%%%%
\section{Notes on Individual Clusters}
\label{sec:pp_constraints}
%%%%%%%%%%%%%%%%%%%%%%%%%%%%%%%%%%%%%%%%%%%%%%%%%%%%%%%%%%%%%%%%%%%%%%%%%%%%%%%

%%%%%%%%%%%%%%%%%%%%%%%%%%%%%%%%%%%%%%%%%%%%%%%%%%%%%%%%%%%%%%

\subsection{Abell 1835 (z=0.25)}
\label{sec:results_a1835}

%%%%%%%%%%%%%%%%%%%%%%%%%%%%%%%%%%%%%%%%%%%%%%%%%%%%%%%%%%%%%%

Abell 1835 is a well studied massive cool core cluster. The cool core was noted to have substructure in the central
10\asecs by \citet{schmidt2001}, and identified as being due the central AGN by \citet{mcnamara2006}. Abell 1835 has also
been extensively studied via the SZ effect \citep{reese2002,benson2004,bonamente2006,sayers2011,mauskopf2012}. The models adopted
were either beta models or generalized beta models, and tend to suggest a shallow slope for the pressure interior
to 10\asec. Previous analysis of Abell 1835 with MUSTANG data \citep{korngut2011} detected the SZ effect decrement, but not
at high significance, which is consistent with a featureless, smooth, broad signal. Our updated MUSTANG reduction
of Abell 1835 is shown in Figure \ref{fig:mustang_maps_1_sample}, and has the same features as in \citet{korngut2011}.

%%%%%%%%%%%%%%%%%%%%%%%%%%%%%%%%%%%%%%%%%%%%%%%%%%%%%%%%%%%%%%

\subsection{Abell 611 (z=0.29)}
\label{sec:results_a611}

%%%%%%%%%%%%%%%%%%%%%%%%%%%%%%%%%%%%%%%%%%%%%%%%%%%%%%%%%%%%%%

Abell 611 is unique among our clusters for the severity of the discrepancy between our joint SZ fitted pressure
profile and that found in the X-rays. The MUSTANG map (Figure~\ref{fig:mustang_maps_1_sample}) shows an enhancement
south of the X-ray centroid, and the Bolocam map shows elongation towards the south-southwest. Weak lensing maps
are suggestive of a southwest-northeast elongation \citep{newman2009, zitrin2015}.
Using the density of galaxies, \citet{lemze2013} find a core and a halo which align with the elongation seen 
in the SZ (Figure~\ref{fig:a611_supp_figs}). We note that AMI
\citep{hurley-walker2012} also sees this elongation, while they also note that Abell 611 is the most relaxed
cluster in their sample and that the X-ray data presented from \citet{laroque2006} is very circular and uniform.
Despite being relaxed, Abell 611 is not listed as a cool core cluster (nor disturbed) \citep{sayers2013}.

In an analysis of the dark matter distribution, \citet{newman2009} find that the core (logarithmic) slope of the
cluster is shallower than an NFW model, with $\beta_{DM} = 0.3$, where the dark matter distribution has been characterized
by yet another generalization of the NFW profile:
\begin{equation}
  \rho(r) = \frac{\rho_0}{(r/r_s)^{\beta_{tot}}(1 + r/r_s)^{3-\beta_{tot}}}
\end{equation}
They find the distribution of dark matter within Abell 611 to be inconsistent with an NFW model. 

%%%%%%%%%%%%%%%%%%%%%%%%%%%%%%%%%%%%%%%%%%%%%%%%%%%%%%%%%%%%%%

\subsection{MACS 1115 (z=0.36)}
\label{sec:results_m1115}

%%%%%%%%%%%%%%%%%%%%%%%%%%%%%%%%%%%%%%%%%%%%%%%%%%%%%%%%%%%%%%

MACS 1115 is listed as a cool core cluster \citep{sayers2013}. It is among seven CLASH clusters that show
unambiguous ultraviolet (UV) excesses attributed to unabsorbed star formation rates of 5-80 $M_{\odot} $yr$^{-1}$
\citep{donahue2015}.
MACS 1115 has a visible point source in the MUSTANG map. The NVSS, at 1.4 GHz, finds the flux of the point source
to be $16.2$ mJy.
MACS 1115 is fit by a fairly steep inner pressure profile slope to the SZ data (Figure~\ref{fig:CI_all}).
Adopting the Bolocam centroid, the inner pressure profile slope is notably reduced, yet the goodness of fit is
not significantly changed. In particular, the Bolocam image shows a north-south elongation (particularly to the
north of the centroids). In contrast, weak and strong lensing \citep{zitrin2015} show a more southeast-northwest
elongation.

%%%%%%%%%%%%%%%%%%%%%%%%%%%%%%%%%%%%%%%%%%%%%%%%%%%%%%%%%%%%%%

\subsection{MACS 0429 (z=0.40)}
\label{sec:results_m0429}

%%%%%%%%%%%%%%%%%%%%%%%%%%%%%%%%%%%%%%%%%%%%%%%%%%%%%%%%%%%%%%

MACS 0429 has been well studied in the X-ray \citep{schmidt2007,comerford2007,maughan2008,allen2008,mann2012}
MACS 0429 is indentified as a cool core cluster \citep[cf.][]{mann2012,sayers2013} 
and has a very bright point source in the MUSTANG image.
At 90 GHz, we find the flux density as $7.67 \pm 0.84$ mJy. The point source subtracted from the Bolocam data
is a $6.0 \pm 1.8$ mJy source at 140 GHz. At 1.4 GHz, NVSS finds the point source to have a flux density of
$138.8 \pm 4.2$ mJy \citep{condon1998}. MACS 0429 is noted as having an excesses UV emission \citep{donahue2015}. 

Despite MACS 0429's stature as a cool core cluster, its pressure profile
(Figure~\ref{fig:CI_all}) is surprisingly shallow in the core, and shows elevated pressure relative to
X-ray derived pressure at moderate radii. The offset between the Bolocam centroid \citep{sayers2013} and ACCEPT
\citep{cavagnolo2009} centroid is 100 kpc, which is notably larger than the X-ray-optical separations of the cluster
peaks and centroids reported in \citet{mann2012} of 12.8 and 19.5 kpc respectively.


%\afterpage{
%\clearpage
%\thispagestyle{empty}
%\begin{figure}
%  \centering
%  \includegraphics[width=0.85\textwidth]{analysis_figures/MACS0429_flux_figure_with_centroid_ptsub_mnsub_9_Jul_2015}
%  \includegraphics[width=0.85\textwidth]{analysis_figures/MBO_Contours_m0429_lens_22_Jan_2015.eps}
%  \caption{MACS 0429}
%  \label{fig:macs_0429params}
%\end{figure}
%\clearpage
%}

%%%%%%%%%%%%%%%%%%%%%%%%%%%%%%%%%%%%%%%%%%%%%%%%%%%%%%%%%%%%%%

\subsection{MACS 1206 (z=0.44)}
\label{sec:results_m1206}

%%%%%%%%%%%%%%%%%%%%%%%%%%%%%%%%%%%%%%%%%%%%%%%%%%%%%%%%%%%%%%

MACS 1206 has been observed extensively \citep[e.g.][]{ebeling2001,ebeling2009,gilmour2009,umetsu2012,
zitrin2012a,biviano2013,sayers2013}. It is not categorized as a cool core or a disturbed cluster
\citep{sayers2013}. Using weak lensing data from Subaru, \citet{umetsu2012} 
find that it the major-minor axis ratio of projected mass is $\gtrsim 1.7$ at $1\sigma$. They infer that
this high ellipticity and alignment with the BCG, optical, X-ray, and LSS shapes are suggestive that
the major axis is aligned close to the plane of the sky. In contrast, this analysis finds that MACS 1206 
has a major-minor axis ratio of $1.85 \pm 0.45$ (Section~\ref{sec:ellgeo}), 
where the major axis is assumed to be along the line of sight.

The point source was found to have a flux density of $0.77 \pm 0.06$ mJy with the best fit model in
\citet{young2014}. In this analysis, we find it to have a flux density of $0.75 \pm 0.08$ mJy. In
\citet{young2014}, substructure is identified that corresponds to an optically-identified subcluster,
which may either be a merging subcluster, or a foreground cluster. A proposal has been accepted for
\emph{XMM-Newton} observations of this substructure (PI: Sarazin).

%\afterpage{
%\clearpage
%\thispagestyle{empty}
%\begin{figure}
%  \centering
%  \includegraphics[width=0.85\textwidth]{analysis_figures/cres/JF_Conf_Intervals_2params_both_default_speedy_9_Feb_2015_m1206.eps}
%  \includegraphics[width=0.85\textwidth]{analysis_figures/cres/PPP_arnaud_v3_log-log_30_Mar_2015_m1206.eps}
%  \caption{MACS 1206}
%  \label{fig:macs_1206params}
%\end{figure}
%\clearpage
%}

%%%%%%%%%%%%%%%%%%%%%%%%%%%%%%%%%%%%%%%%%%%%%%%%%%%%%%%%%%%%%%

\subsection{MACS 0329 (z=0.45)}
\label{sec:results_m0329}

%%%%%%%%%%%%%%%%%%%%%%%%%%%%%%%%%%%%%%%%%%%%%%%%%%%%%%%%%%%%%%

MACS 0329 has a rare distinction of being listed as both a cool core and disturbed cluster. Although it has
been classified as relaxed \citep{schmidt2007}, subtructure has been noted \citep{maughan2008}, and it earns
its cool core and disturbed classifications based on central weighting of X-ray luminsoity and comparing
centroid offsets between optical and X-ray data \citep{sayers2013}. The elongation of the weak lensing and
strong lensing are towards the northwest and southeast of the centroid.

MACS 0329 has two systems with multiple images: one at $z = 6.18$ and the other at $z = 2.17$. The Einstein
radii for these two systems are $r_E = 34$\asecs and $r_E = 28$\asec, respectively \citep{zitrin2012b}, which is
noted as being typical for relaxed, well-concentrated lensing clusters. 


%\afterpage{
%\clearpage
%\thispagestyle{empty}
%\begin{figure}
%  \centering
%  \includegraphics[width=0.85\textwidth]{analysis_figures/cres/JF_Conf_Intervals_2params_both_default_speedy_9_Feb_2015_m0329.eps}
%  \includegraphics[width=0.85\textwidth]{analysis_figures/cres/PPP_arnaud_v3_log-log_23_Feb_2015_m0329.eps}
%  \caption{MACS 0329}
%  \label{fig:macs_0329params}
%\end{figure}
%\clearpage
%}

%%%%%%%%%%%%%%%%%%%%%%%%%%%%%%%%%%%%%%%%%%%%%%%%%%%%%%%%%%%%%%

\subsection{RXJ1347 (z=0.45)}
\label{sec:results_rxj1347}

%%%%%%%%%%%%%%%%%%%%%%%%%%%%%%%%%%%%%%%%%%%%%%%%%%%%%%%%%%%%%%

RXJ1347 is one of the most luminous X-ray clusters, and has been well studied in radio, SZ, lensing, optical
spectroscopy, and X-rays \citep[e.g.][]{schindler1995,allen2002, pointecouteau1999,komatsu2001,kitayama2004,
gitti2007b,ota2008,bradac2008,miranda2008}. X-ray contours have long suggested RXJ1347 is a relaxed system
\citep[e.g.][]{schindler1997}, and it is classified as a cool core cluster \citep[e.g.][]{mann2012,sayers2013}. 

Despite
the classification of being a cool core cluster, it is also observed that there are hot regions, intially
constrained as $kT > 10$ keV \citep[e.g.][]{allen2002,bradac2008}, and more recently constrained to even
hotter temperatures \citep[$kT > 20$ keV][]{johnson2012}, indicative of an unrelaxed cluster. 
\citet{johnson2012} also interpret the two cold fronts as being due to sloshing, where a subscluster has returned
for a second passage.

Indeed, the first sub-arcminute SZ observations \citep{komatsu2001,kitayama2004} saw an enhancement to
the southeast of the cluster X-ray peak, which was suggested as being due to shock heating. This enhancement
was confirmed by MUSTANG \citep{mason2010}. Further measurements were made with CARMA \citep{plagge2013},
which find the 9\% of the thermal energy in the cluster is in sub-arcminute substructure.
At low radio frequencies \citep[][237 MHz and 614 MHz]{ferrari2011},
\citep[][1.4 GHz]{gitti2007a} find evidence for a radio mini-halo in the core of RXJ1347. The cosmic ray electrons
are thought to be reaccelerated because of the shock and sloshing in the cluster \citep{ferrari2011}.

We observe a point source with flux density of $7.40 \pm 0.58$ mJy. Previous analysis of the MUSTANG data put
the point source flux density at 5 mJy \citep{mason2010}. The difference in the flux densities is likely accounted
in (1) the different modeling of point sources; primarily that we filter the double Gaussian, (2) we simultaneously
fit the components, and (3) we almost certainly have a steeper profile in the core than the beta model assumed in
\citet{mason2010}. Lower frequency radio observations found the
flux density of the source to be $10.81 \pm 0.19$ mJy at 28.5 GHz \citep{reese2002}, and $47.6 \pm 1.9$
mJy at 1.4 GHz \citep{condon1998}.

RXJ 1347 is observed to have a UV excess in its BCG \citep{donahue2014}. 

%\afterpage{
%\clearpage
%\thispagestyle{empty}
%\begin{figure}
%  \centering
%  \includegraphics[width=0.85\textwidth]{analysis_figures/cres/JF_Conf_Intervals_2params_both_default_speedy_9_Feb_2015_rxj1347.eps}
%  \includegraphics[width=0.85\textwidth]{analysis_figures/cres/PPP_arnaud_v3_log-log_3_Feb_2015_rxj1347.eps}
%  \caption{RXJ1347}
%  \label{fig:rxj1347params}
%\end{figure}
%\clearpage
%}

%%%%%%%%%%%%%%%%%%%%%%%%%%%%%%%%%%%%%%%%%%%%%%%%%%%%%%%%%%%%%%

\subsection{MACS 1311 (z=0.49)}
\label{sec:results_m1311}

%%%%%%%%%%%%%%%%%%%%%%%%%%%%%%%%%%%%%%%%%%%%%%%%%%%%%%%%%%%%%%


MACS 1311 is listed as a cool core cluster \citep[e.g.][]{sayers2013}, and appears to have quite circular
contours in the X-ray and lensing images, yet has evidence for some disturbance, given its classification
in \citet{mann2012}. However, the SZ contours from Bolocam show some enhancement  to the west, and has
a notable centroid shift ($27.7$\asec) westward from the X-ray centroid. When fitting pressure profiles
to this cluster, it appears that the enhanced SZ pressure at moderate radii ($r \sim 100$\asec) is due
to this enhancement, especially when noting that we do use the X-ray centroid. Adopting the Bolocam centroid
does not change the pressure profile much, and we still observe a pressure enhancement at moderate radii.

%\afterpage{
%\clearpage
%\thispagestyle{empty}
%\begin{figure}
%  \centering
%  \includegraphics[width=0.85\textwidth]{analysis_figures/cres/JF_Conf_Intervals_2params_both_default_speedy_3_May_2015_m1311.eps}
%  \includegraphics[width=0.85\textwidth]{analysis_figures/cres/PPP_arnaud_v3_log-log_26_Apr_2015_m1311.eps}
%  \caption{MACS 1311}
%  \label{fig:macs_1311params}
%\end{figure}
%\clearpage
%}

%%%%%%%%%%%%%%%%%%%%%%%%%%%%%%%%%%%%%%%%%%%%%%%%%%%%%%%%%%%%%%

\subsection{MACS 1423 (z=0.54)}
\label{sec:results_m1423}

%%%%%%%%%%%%%%%%%%%%%%%%%%%%%%%%%%%%%%%%%%%%%%%%%%%%%%%%%%%%%%

MACS 1423 is a cool core cluster \citep{mann2012,sayers2013}. While the Bolocam (SZ) contours are quite concentric,
and suggestive of a relaxed cluster, the centroid is still offset from the X-ray peak by an appreciable angle 
($19.8$\asec). Similarly to MACS 1311, the pressure is slightly less than the ACCEPT X-ray derived pressure in the
core, and slightly greater at moderate radii. While this is expected for a centroid offset, we find that adopting
the Bolocam centroid again yields no substantial difference in the SZ pressure profile.
We observe a point source with flux density of $1.36 \pm 0.13$ mJy

%\afterpage{
%\clearpage
%\thispagestyle{empty}
%\begin{figure}
%  \centering
%  \includegraphics[width=0.85\textwidth]{analysis_figures/cres/JF_Conf_Intervals_2params_both_default_speedy_9_Feb_2015_m1423.eps}
%  \includegraphics[width=0.85\textwidth]{analysis_figures/cres/PPP_arnaud_v3_log-log_24_Feb_2015_m1423.eps}
%  \caption{MACS 1423}
%  \label{fig:macs_1423params}
%\end{figure}
%\clearpage
%}

%%%%%%%%%%%%%%%%%%%%%%%%%%%%%%%%%%%%%%%%%%%%%%%%%%%%%%%%%%%%%%

\subsection{MACS 1149 (z=0.54)}
\label{sec:results_m1149}

%%%%%%%%%%%%%%%%%%%%%%%%%%%%%%%%%%%%%%%%%%%%%%%%%%%%%%%%%%%%%%

MACS 1149 is classified as a disturbed cluster \citep[e.g.][]{mann2012,sayers2013}, and lensing studies have found
that a single DM halo does not describe the cluster well, but rather at least four large-scale DM hales are used to
describe the cluster \citep{smith2009}. A large radial velocity dispersion \citep[1800 km s$^{-1}$][]{ebeling2007} is 
observed, indicative of merger activity along the line of sight. X-ray, SZ, and lensing (particularly 
strong lensing) all show elongation in the northwest-southeast direction. We see a $3\sigma$ feature to the east of
the centroids, but it is not clear that this is associated with any particular feature.

The SZ derived pressure profile roughly matches the shape of the X-ray derived pressure profile (Figure
\ref{fig:CI_all}), with the SZ pressure consistently greater than the X-ray pressure.We calculate
that the axis along the line of sight is $2.08 \pm 0.52$ (Section~\ref{sec:ellgeo}) times greater than the axes in the plane
of the sky. Although we do not find previous analysis of the elongation in the plane of the sky, we would certainly
expect this given (1) the inferred merger activity along the line of sight, and (2) the lensing strength of the cluster.

%\afterpage{
%\clearpage
%\thispagestyle{empty}
%\begin{figure}
%  \centering
%  \includegraphics[width=0.85\textwidth]{analysis_figures/cres/JF_Conf_Intervals_2params_both_default_speedy_9_Feb_2015_m1149.eps}
%  \includegraphics[width=0.85\textwidth]{analysis_figures/cres/PPP_arnaud_v3_log-log_22_Jan_2015_m1149.eps}
%  \caption{MACS 1149}
%  \label{fig:macs_1149params}
%\end{figure}
%\clearpage
%}

%%%%%%%%%%%%%%%%%%%%%%%%%%%%%%%%%%%%%%%%%%%%%%%%%%%%%%%%%%%%%%

\subsection{MACS 0717 (z=0.55)}
\label{sec:results_m0717}

%%%%%%%%%%%%%%%%%%%%%%%%%%%%%%%%%%%%%%%%%%%%%%%%%%%%%%%%%%%%%%

%\begin{figure}
%  \centering
%  \includegraphics[width=0.85\textwidth]{analysis_figures/M0717_mroczkowski_fig1.eps}
%  \caption{From \citet{mroczkowski2012}.}
%  \label{fig:m0717_mroczkowski}
%\end{figure}

Despite MACS 1149's impressive merging activity, MACS 0717 is touted as the most disturbed massive cluster at $z> 0.5$
\citep{ebeling2007}, which appears to be accreting matter along a 6-Mpc-long filament \citep{ebeling2004}, and has the
largest known Einstein radius \citep[$\theta_e \sim 55$\asec;][]{zitrin2009}. Four distinct components are identified
from X-ray and optical analyses \citep{ma2009}, and the lensing analyses \citep{zitrin2009,limousin2012} find agreement
in the location of these four mass peaks with those from the X-ray and optical. 

%%% Rework (subclusters, but without figure...?)
%The four components are labelled in Figure~\ref{fig:m0717_mroczkowski}. 
There are four identified subclusters \citep[labeled A through D][]{mroczkowski2012}. \citet{ma2009} find that subcluster C is the
most massive component, while subcluster A is the least massive, and subclusters B and D are likely remnant cores. The
velocities of the components from spectroscopy are found to be $(v_A, v_B, v_C, v_D) = (+278_{-339}^{+295},+3238_{-242}^{+252},
-733_{-478}^{+486},+831_{-800}^{+843})$ km s$^{-1}$ \citep{ma2009}. 

MACS 0717 has also been observed at 610 MHz with the Giant Metrewave Radio Telescope (GMRT) which reveals both a radio
halo and a radio relic \citep{vanweeren2009}. This is interpreted as likely being due to a diffuse shock acceleration
(DSA).

We observe a foreground radio galaxy, modeled as a point source here, with flux density of $2.08 \pm 0.25$ mJy at 90 GHz. 
This was previously reported with an integrated flux density of $2.8 \pm 0.2$ mJy and an extended shape 14.\asec4 $\times$ 16.\asec1 
\citep{mroczkowski2012}. However, an improved beam modeling has allowed us to model the foreground galaxy given a known beam shape. 
It is also worth
noting that the MUSTANG data itself has been processed slightly different from that presented in \citet{mroczkowski2012};
in this work the map is produced with a common calculated as the mean across detectors.

%\afterpage{
%\clearpage
%\thispagestyle{empty}
%\begin{figure}
%  \centering
%  \includegraphics[width=0.85\textwidth]{analysis_figures/cres/JF_Conf_Intervals_2params_both_default_speedy_9_Feb_2015_m0717.eps}
%  \includegraphics[width=0.85\textwidth]{analysis_figures/cres/PPP_arnaud_v3_log-log_12_Mar_2015_m0717.eps}
%  \caption{MACS 0717}
%  \label{fig:macs_0717params}
%\end{figure}
%\clearpage
%}

%%%%%%%%%%%%%%%%%%%%%%%%%%%%%%%%%%%%%%%%%%%%%%%%%%%%%%%%%%%%%%

\subsection{MACS 0647 (z=0.59)}
\label{sec:results_m0647}

%%%%%%%%%%%%%%%%%%%%%%%%%%%%%%%%%%%%%%%%%%%%%%%%%%%%%%%%%%%%%%

MACS 0647 is at $z = 0.591$ and is classified as neither a cool core nor a disturbed cluster \citep{sayers2013}. 
It was included in the CLASH sample due to its strong lensing properties \citep{postman2012}.
Gravitational lensing \citep{zitrin2011}, X-ray surface brightness \citep{mann2012}, 
and SZ effect (MUSTANG, see Figure \ref{fig:mustang_maps_2_sample}, and Bolocam) maps all
show elongation in an east-west direction. 
In the joint analysis presented here, we see that the spherical model provides an adequate fit to both datasets and we note 
that the spherical assumption allows for a easier interpretation of the mass profile of the cluster.

%%%%%%%%%%%%%%%%%%%%%%%%%%%%%%%%%%%%%%%%%%%%%%%%%%%%%%%%%%%%%%

\subsection{MACS 0744 (z=0.70)}
\label{sec:results_m0744}

%%%%%%%%%%%%%%%%%%%%%%%%%%%%%%%%%%%%%%%%%%%%%%%%%%%%%%%%%%%%%%

MACS 0744 is neither classified as a cool core cluster nor a disturbed cluster \citep{mann2012,sayers2013}, but qualifies
as a relaxed cluster \citep{mann2012}. There is a dense X-ray core, and a doubly peaked red sequence of galaxies as found
by \citet{kartaltepe2008}. The gas is also found to be rather hot: $k_B T_e = 17.9_{-3.4}^{+10.8}$ keV, as determined by combining
SZ and X-ray data \citep{laroque2003}. 

The data presented here is the same as in \citet{korngut2011}, but has been processed differently: again, the primary difference
is in the treatment of the common mode. Additionally, \citet{korngut2011} optimize over the low-pass filtering of the common mode
and do not implement a correction factor for the SNR map. The surface brightness significance of the shock feature is the same, 
but is perhaps less
bowed than the kidney bean shape seen previously.  The excess in \citet{korngut2011} was an exciting results for MUSTANG, 
as it marked the first clear detection of a shock in the SZ that had not been previously been known from X-ray observations. 
\citet{korngut2011}
reanalyze the X-ray data with the knowledge of the shocked region from MUSTANG, and calculate the Mach number of the shock
based on (1) the shock density jump, (2) stagnation condition between the pressures at the edge of the cold front and just
ahead of the shock, and (3) temperature jump across the shock, and find Mach numbers between 1.2 and 2.1, with
a velocity of $1827_{-195}^{+267}$ km s$^{-1}$.

%\afterpage{
%\clearpage
%\thispagestyle{empty}
%\begin{figure}
%  \centering
%  \includegraphics[width=0.85\textwidth]{analysis_figures/cres/JF_Conf_Intervals_2params_both_default_speedy_9_Feb_2015_m0744.eps}
%  \includegraphics[width=0.85\textwidth]{analysis_figures/cres/PPP_arnaud_v3_log-log_26_Feb_2015_m0744.eps}
%  \caption{MACS 0744}
%  \label{fig:macs_0744params}
%\end{figure}
%\clearpage
%}

%%%%%%%%%%%%%%%%%%%%%%%%%%%%%%%%%%%%%%%%%%%%%%%%%%%%%%%%%%%%%%

%%%%%%%%%%%%%%%%%%%%%%%%%%%%%%%%%%%%%%%%%%%%%%%%%%%%%%%%%%%%%%

\subsection{CLJ 1226 (z=0.89)}
\label{sec:results_clj1226}

%%%%%%%%%%%%%%%%%%%%%%%%%%%%%%%%%%%%%%%%%%%%%%%%%%%%%%%%%%%%%%

CLJ 1226 is a well studied high redshift cluster \citep[e.g.][]{mroczkowski2009,bulbul2010,adam2015}. 
\citet{adam2015} find a point source at RA 12:12:00.01 and Dec +33:32:42 with a flux density of 
$6.8 \pm 0.7 \text{ (stat.)} \pm 1.0 \text{ (cal.)}$ mJy at 260 GHz and $1.9 \pm 0.2 \text{ (stat.)}$ at 150 GHz. 
This is not the same point source seen in \citet{korngut2011}, which is reported as a point source
with $4.6\sigma$ significance in surface brightness. In our current analysis, we do not clearly observe a point
source at either location, but we do model the point source found in \citet{adam2015} and fit a flux density of
$0.36 \pm 0.11$ mJy.

In the previous analysis of the MUSTANG data, \citet{korngut2011} find a ridge of significant substructure after 
subtracting a bulk SZ profile (N07, fitted to SZA data). They find that this ridge, southwest of the cluster
center, alongside X-ray profiles, are consistent with a proposed merger scenario.

%\afterpage{
%\clearpage
%\thispagestyle{empty}
%\begin{figure}
%  \centering
%  \includegraphics[width=0.85\textwidth]{analysis_figures/cres/JF_Conf_Intervals_2params_both_default_speedy_9_Feb_2015_clj1226.eps}
%  \includegraphics[width=0.85\textwidth]{analysis_figures/cres/PPP_arnaud_v3_log-log_4_Dec_2014_clj1226.eps}
%  \caption{CLJ 1226}
%  \label{fig:clj_1226params}
%\end{figure}
%\clearpage
%}
\begin{appendix}

\begin{deluxetable*}{l|lllllllllll}
\tabletypesize{\footnotesize}
\tablecolumns{12}
\tablewidth{\columnwidth} 
\tablecaption{Summary of Fitted Pressure Profiles \label{tbl:pressure_profile_results}}
\tablehead{
\colhead{Cluster} & \colhead{$R_{500}^a$} & \colhead{$Y_{sph}(R_{500})$} & \colhead{$P_{500}^a$} & 
        \colhead{$P_0$} & \colhead{$C_{500}$} & \colhead{$\alpha$} & \colhead{$\beta$} & \colhead{$\gamma$} & 
                  \colhead{$k$} & \colhead{$\tilde{\chi}^2$} &  \colhead{d.o.f.}        \\ 
      & \colhead{(Mpc)} & \colhead{($10^{-5}$ Mpc$^2$)} & \colhead{$10^{-3}$ keV cm$^{-3}$} & \colhead{}  & 
      \colhead{} & \colhead{}  & \colhead{}  & \colhead{}  &   \colhead{}    
}
\startdata
Abell 1835 & 1.49 & $22.50_{-4.49}^{+4.12}$ &  5.94 & $2.15 \pm 0.07$ & $0.77_{-0.17}^{+0.23}$ & 
1.05 & 5.49 & $0.78_{-0.13}^{+0.12}$ & 1.08 & 0.99 & 12880   \\ 
Abell 611  & 1.24 &  $8.14_{-2.21}^{+3.68}$ &  4.45 & $35.43 \pm 2.46$ & $2.00_{-0.30}^{+0.40}$ & 
1.05 & 5.49 & $0.00^{+0.15}$ & 0.96 & 1.02 & 12882   \\ 
MACS 1115  & 1.28 & $20.53_{-3.52}^{+3.84}$ &  5.45 & $0.67 \pm 0.04$ & $0.35_{-0.10}^{+0.15}$ & 
1.05 & 5.49 & $0.87_{-0.27}^{+0.18}$ & 1.11 & 1.04 & 12875   \\ 
MACS 0429  & 1.10 & $19.85_{-3.74}^{+4.00}$ &  4.48 & $11.01 \pm 0.77$ & $0.59_{-0.09}^{+0.11}$ & 
1.05 & 5.49 & $0.00^{+0.15}$ & 1.00 & 1.03 & 12875   \\ 
MACS 1206  & 1.61 & $43.24_{-8.27}^{+8.19}$ & 10.59 & $2.39 \pm 0.10$ & $0.74_{-0.14}^{+0.16}$ & 
1.05 & 5.49 & $0.51_{-0.16}^{+0.14}$ & 1.09 & 1.01 & 12874   \\ 
MACS 0329  & 1.19 & $12.91_{-2.37}^{+2.93}$ &  5.93 & $9.30 \pm 0.50$ & $1.18_{-0.28}^{+0.72}$ & 
1.05 & 5.49 & $0.41_{-0.41}^{+0.19}$ & 1.03 & 0.99 & 12876   \\ 
RXJ1347    & 1.67 & $37.69_{-5.11}^{+5.78}$ & 11.71 & $3.24 \pm 0.08$ & $1.18_{-0.48}^{+1.02}$ & 
1.05 & 5.49 & $0.80_{-0.70}^{+0.30}$ & 1.15 & 0.99 & 12880   \\ 
MACS 1311  & 0.93 & $10.16_{-1.73}^{+1.79}$ &  3.99 & $2.75 \pm 0.22$ & $0.35_{-0.05}^{+0.15}$ & 
1.05 & 5.49 & $0.41_{-0.41}^{+0.34}$ & 0.98 & 1.00 & 12881   \\ 
MACS 1423  & 1.09 &  $8.47_{-2.07}^{+2.53}$ &  6.12 & $22.39 \pm 1.71$ & $1.58_{-0.48}^{+0.22}$ & 
1.05 & 5.49 & $0.00^{+0.35}$ & 1.04 & 0.98 & 12876   \\ 
MACS 1149  & 1.53 & $42.77_{-5.67}^{+4.99}$ & 12.28 & $5.50 \pm 0.25$ & $0.83_{-0.03}^{+0.07}$ & 
1.05 & 5.49 & $0.00^{+0.05}$ & 0.87 & 1.00 & 13584   \\ 
MACS 0717  & 1.69 & $43.44_{-8.00}^{+9.28}$ & 14.90 & $21.28 \pm 0.68$ & $1.97_{-0.37}^{+0.53}$ & 
1.05 & 5.49 & $0.00^{+0.25}$ & 0.48 & 1.04 & 12876   \\ 
MACS 0647  & 1.26 & $26.22_{-4.72}^{+5.37}$ &  9.23 & $2.78 \pm 0.11$  & $0.70_{-0.20}^{+0.30}$ & 
1.05 & 5.49 & $0.60_{-0.20}^{+0.15}$ & 1.14 & 1.01 & 12876   \\ 
MACS 0744  & 1.26 & $12.59_{-2.29}^{+3.18}$ & 11.99 & $13.15 \pm 0.81$ & $1.71_{-0.21}^{+0.29}$ & 
1.05 & 5.49 & $0.00^{+0.15}$ & 0.90 & 1.02 & 12875   \\ 
CLJ1226    & 1.00 &  $9.03_{-1.60}^{+2.03}$ & 11.84 & $19.29 \pm 1.25$ & $1.90_{-0.50}^{+0.60}$ & 
1.05 & 5.49 & $0.29_{-0.29}^{+0.36}$ & 0.92 & 1.03 & 12875   \\ 
\hline
All          &  --    &  --    &  --    &  $7.94  \pm 0.10$ & $1.3_{-0.1}^{+0.1}$ & 1.05 & 5.49 & $0.3_{-0.1}^{+0.1}$ & -- & -- & -- \\ 
Cool Core    &  --    &  --    &  --    &  $3.55  \pm 0.06$ & $0.9_{-0.1}^{+0.1}$ & 1.05 & 5.49 & $0.6_{-0.1}^{+0.1}$ & -- & -- & -- \\
Disturbed    &  --    &  --    &  --    &  $12.56 \pm 0.29$ & $1.5_{-0.2}^{+0.1}$ & 1.05 & 5.49 & $0.0^{+0.1}$       & -- & -- & -- \\ 
Well behaved &  --    &  --    &  --    &  $5.34 \pm 0.08$ & $1.2_{-0.1}^{+0.1}$ & 1.05 & 5.49 & $0.5_{-0.1}^{+0.1}$  & -- & -- & -- \\ 
\hline
All (A10)    &  --    &  --    &  --    &  $8.403 h_{70}^{-3/2}$ & 1.177 & 1.0510 & 5.4905 & 0.3081 & -- & -- & -- \\
Cool core (A10) &  --    &  --    &  --    &  $3.249 h_{70}^{-3/2}$ & 1.128 & 1.2223 & 5.4905 & 0.7736 & -- & -- & -- \\
Disturbed (A10) &  --    &  --    &  --    &  $3.202 h_{70}^{-3/2}$ & 1.083 & 1.4063 & 5.4905 & 0.3798 & -- & -- & --
\enddata
\tablecomments{Results from our pressure profile analysis. $Y_{sph}$ is calculated using the tabulated value of $R_{500}$.
    $^a$Values of $R_{500}$ and $P_{500}$ are taken from \citet{sayers2013}. We have assumed A10 values of $\alpha$ and $\beta$.
    The findings from A10 are reproduced in the last three rows. The $h_{70}$ dependence is included for explicit replication
    of A10 results; all $P_0$ values have this dependence (the assumed cosmologies are the same). Well behaved clusters are
    identified in the next section.}
\end{deluxetable*}
